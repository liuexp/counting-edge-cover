\section{Preliminaries}
%\subsection{Definitions}
An edge cover of a graph is a set of edges such that every vertex has at least one adjacent edge in it.
Given a graph $G=(V,E)$ with $e \in E$,  we use $EC(G)$ to denote the set of all edge covers of graph $G$, and $P(G, e)$ to denote the marginal probability over $EC(G)$ that edge $e$ is \emph{not} chosen, or formally, with $X \sim EC(G)$ uniformly,
\begin{equation}
	P(G, e) \triangleq \mathbb{P} \left(\textrm{edge $e$ is not chosen in $X$} \right)
	\label{defpge}
\end{equation}

In this paper, we deal with an extended notion of undirected graphs where dangling edges and free edges are allowed.
\begin{Definition}
	A {\bf dangling edge} $e=(u,\_)$ of a graph is such singleton edge with exactly one end-point vertex $u$, as shown in the Figure \ref{fig:G}.

	A {\bf free edge} $e=(\_, \_)$ of a graph is such edge with no end-point vertex. %Note that a free edge is not a dangling edge.


\end{Definition}

	We use graph to refer graph with or without dangling edges and free edges.
	Edges in the usual sense (i.e. neither dangling nor free), will be referred to as normal edges.
	%and graphs in the usual sense(i.e. graphs with only normal edges) will be refered to as normal graphs.

	We remark that an alternative view to these combinatorial definitions is from Rtw-Mon-CNF.
	A dangling edge is simply a variable which only appears at one clause, and a free edge is a variable
	that does not appear at all, whereas normal edge just corresponds to variables appearing twice.

For a graph $G=(V,E)$, an edge $e = (u,v) \in E$ and a vertex $u \in V$, define
\begin{align*}
    G - e \triangleq& (V, E-e) \\
    e - u \triangleq& (\_, v) \text{(note that here $v$ could be $\_$)} \\
    G - u \triangleq& (V - u, \\
                    &\set{e: e \in E, e\text{ is not incident with }u} \\
                    &\cup \set{e - u: e \in E, e\text{ is incident with }u})
\end{align*}

%For a graph $G=(V,E)$ with an edge (may be dangling or free) $e \in E$ and a vertex $v \in V$,
%$$G-e \triangleq (V, E-e)$$
%and
%\begin{align*}
%G-v \triangleq & (V-\set{v}, \set{(x,y):x \in V-\set{v}, y \in V-\set{v}, (x,y) \in E}  \\
% &\cup \set{(x,\_):x \in V-\set{v}, (v,x) \in E}  \\
% &\cup \set{(\_,\_): (v,\_) \in E \text{ or } (v,v) \in E})
%\end{align*}

Note that here in edge set $E$, duplicates are allowed. We may have multiple dangling edges $(v,\_)$ and many free edges $(\_,\_)$. Recall that here edges are unordered pairs so we treat $(v,\_)$ and $(\_,v)$ as the same.

For example, given a degree-3 vertex $u$ with dangling edge $e$ shown in Figure \ref{fig:G} , the result of $e_1 - u$ is shown in Figure \ref{fig:e-u} and the result of $G-e-u\triangleq (G-e)-u$ is shown in Figure \ref{fig:G-e-u}.

\begin{figure}[htp]
	\begin{subfigure}[b]{0.15\textwidth}
		\centering
		\setlength{\unitlength}{0.8mm}
		\begin{picture}(20,20)
			\put(0,0){\circle*{6}}
			\put(0,0){\line(1,1){10}}
			\put(20,0){\circle*{6}}
			\put(20,0){\line(-1,1){10}}
			\put(10,10){\circle*{6}}
			\put(10,10){\line(0,1){10}}
			\put(14,10){$u$}
			\put(8,15){$e$}
			\put(2,6){$e_1$}
			\put(15,5){$e_2$}
		\end{picture}
		\caption{$G$}
		\label{fig:G}
	\end{subfigure}
	\hfill
    \begin{subfigure}[b]{0.15\textwidth}
		\centering
		\setlength{\unitlength}{0.8mm}
		\begin{picture}(20,20)
			\put(0,0){\circle*{6}}
			\put(0,0){\line(0,1){10}}
			\put(20,0){\circle*{6}}
			\put(20,0){\line(-1,1){10}}
			\put(10,10){\circle*{6}}
			\put(10,10){\line(0,1){10}}
			\put(14,10){$u$}
			\put(8,15){$e$}
			\put(2,6){$e_1$}
			\put(15,5){$e_2$}
		\end{picture}
		\caption{$e_1-u$}
		\label{fig:e-u}
	\end{subfigure}
    \hfill
	\begin{subfigure}[b]{0.15\textwidth}
		\centering
		\setlength{\unitlength}{0.8mm}
		\begin{picture}(20,20)
			\put(0,0){\circle*{6}}
			\put(0,0){\line(0,1){10}}
			\put(20,0){\circle*{6}}
			\put(20,0){\line(0,1){10}}
			\put(2,6){$e_1$}
			\put(15,6){$e_2$}
		\end{picture}
		\caption{$G-e-u$}
		\label{fig:G-e-u}
	\end{subfigure}
	\caption{Dangling edges examples.}
\end{figure}

We use $0$ to denote scalar value $0$, and $\mathbf{0}$ to denote the all-zero vector, and $\set{e_i}_{i=1}^{d}$ denote the $d$-dimensional vector with $i$-th coordinate being $e_i$, so $\set{e_i} = \mathbf{0}$ means $\forall i, e_i = 0$.
We also use the convention that when $d=0, \prod_i^d p_i \triangleq 1$.

In general we use $n$ to refer to the number of vertices in a given graph, and $m$ to refer to the number of edges.
