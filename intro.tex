\section{Introduction}
An edge cover of a graph is a set of edges such that every vertex has at least an adjacent edge in it. For a given input graph, we count the number of edge covers for that graph. This is a \#P-complete problem even when we restrict the input to 3 regular graphs. In this paper, we study the approximation version. For any given parameter $\epsilon>0$, the algorithm output a number $\hat{N}$ such that $(1-\epsilon) N\leq \hat{N} \leq (1+\epsilon) N$, where $N$ is the accurate number of edge covers of the input graph. We also require that the running time of the algorithm is bounded by $Poly(n,1/\epsilon)$, where $n$ is the number of vertices of the given graph. This is called a fully polynomial-time approximation scheme (FPTAS). Our main result of this paper is a FPTAS for edge covers for any graph. Previously, approximation algorithm is only known for 3 regular graphs and the algorithm is randomized~\cite{MFCS09}. The randomized version of FPTAS is called FPRAS, which uses random bits in the algorithms and we require that the final output is within the range $[(1-\epsilon) N, (1+\epsilon) N]$ with high probability.

Edge cover is related to many other graph problems such as (perfect) matching, and $k$-factor problems. All of them are talking about a set of edges which satisfies some local constraint defined on each vertex. For edge cover, it says that at least one incident edge should be chosen; while for matching, it is at most one edge. For generic constraints, it is the Holant framework, which is well studied in terms of exactly counting, recently in approximate counting. For counting matchings, there is a FPRAS based on MCMC and deterministic FPTAS is only known for graphs with bounded degree. For counting perfect matchings, it is a long standing open question if there is a FPRAS or FPTAS for it. For bipartite graphs, there is a FPRAS. The weighted version is exactly computing permanent of a non-negative matrix. This is one great achievement. It is still widely open if there exists a FPTAS for it. The current best deterministic algorithm can only approximate the permanent with an exponential large factor. There are many other counting problems, where there is a FPRAS and we do not know if there is a FPTAS or not such as counting the number of solution for a DNF formula~\cite{}. In this paper, we give a complete FPTAS for a problem, where even FPRAS is only known for very special family of graphs.

Another view point of edge cover is read twice monotone CNF formula (Rtw-Mon-CNF): each edge is viewed as a boolean variable and it is connected with two vertices (read twice); the constraint on each vertex is exactly a monotone CNF constrain as at least one edge variable is assigned to be True. Counting number of solutions for a Boolean formula is another set of interesting problem studied both in exact counting and approximate counting. One famous example is the FPRAS for counting the solutions for a  DNF formula. It is important open question if we can derandomized it. Our FPTAS for counting edge covers can also be viewed as a FPTAS for counting the solutions for a  Rtw-Mon-CNF formula. If we do not restrict that each variable appears in at most two constraints, There is no FPTAS or FPRAS unless NP is equal to P or RP.  

The common overall approach for designing approximate counting algorithms is to relate counting with probability distribution. 
 This is usually referred as ``counting vs sampling" paradigm when one mainly focuses on randomized counting.  If we can compute (estimate) the marginal probability, which in our problem is the probability of a given edge is chosen when we sample a edge cover uniformly at random, we can in turn to approximate count. In randomized FPRAS, we estimate the marginal probability by sampling, and the most successful approach is sampling by Markov chain.
In FPTAS, one calculate the marginal probability directly, and the most successful approach is correlation decay as introduce in \cite{BG08} and \cite{Weitz06}. We elaborate a bit on the ideas. 
 The marginal probability is estimated using only a local neighborhood around the edge. To justify the precision of the estimation, we show that far-away edges have little influence on the marginal distribution.
One most successful example is in anti-ferromagnetic two-spin system, including counting independent set. 
The correlation decay based FPTAS is beyond the best known MCMC based FPRAS and approaches the boundary of tractable and intractable. 
To the best of our knowledge, that was the only example for which the best tractable range for correlation decay based FPTAS exceeds the sampling based FPRAS. This paper offers another such example. FPRAS was \emph{the} solution concept for approximate counting, the recently development of correlation decay based FPTAS is changing the picture. It is interesting question to establish more deep relation between these two approaches.

A set of tools was developed for establishing correlation decay property. These are something like coupling argument, canonic path and so on for establish to rapid mixing for Markov Chains.
There are Self avoid walk tree, computational tree, potential function, bounded variables and so on. Armed with these powerful tools, there are recently many FPTAS were designed for many counting problems. 
Many of these techniques are also used in designing and analyzing the FPRAS for counting edge covers.  

Usually, the correlation decay property only implies FPTAS for system with bounded degree such as~\cite{}. The reason is that
we need to explore a local neighborhood with radius of order $\log n$, then the total running time is sup polynomial $n^{\log n}$ if there is no degree bound. To overcome this, we make use of stronger notion called computationally efficient correlation decay
as introduced in~\cite{LLY12}. The observation is that we will go through a vertex with sup constant degree, the error is also decreased by a supper constant. Thus we do not need to explore a depth of $\log n$ if the degrees are large. The tradeoff relation between degree and decay rate defined by  computationally efficient correlation decay can support FPTAS with unbounded degree system. Previously, this notation is only used in anti-ferromagnetic two-spin system. In this paper, we prove that the distribution defined by edge covers also satisfies this stronger version of correlation decay and thus we give FPTAS for counting edge covers for any graph.  