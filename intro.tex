\section{Introduction}
An edge cover of a graph is a set of edges such that every vertex has at least an adjacent edge in it.
 For any graph without isolated vertices, there is at least one edge cover: the set of all edges. So the decision problem is trivial. There is also a polynomial time algorithm based on maximum machining to compute a edge cover with minimum cardinality.     In this paper, we study the counting version:
 For a given input graph, we count the number of edge covers for that graph. Unlike the decision or optimization problem, counting edge covers is a \#P-complete problem even when we restrict the input to 3 regular graphs. In this paper, we study the approximation version. For any given parameter $\epsilon>0$, the algorithm outputs a number $\hat{N}$ such that $(1-\epsilon) N\leq \hat{N} \leq (1+\epsilon) N$, where $N$ is the accurate number of edge covers of the input graph. We also require that the running time of the algorithm is bounded by $poly(n,1/\epsilon)$, where $n$ is the number of vertices of the given graph. This is called a fully polynomial-time approximation scheme (FPTAS). Our main result of this paper is an FPTAS for counting edge covers for any graph. Previously, approximation algorithm was only known for 3 regular graphs and the algorithm is randomized~\cite{MFCS09}. The randomized version of FPTAS is called fully polynomial-time randomized approximation scheme (FPRAS), which uses random bits in the algorithm and requires that the final output is within the range $[(1-\epsilon) N, (1+\epsilon) N]$ with high probability.

Edge cover is related to many other graph problems such as (perfect) matching, $k$-factor problems and so on. All of them are talking about a set of edges which satisfies some local constraints defined on each vertex. For edge cover, it says that at least one incident edge should be chosen; while for matching, it is at most one edge. For generic constraints, it is the Holant framework~\cite{STOC09,holant}, which is well studied in terms of exactly counting~\cite{holant,HuangL12,CaiGW13}, and recently in approximate counting~\cite{YZ13,McQuillan2013,fibo-approx}. For counting matchings, there is an FPRAS based on Markov Chain Monte Carlo (MCMC) for any graph~\cite{jerrum1989approximating}. Deterministic FPTAS is only known for graphs with bounded degree~\cite{BGKNT07}. For counting perfect matchings, it is a long standing open question if there is an FPRAS (or FPTAS) for it. For bipartite graphs, there is an FPRAS for counting perfect matchings. The weighted version can be viewed as  computing permanent of a non-negative matrix~\cite{app_JSV04}. This is one great achievement of approximate counting. It is still widely open if there exists an FPTAS for it or not. The current best deterministic algorithm can only approximate the permanent with an exponential large factor~\cite{linial1998deterministic,gamarnik2010deterministic}. There are many other counting problems, for which there is an FPRAS and we do not know if there is an FPTAS or not. In this paper, we give a complete FPTAS for a problem, for which even FPRAS was only known for very special family of instances.

Another view point of the edge cover problem is read twice monotone CNF formula (Rtw-Mon-CNF): Each edge is viewed as a Boolean variable and it is connected with two vertices (read twice); the constraint on each vertex is exactly a monotone OR constrain as at least one edge variable is assigned to be True. Counting number of solutions for a Boolean formula is another set of interesting problems studied both in exact counting and approximate counting. One famous example is the FPRAS for counting the solutions for a  DNF formula~\cite{KarpL83,KarpLM89}. It is an important open question to derandomize it~\cite{Trevisan04,gopalan2012dnf}. Our FPTAS for counting edge covers can also be viewed as an FPTAS for counting solutions for a  Rtw-Mon-CNF formula. If we do not restrict that each variable appears in at most two constraints, there is no FPTAS or FPRAS unless NP is equal to P or RP~\cite{dicho_DGJ10}.

The common overall approach for designing approximate counting algorithms is to relate counting with probability distribution.
 This is usually referred as ``counting vs sampling" paradigm when one mainly focuses on randomized counting.  If we can compute (or estimate) the marginal probability, which in our problem is the probability of a given edge is chosen when we sample a edge cover uniformly at random, we can in turn to approximate count. In randomized FPRAS, one estimate the marginal probability by sampling, and the most successful approach is sampling by Markov chain~\cite{MC_JA96}.
In deterministic FPTAS, one calculate the marginal probability directly, and the most successful approach is correlation decay as introduce in \cite{BG08} and \cite{Weitz06}. We elaborate a bit on the ideas.
 The marginal probability is estimated using only a local neighborhood around the edge. To justify the precision of the estimation, we show that far-away edges have little influence on the marginal probability.
One most successful example is in anti-ferromagnetic two-spin systems~\cite{LLY12,SST,LLY13}, including counting independent sets~\cite{Weitz06}. The correlation decay based FPTAS is beyond the best known MCMC based FPRAS and achieves the boundary of approximability~\cite{SS12,galanis2012inapproximability}.
To the best of our knowledge, that was the only example for which the best tractable range for correlation decay based FPTAS exceeds the sampling based FPRAS. This paper provides another such example. FPRAS was \emph{the} solution concept for approximate counting~\cite{dich_DGGJ00}. The recently development of correlation decay based FPTAS is changing the picture. It is interesting to investigate the deep relation between these two approaches.

A set of tools was developed for establishing correlation decay property. These are something like coupling argument, canonic path and so on for establish to the rapid mixing property for Markov Chains~\cite{MC_JA96}.
They are self-avoid walk tree, computation tree, potential function, dangling instance, bounded variables and so on. Armed with these powerful tools, there are recently many FPTAS designed for many counting problems~\cite{LLY12,SST,LLY13,YZ13,fibo-approx}.
Many of these techniques are also used in this paper for designing and analyzing the FPTAS for counting edge covers.

Usually, the correlation decay property only implies FPTAS for system with bounded degree. The reason is that
we need to explore a local neighborhood with radius of order $\log n$, then the total running time is $n^{\log n}$, which is not a polynomial, if there is no degree bound. To overcome this, we make use of a stronger notion called computationally efficient correlation decay
as introduced in~\cite{LLY12}. The observation is that when we go through a vertex with super-constant degree, the error is also decreased by a super-constant rate. Thus we do not need to explore a depth of $\log n$ if the degrees are large. The tradeoff relation between degree and decay rate defined by  computationally efficient correlation decay can support FPTAS with unbounded degree systems. Previously, this notation was only used in anti-ferromagnetic two-spin systems. In this paper, we prove that the distribution defined by edge covers also satisfies this stronger version of correlation decay and thus we give FPTAS for counting edge covers for any graph.
