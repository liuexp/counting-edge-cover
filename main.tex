\documentclass[a4paper,11pt]{article}
\usepackage{indentfirst}
\usepackage{amsmath}
\usepackage{amsthm}
\usepackage{amssymb}
\usepackage{amsfonts}
\usepackage{fancybox}
\usepackage{fancyvrb}
%\usepackage{minted}
\usepackage{color}
\usepackage{makeidx}
\usepackage{xeCJK}
%\setCJKmainfont[BoldFont={Adobe Heiti Std}, ItalicFont={AR PL New Kai}]{Adobe Song Std}
\setCJKmainfont[BoldFont={Adobe Heiti Std}, ItalicFont={Adobe Kaiti Std}]{Adobe Song Std}
\usepackage{graphicx}
\usepackage{geometry}
\usepackage{array}
%\usepackage{gnuplot-lua-tikz}
\usepackage{cite}
\usepackage{url}
\usepackage{enumerate}
%\geometry{left=1.5cm, right=1.5cm, top=1.5cm, bottom=1.5cm}
\geometry{left=1in, right=1in, top=1in, bottom=1in}
\usepackage{wrapfig}
%\usepackage{lettrine}
\usepackage{abstract}
\usepackage{subcaption}

% THEOREMS -------------------------------------------------------
\newtheorem{Thm}{Theorem}
\newtheorem{Cor}[Thm]{Corollary}
\newtheorem{Conj}[Thm]{Conjecture}
\newtheorem{Lem}[Thm]{Lemma}
\newtheorem{Prop}[Thm]{Proposition}
\newtheorem{Prob}{Problem}
\newtheorem{Exam}{Example}
\newtheorem{Def}[Thm]{Definition}
\newtheorem{Rem}[Thm]{Remark}
\newtheorem{Not}[Thm]{Notation}
\newtheorem*{Sol}{Solution}

% MATH -----------------------------------------------------------
\newcommand{\norm}[1]{\left\Vert#1\right\Vert}
\newcommand{\abs}[1]{\left\vert#1\right\vert}
\newcommand{\set}[1]{\left\{#1\right\}}
\newcommand{\Real}{\mathbb R}
\newcommand{\eps}{\varepsilon}
\newcommand{\To}{\longrightarrow}
\newcommand{\BX}{\mathbf{B}(X)}
\newcommand{\A}{\mathcal{A}}
\newcommand{\CommentS}[1]{}
% CODE ----------------------------------------------------------
\newcommand{\PltImg}[1]{
\begin{center}
\input{#1}
\end{center}
}

\newenvironment{code}%
{\vglue 5pt \VerbatimEnvironment\begin{Sbox}\begin{minipage}{0.9\textwidth}\begin{small}\begin{Verbatim}}%
{\end{Verbatim}\end{small}\end{minipage}\end{Sbox}\setlength{\shadowsize}{2pt}\shadowbox{\TheSbox}\vglue 5pt}


\usepackage{pgf}
\usepackage{tikz}
\usetikzlibrary{arrows}
\usetikzlibrary{shapes}
\tikzstyle{weight} = [font=\small]
\tikzstyle{edge} = [draw,thick,->,every node/.style={font=\sffamily\small}]
\tikzstyle{vertex}=[circle,fill=blue!20,minimum size=20pt,inner sep=0pt]
%\usetikzlibrary{automata}
%\usepackage[latin1]{inputenc}
\usepackage{verbatim}
\usepackage{listings}
%\usepackage{algorithmic} %old version; we can use algorithmicx instead
%\usepackage{algorithm}
%\usepackage[noend]{algpseudocode} %for pseudo code, include algorithmicsx automatically
\usepackage[ruled,vlined]{algorithm2e}

\lstdefinelanguage{Smalltalk}{
  morekeywords={self,super,true,false,nil,thisContext}, % This is overkill
  morestring=[d]',
  morecomment=[s]{"}{"},
  alsoletter={\#:},
  escapechar={!},
  literate=
    {BANG}{!}1
    {UNDERSCORE}{\_}1
    {\\st}{Smalltalk}9 % convenience -- in case \st occurs in code
    % {'}{{\textquotesingle}}1 % replaced by upquote=true in \lstset
    {_}{{$\leftarrow$}}1
    {>>>}{{\sep}}1
    {^}{{$\uparrow$}}1
    {~}{{$\sim$}}1
    {-}{{\sf -\hspace{-0.13em}-}}1  % the goal is to make - the same width as +
    %{+}{\raisebox{0.08ex}{+}}1		% and to raise + off the baseline to match -
    {-->}{{\quad$\longrightarrow$\quad}}3
	, % Don't forget the comma at the end!
  tabsize=2
}[keywords,comments,strings]

\lstloadlanguages{C++, Lisp, Haskell, Python, Smalltalk, Mathematica} %, Java,bash,Gnuplot,make,Matlab,PHP,Prolog,R,Ruby,sh,SQL,TeX,XML}

%--------------Now the document begins------------------

\title{A Simple FPTAS for Counting Edge Covers}
\date{}
\begin{document}
\author{
	Chengyu Lin
	\thanks{Shanghai Jiaotong University. {\tt linmrain@gmail.com}}
	\and
	Jingcheng Liu
	\thanks{Shanghai Jiaotong University. {\tt liuexp@gmail.com}}
	\and
	Pinyan Lu\thanks{Microsoft Research Asia. {\tt pinyanl@microsoft.com}}
}
\maketitle
\begin{abstract}
An edge cover of a graph is a set of edges such that every vertex has at least an adjacent edge in it. We design a very simple deterministic fully polynomial-time approximation scheme  (FPTAS) for counting the number of edge covers for any graph. Previously, approximation algorithm is only known for 3 regular graphs and it is randomized. Our main technique is correlation decay, which is a powerful tool to design FPTAS for counting problems. In order to get FPTAS for general graphs without degree bound, we make use of a stronger notion called computationally efficient correlation decay, which is introduced in [Li, Lu, Yin SODA 2012].  
\end{abstract}

\section{Introduction}
An edge cover of a graph is a set of edges such that every vertex has at least an adjacent edge in it. For a given input graph, we count the number of edge covers for that graph. This is a \#P-complete problem even when we restrict the input to 3 regular graphs. In this paper, we study the approximation version. For any given parameter $\epsilon>0$, the algorithm output a number $\hat{N}$ such that $(1-\epsilon) N\leq \hat{N} \leq (1+\epsilon) N$, where $N$ is the accurate number of edge covers of the input graph. We also require that the running time of the algorithm is bounded by $Poly(n,1/\epsilon)$, where $n$ is the number of vertices of the given graph. This is called a fully polynomial-time approximation scheme (FPTAS). Our main result of this paper is a FPTAS for edge covers for any graph. Previously, approximation algorithm is only known for 3 regular graphs and the algorithm is randomized~\cite{MFCS09}. The randomized version of FPTAS is called FPRAS, which uses random bits in the algorithms and we require that the final output is within the range $[(1-\epsilon) N, (1+\epsilon) N]$ with high probability.

Edge cover is related to many other graph problems such as (perfect) matching, and $k$-factor problems. All of them are talking about a set of edges which satisfies some local constraint defined on each vertex. For edge cover, it says that at least one incident edge should be chosen; while for matching, it is at most one edge. For generic constraints, it is the Holant framework, which is well studied in terms of exactly counting, recently in approximate counting. For counting matchings, there is a FPRAS based on MCMC and deterministic FPTAS is only known for graphs with bounded degree. For counting perfect matchings, it is a long standing open question if there is a FPRAS or FPTAS for it. For bipartite graphs, there is a FPRAS. The weighted version is exactly computing permanent of a non-negative matrix. This is one great achievement. It is still widely open if there exists a FPTAS for it. The current best deterministic algorithm can only approximate the permanent with an exponential large factor. There are many other counting problems, where there is a FPRAS and we do not know if there is a FPTAS or not such as counting the number of solution for a DNF formula~\cite{}. In this paper, we give a complete FPTAS for a problem, where even FPRAS is only known for very special family of graphs.

Another view point of edge cover is read twice monotone CNF formula (Rtw-Mon-CNF): each edge is viewed as a boolean variable and it is connected with two vertices (read twice); the constraint on each vertex is exactly a monotone CNF constrain as at least one edge variable is assigned to be True. Counting number of solutions for a Boolean formula is another set of interesting problem studied both in exact counting and approximate counting. One famous example is the FPRAS for counting the solutions for a  DNF formula. It is important open question if we can derandomized it. Our FPTAS for counting edge covers can also be viewed as a FPTAS for counting the solutions for a  Rtw-Mon-CNF formula. If we do not restrict that each variable appears in at most two constraints, There is no FPTAS or FPRAS unless NP is equal to P or RP.  

The common overall approach for designing approximate counting algorithms is to relate counting with probability distribution. 
 This is usually referred as ``counting vs sampling" paradigm when one mainly focuses on randomized counting.  If we can compute (estimate) the marginal probability, which in our problem is the probability of a given edge is chosen when we sample a edge cover uniformly at random, we can in turn to approximate count. In randomized FPRAS, we estimate the marginal probability by sampling, and the most successful approach is sampling by Markov chain.
In FPTAS, one calculate the marginal probability directly, and the most successful approach is correlation decay as introduce in \cite{BG08} and \cite{Weitz06}. We elaborate a bit on the ideas. 
 The marginal probability is estimated using only a local neighborhood around the edge. To justify the precision of the estimation, we show that far-away edges have little influence on the marginal distribution.
One most successful example is in anti-ferromagnetic two-spin system, including counting independent set. 
The correlation decay based FPTAS is beyond the best known MCMC based FPRAS and approaches the boundary of tractable and intractable. 
To the best of our knowledge, that was the only example for which the best tractable range for correlation decay based FPTAS exceeds the sampling based FPRAS. This paper offers another such example. FPRAS was \emph{the} solution concept for approximate counting, the recently development of correlation decay based FPTAS is changing the picture. It is interesting question to establish more deep relation between these two approaches.

A set of tools was developed for establishing correlation decay property. These are something like coupling argument, canonic path and so on for establish to rapid mixing for Markov Chains.
There are Self avoid walk tree, computational tree, potential function, bounded variables and so on. Armed with these powerful tools, there are recently many FPTAS were designed for many counting problems. 
Many of these techniques are also used in designing and analyzing the FPRAS for counting edge covers.  

Usually, the correlation decay property only implies FPTAS for system with bounded degree such as~\cite{}. The reason is that
we need to explore a local neighborhood with radius of order $\log n$, then the total running time is sup polynomial $n^{\log n}$ if there is no degree bound. To overcome this, we make use of stronger notion called computationally efficient correlation decay
as introduced in~\cite{LLY12}. The observation is that we will go through a vertex with sup constant degree, the error is also decreased by a supper constant. Thus we do not need to explore a depth of $\log n$ if the degrees are large. The tradeoff relation between degree and decay rate defined by  computationally efficient correlation decay can support FPTAS with unbounded degree system. Previously, this notation is only used in anti-ferromagnetic two-spin system. In this paper, we prove that the distribution defined by edge covers also satisfies this stronger version of correlation decay and thus we give FPTAS for counting edge covers for any graph.  


\section{Preliminaries}
\subsection{Definitions}
An edge cover of a graph is a set of edges such that every vertex has at least an adjacent edge in it.
Given a graph $G=(V,E)$ with $e \in E$,  we use $EC(G)$ to denote the set of all edge covers of graph $G$, and $P(G, e)$ to denote the marginal probability over $EC(G)$ that edge $e$ is not chosen, or formally, with $X \sim EC(G)$ uniformly,
\begin{equation}
	P(G, e) \triangleq \mathbb{P} \left(\textrm{edge $e$ is not chosen in $X$} \right)
	\label{defpge}
\end{equation}

In this paper, we deal with an extended notion of undirected graphs where dangling edges and free edges may be allowed.
\begin{Def}
	A {\bf dangling edge} $e=(u,\_)$ of a graph is such singleton edge with only one end-point vertex $u$, as shown in the Figure \ref{fig:G}.

	A {\bf free edge} $e=(\_, \_)$ of a graph is such edge with no end-point vertex. Note that a free edge is not a dangling edge.


\end{Def}

	We use graph to refer graph with or without dangling edges and free edges.
	Edges in the usual sense (i.e. neither dangling nor free), will be referred to as normal edges.
	%and graphs in the usual sense(i.e. graphs with only normal edges) will be refered to as normal graphs.

	We remark that an alternative view to these combinatorial definitions is from Rtw-Mon-CNF.
	A dangling edge is simply a variable which only appears at one clause, and a free edge is a variable
	that does not appear at all, whereas normal edge just corresponds to variables appearing twice.

For a graph $G=(V,E)$, an edge $e = (u,v) \in E$ and a vertex $u \in V$,
we define
\begin{align*}
G - e &\triangleq (V, E-e) \\
e - u &\triangleq (\_, v) \text{(note that here $v$ could be $\_$)} \\
G - u &\triangleq (V - u, \set{e: e \in E, e\text{ is not incident with }u} \cup \set{e - u: e \in E, e\text{ is incident with }u})
\end{align*}

%For a graph $G=(V,E)$ with an edge (may be dangling or free) $e \in E$ and a vertex $v \in V$,
%$$G-e \triangleq (V, E-e)$$
%and
%\begin{align*}
%G-v \triangleq & (V-\set{v}, \set{(x,y):x \in V-\set{v}, y \in V-\set{v}, (x,y) \in E}  \\
% &\cup \set{(x,\_):x \in V-\set{v}, (v,x) \in E}  \\
% &\cup \set{(\_,\_): (v,\_) \in E \text{ or } (v,v) \in E})
%\end{align*}

Note that here in edge set $E$, duplicates are allowed. We may have multiple dangling edges $(v,\_)$, many free edges $(\_,\_)$ and even parallel edges between two vertices $(u,v)$. For simplicity we treat $(v,\_)$ and $(\_,v)$ as the same.

For example, given a degree-3 vertex $u$ with dangling edge $e$ shown in Figure \ref{fig:G} , the result of $G-e-u$ is shown in Figure \ref{fig:G-e-u} and the result of $e_1 - u$ is shown in Figure \ref{fig:e-u}.

\begin{figure}[htp]
	\begin{subfigure}[b]{0.3\textwidth}
		\centering
		\setlength{\unitlength}{1mm}
		\begin{picture}(20,20)
			\put(0,0){\circle*{10}}
			\put(0,0){\line(1,1){10}}
			\put(20,0){\circle*{10}}
			\put(20,0){\line(-1,1){10}}
			\put(10,10){\circle*{10}}
			\put(10,10){\line(0,1){10}}
			\put(14,10){$u$}
			\put(8,15){$e$}
			\put(2,6){$e_1$}
			\put(15,5){$e_2$}
		\end{picture}
		\caption{$G$}
		\label{fig:G}
	\end{subfigure}
	\hfill
	\begin{subfigure}[b]{0.3\textwidth}
		\centering
		\setlength{\unitlength}{1mm}
		\begin{picture}(20,20)
			\put(0,0){\circle*{10}}
			\put(0,0){\line(0,1){10}}
			\put(20,0){\circle*{10}}
			\put(20,0){\line(0,1){10}}
			\put(2,6){$e_1$}
			\put(15,6){$e_2$}
		\end{picture}
		\caption{$G-e-u$}
		\label{fig:G-e-u}
	\end{subfigure}
    \begin{subfigure}[b]{0.3\textwidth}
		\centering
		\setlength{\unitlength}{1mm}
		\begin{picture}(20,20)
			\put(0,0){\circle*{10}}
			\put(0,0){\line(0,1){10}}
			\put(20,0){\circle*{10}}
			\put(20,0){\line(-1,1){10}}
			\put(10,10){\circle*{10}}
			\put(10,10){\line(0,1){10}}
			\put(14,10){$u$}
			\put(8,15){$e$}
			\put(2,6){$e_1$}
			\put(15,5){$e_2$}
		\end{picture}
		\caption{$e_1-u$}
		\label{fig:e-u}
	\end{subfigure}
	\caption{Dangling edges examples.}
\end{figure}

We use $0$ to denote scalar value $0$, and $\mathbf{0}$ to denote vector value 0, and $\set{e_i}_{i=1}^{d}$ denote the $d$-dimensional vector with $i$-th coordinate being $e_i$, so $\set{e_i} = \mathbf{0}$ means $\forall i, e_i = 0$.
We also use the convention that when $d=0, \prod_i^d p_i \triangleq 1$.

In general we use $n$ to refer to the number of vertices in a given graph, and $m$ to refer to the number of edges.


\section{The Recursion}

First we show the recursion for computing the marginal probability $P(G, e)$.

\subsection{$e$ is free edge}
%This case is trivial.
\begin{Prop}
	\[P(G,e) = \frac{1}{2}\]
\end{Prop}
\begin{proof}
	If $e$ is a free edge, then any solution with $e$ chosen is in one-to-one correspondence to any solution with $e$ not chosen. Hence exactly half of the solutions in $EC(G)$ doesn't choose $e$, so $P(G, e) = \frac{1}{2}$.
\end{proof}

\subsection{$e$ is dangling}
\begin{Prop}
For graph $G=(V,E)$ with a dangling edge $e=(u,\_)$, denote the $d$
edges incident with $u$ except $e$ as $e_1, e_2, \ldots, e_d$,
let $G_i = G - e - u - \sum_{k=1}^{i-1} e_k$ (specifically, $G_1 = G - e - u$), so we have%. $P(G,e)$ satisfies
	\begin{equation}
		P(G, e) = \frac{1-\prod_{i=1}^d P(G_i, e_i)}{2 - \prod_{i=1}^d P(G_i, e_i)} %= \frac{1}{2} - \frac{0.5 \prod_{i=1}^d P(G_i, e_i)}{2 - \prod_{i=1}^d P(G_i, e_i)}
		\label{propp3rg}
	\end{equation}
\end{Prop}
\begin{proof}
	For $\alpha \in \set{0,1}^d$, let $EC_{\alpha}(G-e-u)$ be the set of edge coverings in $G-e-u$ such that its restriction onto $\set{e_i}_{i=1}^{d}$ is consistent with $\alpha$, denote $Z_{\alpha} = \norm{EC_{\alpha}(G-e-u)}$, and $Z = \sum_{\alpha \in \set{0,1}^d} Z_{\alpha}$. % \triangleq \set{X : X\subseteq E, $

		Also note that as long as $\alpha \neq 0$, counting edge coverings with restriction $\alpha$ is the same in either $G$, $G-e$, or $G-e-u$, so it's enough to work with $G-e-u$. Note that in $G-e-u$, for every $i$, $e_i$ is either dangling or free, but not normal.
	\begin{align*}
		P(G,e) = & \frac{\norm{EC(G-e)}}{\norm{EC(G)}} \\
		=& \frac{\sum_{\alpha \in \set{0,1}^d, \alpha \neq \mathbf{0}} Z_{\alpha} }{ Z_{\mathbf{0}} + 2 \sum_{\alpha \in \set{0,1}^d, \alpha \neq \mathbf{0}} Z_{\alpha}} \\
		=& \frac{1 - \frac{Z_{\mathbf{0}}}{Z}}{ 2 - \frac{Z_{\mathbf{0}}}{Z}}.
	\end{align*}

	Now consider the term $\frac{Z_{\mathbf{0}}}{Z}$, it says the probability that a uniformly random solution drawn from $EC(G-e-u)$ picked none of $\set{e_i}_{i=1}^{d}$, so
	\begin{align*}
		\frac{Z_{\mathbf{0}}}{Z}=\mathbb{P} \left( \set{e_i} = \mathbf{0}\right) = \mathbb{P} (e_1 = 0) \prod_{i=2}^d \mathbb{P} \left(e_i = 0 \mid \set{e_j}_{j=1}^{i-1} = \mathbf{0}\right) = \prod_{i=1}^d P(G_i, e_i).
	\end{align*}

	Hence concludes the proof.
	
\end{proof}

%\begin{Cor}
%	For nontrivial dangling graphs,
%	\[P(G, e) \leq \frac{1}{2}\]
%\end{Cor}
%
%In fact this corrolary comes no surprise, because by looking combinatorially, picking a dangling edge should definitely yields more solutions.
%
%As a side note, note that $\forall i, G_i$ is a dangling graph (maybe trivial dangling graphs though), although $e_i$ can be a free edge.

%Note that $\forall i, e_i$ is either dangling or free, but not normal. (FIXME \footnote{$e_i$ in $G$ must be modified to be not normal in $G_i$.})

\subsection{$e$ is normal edge}
%Here we focus on graphs with no dangling edges and no free edges, as trivial dangling graphs is just trivial base cases, and nontrivial dangling graphs has been handled in the previous section.

%Here's a typical example of converting a general graph to dangling graphs.
%As an illustration, picking a normal edge $e=(u,v)$ in figure \ref{fig:generalG}, again we want to write the recursion of $P(G,e)$ for $G$.
By definition we have
\begin{equation}
	P(G,e) = \frac{\norm{EC(G-e)}}{\norm{EC(G-e)} + \norm{EC(G-e-u-v)} }.
\end{equation}


	For $e=(u,v)$ as a normal edge, let $\set{e_i}$ be the set of edges incident with vertex $u$ except $e$, and $\set{f_i}$ is the set of edges incident with vertex $v$ except $e$, and $d_1 = \norm{\set{e_i}}, d_2 = \norm{\set{f_i}}$, now for $\alpha \in \set{0,1}^{d_1}, \beta \in \set{0,1}^{d_2}$, we use $E_{\alpha,\beta}^G$ to denote the set of edge coverings in $G$ such that its restriction to $\set{e_i}_{i=1}^{d_1}$ is consistent with $\alpha$, and restriction to $\set{f_i}_{i=1}^{d_2}$ is consistent with $\beta$.

	Denote $Z_{\alpha, \beta}^G \triangleq \norm{EC_{\alpha, \beta}(G)}$, $G_1 \triangleq G-e, G_2 \triangleq G-e-u-v$, %and $C_1 \triangleq \norm{EC(G-e)}$, $C_2 \triangleq \norm{EC(G-e-u-v)}$,
	now for $\alpha \neq \mathbf{0} , \beta \neq \mathbf{0}$, we also have working with $G_1$ and working with $G_2$ is the same with restriction to $\alpha$ and $\beta$, or formally,
\[\norm{EC(G-e)} = \sum_{\alpha \neq \mathbf{0}, \beta \neq \mathbf{0}} Z_{\alpha, \beta}^{G_1} = \sum_{\alpha \neq \mathbf{0}, \beta \neq \mathbf{0}} Z_{\alpha, \beta}^{G_2}\]
%\[C_2 = \sum_{\alpha , \beta} Z_{\alpha, \beta}^{G_2}\]

Since we are only working with $G_2$, we may simply denote $Z_{\alpha, \beta} \triangleq Z_{\alpha, \beta}^{G_2}$, now let $Z = \sum_{\alpha , \beta} Z_{\alpha, \beta}$, and $\mathbb{P} \left( \alpha = 0, \beta = 0 \right) \triangleq \frac{Z_{\mathbf{0},\mathbf{0}}}{Z}, \mathbb{P} \left( \alpha = \mathbf{0} \right) \triangleq \frac{\sum_{\beta} Z_{\mathbf{0}, \beta} }{Z}, \mathbb{P} \left( \beta = \mathbf{0} \right) \triangleq \frac{\sum_{\alpha} Z_{ \alpha , \mathbf{0}} }{Z}$ we have,

\begin{Prop}
	
\[P(G,e) =  1 - \frac{1}{2 + \mathbb{P}\left( \alpha = 0, \beta = 0 \right) - \mathbb{P} \left( \alpha = 0 \right) - \mathbb{P} \left( \beta = 0 \right)}\]
\end{Prop}
\begin{proof}
	\begin{align*}
P(G,e) &= \frac{\sum_{\alpha \neq \mathbf{0}, \beta \neq \mathbf{0}} Z_{\alpha, \beta}}{Z + \sum_{\alpha \neq \mathbf{0}, \beta \neq \mathbf{0}} Z_{\alpha, \beta}} \\
&=\frac{Z - \sum_{\alpha}Z_{\alpha,\mathbf{0}} - \sum_{\beta} Z_{\mathbf{0}, \beta} + Z_{\mathbf{0}, \mathbf{0}}}{2Z - \sum_{\alpha}Z_{\alpha,\mathbf{0}} - \sum_{\beta} Z_{\mathbf{0}, \beta} + Z_{\mathbf{0}, \mathbf{0}}} \\
&= 1 - \frac{1}{2 + \mathbb{P}\left( \alpha = 0, \beta = 0 \right) - \mathbb{P} \left( \alpha = 0 \right) - \mathbb{P} \left( \beta = 0 \right)}
	\end{align*}
\end{proof}


\begin{figure}[htp]
	\begin{subfigure}[b]{0.3\textwidth}
		\centering
		\setlength{\unitlength}{1mm}
		\begin{picture}(20,30)
			\put(0,0){\circle*{10}}
			\put(0,0){\line(1,1){10}}
			\put(20,0){\circle*{10}}
			\put(20,0){\line(-1,1){10}}
			\put(10,10){\circle*{10}}
			\put(10,10){\line(0,1){10}}
			\put(10,20){\circle*{10}}
			\put(10,20){\line(1,1){10}}
			\put(10,20){\line(-1,1){10}}
			\put(20,30){\circle*{10}}
			\put(0,30){\circle*{10}}
			\put(14,20){$v$}
			\put(14,10){$u$}
			\put(8,14.5){$e$}
			\put(2,6){$e_1$}
			\put(15,5){$e_2$}
			\put(3,28){$f_1$}
			\put(13,28){$f_2$}
		\end{picture}
		\caption{$G$}
		\label{fig:generalG}
	\end{subfigure}
	\hfill
	\begin{subfigure}[b]{0.3\textwidth}
		\centering
		\setlength{\unitlength}{1mm}
		\begin{picture}(20,30)
			\put(0,1){\circle*{10}}
			\put(0,0){\line(1,1){10}}
			\put(20,1){\circle*{10}}
			\put(20,0){\line(-1,1){10}}
			\put(10,10){\circle*{10}}
			\put(10,20){\circle*{10}}
			\put(10,20){\line(1,1){10}}
			\put(10,20){\line(-1,1){10}}
			\put(20,30){\circle*{10}}
			\put(0,30){\circle*{10}}
			\put(14,20){$v$}
			\put(14,10){$u$}
			\put(2,6){$e_1$}
			\put(15,5){$e_2$}
			\put(3,28){$f_1$}
			\put(13,28){$f_2$}
		\end{picture}
		\caption{$G_1 = G-e$}
		\label{fig:generalG-e}
	\end{subfigure}
	\hfill
	\begin{subfigure}[b]{0.3\textwidth}
		\centering
		\setlength{\unitlength}{1mm}
		\begin{picture}(20,20)
			\put(0,1){\circle*{10}}
			\put(0,1){\line(0,1){10}}
			\put(20,1){\circle*{10}}
			\put(20,1){\line(0,1){10}}
			\put(20,30){\circle*{10}}
			\put(20,30){\line(0,-1){10}}
			\put(0,30){\circle*{10}}
			\put(0,30){\line(0,-1){10}}
			\put(2,6){$e_1$}
			\put(15,6){$e_2$}
			\put(2,24){$f_1$}
			\put(15,24){$f_2$}
		\end{picture}
		\caption{$G_2 = G-e-u-v$}
		\label{fig:generalG-e-u-v}
	\end{subfigure}
	\caption{Normal edge examples.}
\end{figure}

%First consider the term $ \mathbb{P}\left( \alpha = 0, \beta = 0 \right) = \frac{\sum_{\alpha = \mathbf{0}, \beta = \mathbf{0}} Z_{\alpha, \beta}^{G_2}}{Z} $, it says the probability in edge coverings of $G_2$ that none of the edges $\set{e_i}$ and none of $\set{f_i}$ is chosen.

Let

$G_i^1 \triangleq G - e - u - v - \sum_{k=1}^{i-1} e_k$,

$G_i^2 \triangleq G-e-u-v - \sum_{k=1}^{d_1}e_k - \sum_{k=1}^{i-1} f_k$,

$G_i^3 \triangleq G - e - u - v - \sum_{k=1}^{i-1} f_k$,

so we have

\begin{Prop}
	\begin{align*}
		\mathbb{P}\left( \alpha = 0\right) &= \prod_{i=1}^{d_1} P(G_i^1, e_i) \\
		\mathbb{P}\left( \beta = 0\right) &= \prod_{i=1}^{d_2} P(G_i^3, f_i) \\
		\mathbb{P}\left( \alpha = 0, \beta = 0 \right) &= \prod_{i=1}^{d_1} P(G_i^1, e_i) \cdot \prod_{i=1}^{d_2} P(G_i^2, f_i)
	\end{align*}
\end{Prop}

\begin{proof}
	\begin{align*}
		\mathbb{P}\left( \alpha = 0\right) =&\mathbb{P} \left( \set{e_i} = \mathbf{0}\right) =	\prod_{i=1}^{d_1} P(G_i^1, e_i) \\
		\mathbb{P}\left( \beta = 0\right) =&\mathbb{P} \left( \set{f_i} = \mathbf{0}\right) =	\prod_{i=1}^{d_2} P(G_i^3, f_i) \\
		\mathbb{P}\left( \alpha = 0, \beta = 0 \right) =& \mathbb{P} \left( \alpha = 0 \right) \cdot \mathbb{P}\left( \beta = 0 \mid \alpha = 0 \right) \\
		=& \mathbb{P} \left( \set{e_i} = \mathbf{0}\right) \cdot \mathbb{P} \left( \set{f_i} = \mathbf{0} \mid \set{e_i} = \mathbf{0} \right) \\
=&\prod_{i=1}^{d_1} \mathbb{P} \left( e_i = 0 \mid \set{e_j}_{j=1}^{i-1} = \mathbf{0} \right) \cdot \prod_{i=1}^{d_2} \mathbb{P} \left( f_i = 0 \mid \set{e_j}_{j=1}^{d_1} = \mathbf{0},\set{f_j}_{j=1}^{i-1} = \mathbf{0} \right) \\
=& \prod_{i=1}^{d_1} P(G_i^1, e_i) \cdot \prod_{i=1}^{d_2} P(G_i^2, f_i)
	\end{align*}
\end{proof}

\begin{Cor}
	\[P(G,e) =  1 - \frac{1}{2 + \prod_{i=1}^{d_1} P(G_i^1, e_i) \cdot \prod_{i=1}^{d_2} P(G_i^2, f_i) - \prod_{i=1}^{d_1} P(G_i^1, e_i) - \prod_{i=1}^{d_2} P(G_i^3, f_i)}\]
\end{Cor}

Note that for every $i$, $e_i$ is dangling or free in $G_i^1$, $f_i$ is dangling or free in $G_i^3$, and in $G_i^2$, neither $e_i$ nor $f_i$ is normal.
%(FIXME \footnote{$e_i,f_i$ in $G$ must be modified to be not normal in $G_i$.})




\section{Algorithm for Estimating Marginal Probability}

\IncMargin{1em}
\begin{algorithm}[H]
\SetKwInOut{Input}{input}\SetKwInOut{Output}{output}
\emph{ \textbf{function} $P(G, e, L):$}
\BlankLine
\Input{Graph $G$; edge $e$; Recursion depth $L$; }
\Output{Estimate of $P(G,e)$ up to depth $L$ .}
\Begin{
	\If{$L\leq0$ }{\Return{ $\frac{1}{2}$}}
	\ElseIf{$e$ is free }{
		\Return{ $\frac{1}{2}$}\;
	}
	\ElseIf{$e$ is dangling }{
		$L' \leftarrow L - \lceil \log_6{(d+1)} \rceil$\;
		%$L' \leftarrow L - \lceil d/5 \rceil$\;
		\Return{ $\frac{1-\prod_{i=1}^d P(G_i, e_i, L')}{2 - \prod_{i=1}^d P(G_i, e_i, L')}$} \;
	}
	\Else(// $e$ is normal ){
		$X \leftarrow \prod_{i=1}^{d_1} P(G_i^1, e_i, L)$\;
		$Y \leftarrow \prod_{i=1}^{d_2} P(G_i^2, f_i, L)$\;
		$Z \leftarrow \prod_{i=1}^{d_2} P(G_i^3, f_i, L)$\;
		\Return{ $1 - \cfrac{1}{2+ X\cdot Y  - X - Z }$ }\;
	}
 }
 \caption{Estimate $P(G,e)$ up to depth $L$}
\end{algorithm}
\DecMargin{1em}

We may compute the marginal probability $P(G, e)$ exactly with the previous recursion, but
that could take recursion depth of $O(n)$ which results in exponential computation time.
So here we use a truncated computation tree for an estimate of $P(G,e)$.

As a remark, the recursion depth used here is actually the so-called $M$-based depth introduced in \cite{LLY12} with $M=6$. 
%In fact $M$ could take any value as long as $M \geq 6$. %, and a slightly extended recursion depth could improve the algorithm running time for large $d$.
%Note the recursion depth used here is just a natural generalization of the so-called $M$-based depth introduced in \cite{LLY12}, we remark that it's sufficient to get an FPTAS with $M\geq 6$ using the $M$-based depth, here we show a stronger efficient algorithm via a slightly modified notion of recursion depth.

%\subsection{Analysis of the Algorithm}

Note that the normal case is invoked only once, so the algorithm keeps exploring in the third cases, until it hits the first 2 cases. We remark that an alternative view of the recursion depth is, we replace every node with degree greater than 6 with a $6$-ary branching subtree.
Now with this alternative view, it is easy to see that the nodes involved in the branching tree up to depth $L$ is at most $6^L$,
and for the initial normal edge case it involves at most $n$ subtrees, and for second-to-base-case nodes (i.e. nodes with $0<L \leq \lceil \log_6{(d+1)} \rceil $ ) they involve at most $n$ extra base cases, 
so the algorithm $P(G,e,L)$ has running time $O(n^2 \cdot 6^L)$.
%Now let $B(L)$ be the set of vertices in the recursion tree involved, and $R(L) \triangleq \norm{B(L)}$,
% by the recursion on $P(G,e,L)$ in the third case we have the recursive relation for $R(L)$,
%

 % Note that the corner case when $e$ is free so this is \leq rather than =
% \begin{align*}
%	 %R(L) \leq d R(L-d/5) , L > 0\\
%	 R(L) \leq d R(L-\lceil \log_6{(d+1)} \rceil) , L > 0\\
%	 R(L) = 1, L\leq 0
% \end{align*}

 %Therefore we conclude that $R(L) \leq d^{1+\frac{L}{\log_6{(d+1)}}} \leq d\cdot 6^L$, in other words, the running time of the above algorithm with recursion depth $L$ is at most $O(n\cdot 6^L)$.


\section{Correlation Decay Property}

In the last section, we show an algorithm $P(G,e,L)$ for estimating the marginal probability $P(G,e)$,
so here we establish the exponential correlation decay property, in the stronger sense with the
%modified recursion depth,
$M$-based depth,
of the estimation error in $P(G,e,L)$.%, hence $P(G,e,L)$

\begin{Thm}
	\label{cd-main-theorem}
	Given graph $G$, edge $e$ and depth $L$,
	\[\abs{P(G,e,L) - P(G,e)} \leq 3\cdot(\frac{1}{2})^{L+1}\]
\end{Thm}

Such phenomenon is usually referred to as exponential correlation decay. Before we prove the main theorem, we will introduce a few useful propositions and lemmas.

\begin{Prop}
	\[P(G, e) \leq \frac{1}{2}\]
\end{Prop}

\begin{proof}
	Although one may examine this case by case algebraically, this propositions can be seen quite obvious combinatorially in that, for any edge cover $X \in EC(G)$ s.t. $e \notin X$, $X+e$ is also an edge cover in $G$, and $\forall X,Y \in EC(G)$ s.t. $X \neq Y, e \notin X, e\notin Y$, we have $X+e \neq Y+e$. So the edge covers with $e$ chosen is at least as many as the edge covers with $e$ not chosen, hence the proposition follows.
\end{proof}

We remark that our algorithm also guarantees that $P(G,e,L) \leq \frac{1}{2}$, since $\frac{1 - \prod_i x_i}{2 - \prod_i x_i} = \frac{1}{2} - \frac{\prod_i x_i}{2(2 - \prod_i x_i)}$, and $X\cdot Y - X - Z \leq 0$.

For notational convenience, given a d-dimensional vector ${\bf x} \in [0, \frac{1}{2}]^d$, we denote
\[ f({\bf x}) \triangleq \frac{1- \prod_i x_i}{2 - \prod_i x_i}\]

Given a $d_1$-dimensional vector ${\bf x} \in [0, \frac{1}{2}]^{d_1}$ and two $d_2$-dimensional vectors ${\bf y,z} \in [0, \frac{1}{2}]^{d_2}$, let
\[ g({\bf x,y,z}) \triangleq  1- \frac{1}{2+\prod_i x_i \cdot \prod_i y_i - \prod_i x_i - \prod_i z_i} \]


%	\begin{Lem}
%		For $d$-variate function $f({\bf x})$,and a d-dimensional vector ${\bf x} \in [0, \frac{1}{2}]^d$,
%		\begin{align*}
%			\sum_i^d \abs{\cfrac{\partial f({\bf x})}{\partial x_i}} \leq & \min{\set{\frac{1}{2}, d \left( \frac{1}{2} %\right)^{d-1}}}
%		\end{align*}
%	\end{Lem}
%
%	\begin{proof}
%		Denote $i^*$ be one of the indices of smallest $x_i$, since $x_i \leq \frac{1}{2}$, we have
%	\begin{align*}
%		\sum_i^d \abs{\cfrac{\partial f({\bf x})}{\partial x_i}}  =& \sum_i^d \cfrac{ \prod_{k \neq i}^d x_k  }{\left( 2 - \prod_i x_i \right)^2}
%		\leq  d \prod_{k \neq i^*}^d x_k
%		\leq  d \left( \frac{1}{2} \right)^{d-1}
%	\end{align*}
%
%	So for $d \geq 4$ we have $\sum_i \abs{\cfrac{\partial f({\bf x})}{\partial x_i}} \leq \frac{1}{2}$.
%
%	For $d=0, \sum_i\abs{ \cfrac{\partial f({\bf x})}{\partial x_i}} = 0$.
%
%	Now consider $d=1$, $\sum_i\abs{ \cfrac{\partial f({\bf x})}{\partial x_i}} = \frac{1}{\left( 2 - x_1 \right)^2} \leq \frac{4}{9} $.
%
%	Next consider $d=2$,  $\sum_i\abs{ \cfrac{\partial f({\bf x})}{\partial x_i}} = \frac{x_1 + x_2}{\left( 2 - x_1x_2 \right)^2} \leq \frac{16}{49} $.
%
%	Finally for $d=3$,  $\sum_i\abs{ \cfrac{\partial f({\bf x})}{\partial x_i}} = \frac{x_1 + x_2 + x_3}{\left( 2 - x_1x_2x_3 \right)^2} \leq \frac{32}{75} $.
%	\end{proof}
%
%	\begin{Cor}
%		\label{meanvalue1}
%		For $d$-variate function $f$, given estimated $\hat{\bf x}$ for true value ${\bf x}$ , let $\epsilon = \max_i{\abs{x_i - \hat{x_i}}}$,
%		\begin{align*}
%			\abs{f(\hat{\bf x}) - f({\bf x})}
%		\leq  \min{\set{\frac{1}{2}, d \left( \frac{1}{2} \right)^{d-1}}} \cdot \epsilon
%		\end{align*}
%	\end{Cor}
%	\begin{proof}
%		By Mean Value Theorem, $\exists \tilde{\bf x}$ s.t. $\forall i, \min\set{x_i,\hat{x_i}} \leq \tilde{x_i} \leq %\max\set{x_i, \hat{x_i}}$.

%		\begin{align*}
%			\abs{f(\hat{\bf x}) - f({\bf x})} \leq& \sum_i\abs{ \frac{\partial f(\tilde{\bf x})}{\partial x_i}} \cdot \epsilon
%		\leq  \min{\set{\frac{1}{2}, d \left( \frac{1}{2} \right)^{d-1}}} \cdot \epsilon
%		\end{align*}
%	\end{proof}

	\begin{Lem}
		\label{meanvalue1}
		For $d$-variate function $f$, given estimated $\hat{\bf x}$ for true value ${\bf x}$ such that $\hat{\bf x} \in  [0, \frac{1}{2}]^d, {\bf x} \in [0, \frac{1}{2}]^d$, let $\epsilon = \max_i{\abs{x_i - \hat{x_i}}}$,
		\begin{align*}
			\abs{f(\hat{\bf x}) - f({\bf x})}
		\leq  \min{\set{\frac{1}{2}, d \left( \frac{1}{2} \right)^{d-1}}} \cdot \epsilon
		\end{align*}
	\end{Lem}

	\begin{proof}
        First we show for any d-dimensional vector ${\bf x} \in [0, \frac{1}{2}]^d$,
        \begin{align}
			\label{mv1pd}
			\sum_i^d \abs{\cfrac{\partial f({\bf x})}{\partial x_i}} \leq & \min{\set{\frac{1}{2}, d \left( \frac{1}{2} \right)^{d-1}}}
		\end{align} 
        
    	For $d=0, \sum_i\abs{ \cfrac{\partial f({\bf x})}{\partial x_i}} = 0$.

    	For $d=1$, $\sum_i\abs{ \cfrac{\partial f({\bf x})}{\partial x_i}} = \frac{1}{\left( 2 - x_1 \right)^2} \leq \frac{4}{9} $.

    	For $d=2$,  $\sum_i\abs{ \cfrac{\partial f({\bf x})}{\partial x_i}} = \frac{x_1 + x_2}{\left( 2 - x_1x_2 \right)^2} \leq \frac{16}{49} $.

    	For $d=3$,  $\sum_i\abs{ \cfrac{\partial f({\bf x})}{\partial x_i}} = \frac{x_1 + x_2 + x_3}{\left( 2 - x_1x_2x_3 \right)^2} \leq \frac{32}{75} $.

		Now denote $i^*$ be one of the indices of smallest $x_i$, since $x_i \leq \frac{1}{2}$, we have
    	\begin{align*}
    		\sum_i^d \abs{\cfrac{\partial f({\bf x})}{\partial x_i}}  =& \sum_i^d \cfrac{ \prod_{k \neq i}^d x_k  }{\left( 2 - \prod_i x_i \right)^2}
    		\leq  d \prod_{k \neq i^*}^d x_k
    		\leq  d \left( \frac{1}{2} \right)^{d-1}
    	\end{align*}

		So for $d \geq 4, \sum_i \abs{\cfrac{\partial f({\bf x})}{\partial x_i}} \leq \frac{1}{2}$, we have verified the inequality relations (\ref{mv1pd}).
        
		Finally by (\ref{mv1pd}) and Mean Value Theorem, $\exists \tilde{\bf x}$ s.t. $\forall i, \min\set{x_i,\hat{x_i}} \leq \tilde{x_i} \leq \max\set{x_i, \hat{x_i}}$.

		\begin{align*}
			\abs{f(\hat{\bf x}) - f({\bf x})} \leq& \sum_i\abs{ \frac{\partial f(\tilde{\bf x})}{\partial x_i}} \cdot \epsilon
		\leq  \min{\set{\frac{1}{2}, d \left( \frac{1}{2} \right)^{d-1}}} \cdot \epsilon
		\end{align*}
	\end{proof}

%	\begin{Lem}
%Given a $d_1$-dimensional vector ${\bf x} \in [0, \frac{1}{2}]^{d_1}$ and two $d_2$-dimensional vectors ${\bf y,z} \in [0, \frac{1}{2}]^{d_2}$,
%\label{lemnormalpd}
%		\begin{align*}
%			 \sum_i\abs{ \frac{\partial g({\bf x,y,z})}{\partial x_i} } \leq & 1 \\
%			\sum_i \abs{ \frac{\partial g({\bf x,y,z})}{\partial y_i} } \leq & 1 \\
%			\sum_i \abs{ \frac{\partial g({\bf x,y,z})}{\partial z_i} } \leq & 1
%		\end{align*}
%	\end{Lem}
%
	
	\begin{Lem}
		\label{meanvalue2}
		Given estimated $\hat{\bf x},\hat{\bf y},\hat{\bf z}$ for true value $ {\bf x}, {\bf y}, {\bf z}$ respectively, such that ${\bf x} \in [0, \frac{1}{2}]^{d_1} , {\bf y,z} \in [0, \frac{1}{2}]^{d_2}, \hat{\bf x} \in [0, \frac{1}{2}]^{d_1} , \hat{\bf y},\hat{\bf z} \in [0, \frac{1}{2}]^{d_2}$,
		let $\epsilon = \max \set{\abs{x_i - \hat{x_i}}, \abs{y_i - \hat{y_i} } , \abs{z_i - \hat{z_i}}}$,
		\begin{align*}
		\abs{g(\hat{\bf x}, \hat{\bf y}, \hat{\bf z}) - g({\bf x,y,z})}
		\leq  3\epsilon
		\end{align*}

	\end{Lem}

	\begin{proof}
		First for any ${\bf x} \in [0, \frac{1}{2}]^{d_1} , {\bf y,z} \in [0, \frac{1}{2}]^{d_2}$,
		\begin{align*}
		\sum_k^{d_1} \abs{ \frac{\partial g({\bf x,y,z})}{\partial x_k} } &= \sum_k^{d_1} \frac{\prod_{i\neq k}^{d_1} x_i \cdot \left( 1 - \prod_i^{d_2} y_i \right)}{(2+\prod_i x_i \cdot \prod_i y_i - \prod_i x_i - \prod_i z_i)^2}
		\leq d_1\cdot  \frac{1}{2^{d_1 - 1}}  \leq 1 \\
		\sum_k^{d_2} \abs{ \frac{\partial g({\bf x,y,z})}{\partial y_k} } &= \sum_k^{d_2} \frac{\prod_{i\neq k}^{d_2} y_i \cdot \prod_{i}^{d_1} x_i}{(2+\prod_i x_i \cdot \prod_i y_i - \prod_i x_i - \prod_i z_i)^2}
		\leq d_2 \cdot \frac{1}{2^{d_1 + d_2 - 1}} \leq 1 \\
		\sum_k^{d_2} \abs{ \frac{\partial g({\bf x,y,z})}{\partial z_k} } &= \sum_k^{d_2} \frac{\prod_{i\neq k}^{d_2} z_i }{(2+\prod_i x_i \cdot \prod_i y_i - \prod_i x_i - \prod_i z_i)^2}
		\leq d_2 \cdot \frac{1}{2^{d_2 - 1}} \leq 1
		\end{align*}

		Then by Mean Value Theorem, $\exists \tilde{\bf x}, \tilde{\bf y},  \tilde{\bf z}$ s.t. $\forall i, \min\set{x_i,\hat{x_i}} \leq \tilde{x_i} \leq \max\set{x_i, \hat{x_i}}, \min\set{y_i,\hat{y_i}} \leq \tilde{y_i} \leq \max\set{y_i, \hat{y_i}}, \min\set{z_i,\hat{z_i}} \leq \tilde{z_i} \leq \max\set{z_i, \hat{z_i}}$.
		\begin{align*}
		\abs{g(\hat{\bf x}, \hat{\bf y}, \hat{\bf z}) - g({\bf x,y,z})} \leq &
		\left(  \sum_i\abs{ \frac{\partial g(\tilde{\bf x},\tilde{\bf y},\tilde{\bf z} )}{\partial x_i}} + \sum_i\abs{ \frac{\partial g(\tilde{\bf x},\tilde{\bf y},\tilde{\bf z} )}{\partial y_i}} +  \sum_i\abs{ \frac{\partial g(\tilde{\bf x},\tilde{\bf y},\tilde{\bf z} )}{\partial z_i}} \right)  \epsilon
		\leq  3\epsilon
		\end{align*}
	\end{proof}

	Now we are ready for the proof of {\bf Theorem \ref{cd-main-theorem}}.

	\begin{proof}

		Note that the recursion for normal edge case is applied only once, so it's sufficient to show for free or dangling edge $e$,
		\[\abs{P(G,e,L) - P(G,e)} \leq (\frac{1}{2})^{L+1}, \textrm{for free or dangling edge $e$}\]
		then the case of normal edge automatically follows from Lemma \ref{meanvalue2} that
		\[\abs{P(G,e,L) - P(G,e)} \leq 3\cdot(\frac{1}{2})^{L+1}, \textrm{for normal edge $e$}\]

		Now we prove by induction with induction hypothesis:
		%\[\abs{P(G,e,L) - P(G,e)} \leq 3\cdot(\frac{1}{2})^{L+1}, \textrm{for normal edge $e$}\]
		\[\abs{P(G,e,L) - P(G,e)} \leq (\frac{1}{2})^{L+1}, \textrm{for free or dangling edge $e$}\]
		
		For base case $L=0, \abs{P(G,e,L) - P(G,e)} \leq \frac{1}{2}$ holds when $e$ is free or dangling.% When $e$ is normal, since the normal case only appears once, we have the maximal estimation error for the first two cases that $\epsilon \leq \frac{1}{2}$, so to sum up we have $\abs{P(G,e,L) - P(G,e)} \leq 3\cdot\frac{1}{2}$.

		Now suppose for $L<k$ we have the induction hypothesis true, now we try to show it's true for $L=k$.

		{\bf Case 1}, $e$ is free edge, then $\abs{P(G,e,L) - P(G,e)} = 0$.

		{\bf Case 2}, $e=(u,\_)$ is dangling with $deg(u)=d+1$, then by induction hypothesis we have
		$\epsilon \leq \frac{1}{2}^{L-\lceil \log_6{d+1}\rceil}$.

		First by Lemma \ref{meanvalue1} we need to show that for $d \leq 4$,
		\[\frac{1}{2^{1+L-\lceil \log_6{(d+1)}\rceil}} \leq \frac{1}{2^L}\]

		which is obvious because $\lceil\log_6{(d+1)}\rceil \leq 1$.

		Next we show for $d \geq 5$,
		\[ d\cdot \left( \frac{1}{2} \right)^{d-1 + L - \lceil \log_6{(d+1)}\rceil}  \leq \left( \frac{1}{2} \right)^L \]

		Namely for $d \geq 5$,
		\[ \log_2 d + \lceil \log_6{(d+1)} \rceil \leq d-1\]

		For $d=5,6$, one can directly examine that as $\log_2 d < 3$ and $\log_6 6 =1, \log_6 7 < 2$.

        %For $d\geq7$, by simply taking the derivative one can show that
		%\[ \log_2 d + \log_6{(d+1)} \leq d-1\]
		% @mrain:The derivative is wrong.
%        Since $\frac{d \log_2 x + \log_6 (x+1)}{dx} = \frac{1}{x} + \frac{1}{x+1} < 1$ when $x \geq 7$.
%        So for $d \ge 7$,

		For $d\geq 7$, note that the function $f(x) = d-2 -\log_2 d - \log_6{(d+1)}$ is monotonically increasing, and $f(7)>0$, so we have
        \[ \log_2 d + \log_6{(d+1)} + 1 \leq d-1\]

%		$\epsilon \leq \frac{1}{2}^{L - \lceil d/5 \rceil}$.
%		First we need to show that for $d \leq 5$,
%		\[\frac{1}{2^{1+L-\lceil d/5\rceil}} \leq \frac{1}{2^L}\]
%
%		which is obvious because $\lceil d/5 \rceil \leq 1$,
%
%		Next we show for $d \geq 6$,
%		\[ d\cdot \left( \frac{1}{2} \right)^{d-1 + L - \lceil d/5 \rceil}  \leq \left( \frac{1}{2} \right)^L \]
%
%		Namely for $d \geq 6$,
%		\[ \log_2 d + \lceil d/5 \rceil \leq d-1\]
%
%		This is followed from that the function $f(x) = x - 2 - d/5 - \log_2 d$ is monotonically increasing, and $f(6)>0$.

		Therefore, the hypothesis for $L=k$ is verified.
		\[\abs{P(G,e,L) - P(G,e)} \leq (\frac{1}{2})^{L+1}, \textrm{for free or dangling edge $e$}\]

		%{\bf Case 3}, $e$ is normal edge. Since the normal case only appears once at the root of the computation tree,
		%by induction hypothesis we have the maximal estimation error $\epsilon \leq (\frac{1}{2})^{L+1}$,
		%now we have the first part of the hypothesis for $L=k$ verified.
		%\[\abs{P(G,e,L) - P(G,e)} \leq 3\cdot(\frac{1}{2})^{L+1}, \textrm{for normal edge $e$}\]
		
		To sum up, the case of free or dangling edge and the case of normal edge together conclude the proof for our main theorem.
	\end{proof}


\section{Counting Edge Covers}

Finally, we present the procedures for approximately counting edge covers given good estimations of the marginal probability $P(G,e)$, hence an FPTAS for the approximate counting of edge covers problem.

\begin{Prop}
	
	Let $Z(G) \triangleq \norm{EC(G)}$, order the set of edges in $G$ in any order as $\set{e_i}$, define $G_1 \triangleq G, G_i \triangleq G_{i-1} - e_{i-1} - u_{i-1} - v_{i-1}, 1 < i \leq m $.

	\[ Z(G) = \frac{1}{\prod_{i=1}^m (1 - P(G_i, e_i))} \]

\end{Prop}

\begin{proof}
	Note that $EC(G) \neq \emptyset$, since the set of all edges $E$ is an edge cover.

	Now with $X \sim EC(G)$ uniformly, $\mathbb{P}(X=E)$ has two equivalent expressions,
	\begin{align*}
		\mathbb{P} (X = E) =& \frac{1}{Z(G)} \\
		\mathbb{P} (X = E) =& \prod_i \mathbb{P} \left(e_i = 1 \mid \set{e_j}_{j=1}^{i-1} = \mathbf{1} \right) \\
		=&\prod_i P(G_i, e_i)
	\end{align*}

	Therefore we have 
	\[ Z(G) = \frac{1}{\prod_{i=1}^m (1 - P(G_i, e_i))} \]
\end{proof}

We now show the main theorem of this section.
Define $Z(G, L) \triangleq \frac{1}{\prod_{i=1}^m (1 - P(G_i, e_i, L)}$ as the estimated number of edge covers given estimated $P(G_i, e_i, L)$

\begin{Thm}
	For $0< \epsilon <1$, take $L=\log m + \log(8/ \epsilon) $, 
	\[ 1- \epsilon \leq \frac{Z(G, L)}{Z(G)} \leq 1+ \epsilon\]
\end{Thm}

\begin{proof}

	\begin{align*}
		\frac{Z(G,L)}{Z(G)} &= \prod_{i=1}^m \frac{1-P(G_i, e_i, L)}{1-P(G_i, e_i)}
	\end{align*}

	Note that by (FIXME refer to cd-main-theorem), 

	\[\abs{P(G_i, e_i, L) - P(G,e)} \leq \frac{\epsilon}{4m}\]

	Since $1-P(G,e) \geq \frac{1}{2}$,
	\[ \frac{\abs{P(G_i, e_i, L) - P(G,e)}}{1 - P(G,e)} \leq \frac{\epsilon}{2m}\]
	
	Namely $\forall i$,
	\[ \left( 1 - \frac{\epsilon}{2m} \right) \leq \frac{1-P(G_i, e_i, L)}{1 - P(G,e)} \leq \left( 1 + \frac{\epsilon}{2m} \right)\]
	So we have
	\[ \left( 1 - \frac{\epsilon}{2m} \right)^m \leq \prod_{i=1}^m \frac{1-P(G_i, e_i, L)}{1 - P(G,e)} \leq \left( 1 + \frac{\epsilon}{2m} \right)^m\]
	\[ 1- \epsilon \leq \frac{Z(G, L)}{Z(G)} \leq 1+ \epsilon\]


\end{proof}

To sum up, run $Z(G, L)$ with $L = \log m + \log(8/ \epsilon)$, is the FPTAS for counting edge covers,
the total running time is $O(m^2 \cdot n \cdot \frac{1}{\epsilon})$.


\bibliographystyle{amsalpha}

\bibliography{refs}
\end{document}
