\section{The Recursion}

First we show the recursion for computing the marginal probability $P(G, e)$.

\subsection{$e$ is free edge}
%This case is trivial.
\begin{Prop}
	\[P(G,e) = \frac{1}{2}\]
\end{Prop}
\begin{proof}
	If $e$ is a free edge, then any edge cover with $e$ chosen is in one-to-one correspondence to any edge cover with $e$ not chosen. Hence exactly half of the edge covers in $EC(G)$ doesn't choose $e$, so $P(G, e) = \frac{1}{2}$.
\end{proof}

\subsection{$e$ is dangling}
\begin{Prop}
For graph $G=(V,E)$ with a dangling edge $e=(u,\_)$, denote the $d$
edges incident with $u$ except $e$ as $e_1, e_2, \ldots, e_d$,
let $G_i = G - e - u - \sum_{k=1}^{i-1} e_k$ (specifically, $G_1 = G - e - u$), we have%. $P(G,e)$ satisfies
	\begin{equation}
		P(G, e) = \frac{1-\prod_{i=1}^d P(G_i, e_i)}{2 - \prod_{i=1}^d P(G_i, e_i)} %= \frac{1}{2} - \frac{0.5 \prod_{i=1}^d P(G_i, e_i)}{2 - \prod_{i=1}^d P(G_i, e_i)}
		\label{propp3rg}
	\end{equation}
\end{Prop}
\begin{proof}
	For $\alpha \in \set{0,1}^d$, let $EC_{\alpha}(G-e-u)$ be the set of edge coverings in $G-e-u$ such that its restriction onto $\set{e_i}_{i=1}^{d}$ is consistent with $\alpha$, denote $Z_{\alpha} = \norm{EC_{\alpha}(G-e-u)}$, and $Z = \sum_{\alpha \in \set{0,1}^d} Z_{\alpha} = \norm{EC(G-e-u)}$. % \triangleq \set{X : X\subseteq E, $

		Also note that as long as $\alpha \neq 0$, counting edge coverings with restriction $\alpha$ is the same in either $G$, $G-e$, or $G-e-u$, so it's enough to work with $G-e-u$. Note that in $G-e-u$, for every $i$, $e_i$ is either dangling or free, but not normal.
	\begin{align*}
		P(G,e) = & \frac{\norm{EC(G-e)}}{\norm{EC(G)}} \\
		=& \frac{\sum_{\alpha \in \set{0,1}^d, \alpha \neq \mathbf{0}} Z_{\alpha} }{ Z_{\mathbf{0}} + 2 \sum_{\alpha \in \set{0,1}^d, \alpha \neq \mathbf{0}} Z_{\alpha}} \\
		=& \frac{1 - \frac{Z_{\mathbf{0}}}{Z}}{ 2 - \frac{Z_{\mathbf{0}}}{Z}}.
	\end{align*}

	Now consider the term $\frac{Z_{\mathbf{0}}}{Z}$, it says the probability that a uniformly random edge cover drawn from $EC(G-e-u)$ picked none of $\set{e_i}_{i=1}^{d}$, so
	\begin{align*}
		\frac{Z_{\mathbf{0}}}{Z}=\mathbb{P} \left( \set{e_i} = \mathbf{0}\right) = \mathbb{P} (e_1 = 0) \prod_{i=2}^d \mathbb{P} \left(e_i = 0 \mid \set{e_j}_{j=1}^{i-1} = \mathbf{0}\right) = \prod_{i=1}^d P(G_i, e_i).
	\end{align*}

	Hence concludes the proof.
	
\end{proof}

%\begin{Cor}
%	For nontrivial dangling graphs,
%	\[P(G, e) \leq \frac{1}{2}\]
%\end{Cor}
%
%In fact this corrolary comes no surprise, because by looking combinatorially, picking a dangling edge should definitely yields more solutions.
%
%As a side note, note that $\forall i, G_i$ is a dangling graph (maybe trivial dangling graphs though), although $e_i$ can be a free edge.

%Note that $\forall i, e_i$ is either dangling or free, but not normal. (FIXME \footnote{$e_i$ in $G$ must be modified to be not normal in $G_i$.})

\subsection{$e$ is normal edge}
%Here we focus on graphs with no dangling edges and no free edges, as trivial dangling graphs is just trivial base cases, and nontrivial dangling graphs has been handled in the previous section.

%Here's a typical example of converting a general graph to dangling graphs.
%As an illustration, picking a normal edge $e=(u,v)$ in figure \ref{fig:generalG}, again we want to write the recursion of $P(G,e)$ for $G$.
By definition we have
\begin{equation}
	P(G,e) = \frac{\norm{EC(G-e)}}{\norm{EC(G-e)} + \norm{EC(G-e-u-v)} }.
\end{equation}


	For $e=(u,v)$ as a normal edge, let $\set{e_i}$ be the set of edges incident with vertex $u$ except $e$, and $\set{f_i}$ is the set of edges incident with vertex $v$ except $e$, and $d_1 = \norm{\set{e_i}}, d_2 = \norm{\set{f_i}}$, now for $\alpha \in \set{0,1}^{d_1}, \beta \in \set{0,1}^{d_2}$, we use $E_{\alpha,\beta}^G$ to denote the set of edge coverings in $G$ such that its restriction to $\set{e_i}_{i=1}^{d_1}$ is consistent with $\alpha$, and restriction to $\set{f_i}_{i=1}^{d_2}$ is consistent with $\beta$.

	Denote $Z_{\alpha, \beta}^G \triangleq \norm{EC_{\alpha, \beta}(G)}$, $G_1 \triangleq G-e, G_2 \triangleq G-e-u-v$, %and $C_1 \triangleq \norm{EC(G-e)}$, $C_2 \triangleq \norm{EC(G-e-u-v)}$,
	now for $\alpha \neq \mathbf{0} , \beta \neq \mathbf{0}$, we also have working with $G_1$ and working with $G_2$ is the same with restriction to $\alpha$ and $\beta$, or formally,
\[\norm{EC(G-e)} = \sum_{\alpha \neq \mathbf{0}, \beta \neq \mathbf{0}} Z_{\alpha, \beta}^{G_1} = \sum_{\alpha \neq \mathbf{0}, \beta \neq \mathbf{0}} Z_{\alpha, \beta}^{G_2}\]
%\[C_2 = \sum_{\alpha , \beta} Z_{\alpha, \beta}^{G_2}\]

Since we are only working with $G_2$, we may simply denote $Z_{\alpha, \beta} \triangleq Z_{\alpha, \beta}^{G_2}$, now let $Z = \sum_{\alpha , \beta} Z_{\alpha, \beta}$, and $\mathbb{P} \left( \alpha = 0, \beta = 0 \right) \triangleq \frac{Z_{\mathbf{0},\mathbf{0}}}{Z}, \mathbb{P} \left( \alpha = \mathbf{0} \right) \triangleq \frac{\sum_{\beta} Z_{\mathbf{0}, \beta} }{Z}, \mathbb{P} \left( \beta = \mathbf{0} \right) \triangleq \frac{\sum_{\alpha} Z_{ \alpha , \mathbf{0}} }{Z}$ we have,

\begin{Prop}
	
\[P(G,e) =  1 - \frac{1}{2 + \mathbb{P}\left( \alpha = 0, \beta = 0 \right) - \mathbb{P} \left( \alpha = 0 \right) - \mathbb{P} \left( \beta = 0 \right)}\]
\end{Prop}
\begin{proof}
	\begin{align*}
P(G,e) &= \frac{\sum_{\alpha \neq \mathbf{0}, \beta \neq \mathbf{0}} Z_{\alpha, \beta}}{Z + \sum_{\alpha \neq \mathbf{0}, \beta \neq \mathbf{0}} Z_{\alpha, \beta}} \\
&=\frac{Z - \sum_{\alpha}Z_{\alpha,\mathbf{0}} - \sum_{\beta} Z_{\mathbf{0}, \beta} + Z_{\mathbf{0}, \mathbf{0}}}{2Z - \sum_{\alpha}Z_{\alpha,\mathbf{0}} - \sum_{\beta} Z_{\mathbf{0}, \beta} + Z_{\mathbf{0}, \mathbf{0}}} \\
&= 1 - \frac{1}{2 + \mathbb{P}\left( \alpha = 0, \beta = 0 \right) - \mathbb{P} \left( \alpha = 0 \right) - \mathbb{P} \left( \beta = 0 \right)}
	\end{align*}
\end{proof}


\begin{figure}[htp]
	\begin{subfigure}[b]{0.3\textwidth}
		\centering
		\setlength{\unitlength}{1mm}
		\begin{picture}(20,30)
			\put(0,0){\circle*{10}}
			\put(0,0){\line(1,1){10}}
			\put(20,0){\circle*{10}}
			\put(20,0){\line(-1,1){10}}
			\put(10,10){\circle*{10}}
			\put(10,10){\line(0,1){10}}
			\put(10,20){\circle*{10}}
			\put(10,20){\line(1,1){10}}
			\put(10,20){\line(-1,1){10}}
			\put(20,30){\circle*{10}}
			\put(0,30){\circle*{10}}
			\put(14,20){$v$}
			\put(14,10){$u$}
			\put(8,14.5){$e$}
			\put(2,6){$e_1$}
			\put(15,5){$e_2$}
			\put(3,28){$f_1$}
			\put(13,28){$f_2$}
		\end{picture}
		\caption{$G$}
		\label{fig:generalG}
	\end{subfigure}
	\hfill
	\begin{subfigure}[b]{0.3\textwidth}
		\centering
		\setlength{\unitlength}{1mm}
		\begin{picture}(20,30)
			\put(0,1){\circle*{10}}
			\put(0,0){\line(1,1){10}}
			\put(20,1){\circle*{10}}
			\put(20,0){\line(-1,1){10}}
			\put(10,10){\circle*{10}}
			\put(10,20){\circle*{10}}
			\put(10,20){\line(1,1){10}}
			\put(10,20){\line(-1,1){10}}
			\put(20,30){\circle*{10}}
			\put(0,30){\circle*{10}}
			\put(14,20){$v$}
			\put(14,10){$u$}
			\put(2,6){$e_1$}
			\put(15,5){$e_2$}
			\put(3,28){$f_1$}
			\put(13,28){$f_2$}
		\end{picture}
		\caption{$G_1 = G-e$}
		\label{fig:generalG-e}
	\end{subfigure}
	\hfill
	\begin{subfigure}[b]{0.3\textwidth}
		\centering
		\setlength{\unitlength}{1mm}
		\begin{picture}(20,20)
			\put(0,1){\circle*{10}}
			\put(0,1){\line(0,1){10}}
			\put(20,1){\circle*{10}}
			\put(20,1){\line(0,1){10}}
			\put(20,30){\circle*{10}}
			\put(20,30){\line(0,-1){10}}
			\put(0,30){\circle*{10}}
			\put(0,30){\line(0,-1){10}}
			\put(2,6){$e_1$}
			\put(15,6){$e_2$}
			\put(2,24){$f_1$}
			\put(15,24){$f_2$}
		\end{picture}
		\caption{$G_2 = G-e-u-v$}
		\label{fig:generalG-e-u-v}
	\end{subfigure}
	\caption{Normal edge examples.}
\end{figure}

%First consider the term $ \mathbb{P}\left( \alpha = 0, \beta = 0 \right) = \frac{\sum_{\alpha = \mathbf{0}, \beta = \mathbf{0}} Z_{\alpha, \beta}^{G_2}}{Z} $, it says the probability in edge coverings of $G_2$ that none of the edges $\set{e_i}$ and none of $\set{f_i}$ is chosen.

Let

$G_i^1 \triangleq G - e - u - v - \sum_{k=1}^{i-1} e_k$,

$G_i^2 \triangleq G-e-u-v - \sum_{k=1}^{d_1}e_k - \sum_{k=1}^{i-1} f_k$,

$G_i^3 \triangleq G - e - u - v - \sum_{k=1}^{i-1} f_k$,

so we have

\begin{Prop}
	\begin{align*}
		\mathbb{P}\left( \alpha = 0\right) &= \prod_{i=1}^{d_1} P(G_i^1, e_i) \\
		\mathbb{P}\left( \beta = 0\right) &= \prod_{i=1}^{d_2} P(G_i^3, f_i) \\
		\mathbb{P}\left( \alpha = 0, \beta = 0 \right) &= \prod_{i=1}^{d_1} P(G_i^1, e_i) \cdot \prod_{i=1}^{d_2} P(G_i^2, f_i)
	\end{align*}
\end{Prop}

\begin{proof}
	\begin{align*}
		\mathbb{P}\left( \alpha = 0\right) =&\mathbb{P} \left( \set{e_i} = \mathbf{0}\right) =	\prod_{i=1}^{d_1} P(G_i^1, e_i) \\
		\mathbb{P}\left( \beta = 0\right) =&\mathbb{P} \left( \set{f_i} = \mathbf{0}\right) =	\prod_{i=1}^{d_2} P(G_i^3, f_i) \\
		\mathbb{P}\left( \alpha = 0, \beta = 0 \right) =& \mathbb{P} \left( \alpha = 0 \right) \cdot \mathbb{P}\left( \beta = 0 \mid \alpha = 0 \right) \\
		=& \mathbb{P} \left( \set{e_i} = \mathbf{0}\right) \cdot \mathbb{P} \left( \set{f_i} = \mathbf{0} \mid \set{e_i} = \mathbf{0} \right) \\
=&\prod_{i=1}^{d_1} \mathbb{P} \left( e_i = 0 \mid \set{e_j}_{j=1}^{i-1} = \mathbf{0} \right) \cdot \prod_{i=1}^{d_2} \mathbb{P} \left( f_i = 0 \mid \set{e_j}_{j=1}^{d_1} = \mathbf{0},\set{f_j}_{j=1}^{i-1} = \mathbf{0} \right) \\
=& \prod_{i=1}^{d_1} P(G_i^1, e_i) \cdot \prod_{i=1}^{d_2} P(G_i^2, f_i)
	\end{align*}
\end{proof}

\begin{Cor}
	\[P(G,e) =  1 - \frac{1}{2 + \prod_{i=1}^{d_1} P(G_i^1, e_i) \cdot \prod_{i=1}^{d_2} P(G_i^2, f_i) - \prod_{i=1}^{d_1} P(G_i^1, e_i) - \prod_{i=1}^{d_2} P(G_i^3, f_i)}\]
\end{Cor}

Note that for every $i$, $e_i$ is dangling or free in $G_i^1$, $f_i$ is dangling or free in $G_i^3$, and in $G_i^2$, neither $e_i$ nor $f_i$ is normal.
%(FIXME \footnote{$e_i,f_i$ in $G$ must be modified to be not normal in $G_i$.})

