\documentclass[mathserif]{beamer}
\usepackage{indentfirst}
\usepackage{amsmath,amsthm,amssymb}
\usepackage{fancybox}
\usepackage{fancyvrb}
%\usepackage{minted}
\usepackage{color}
\usepackage{makeidx}
%\usepackage{xeCJK}
%\usepackage{fontspec}
%\usepackage{lmodern}
\usepackage{subcaption}
%\setCJKmainfont[BoldFont={Adobe Heiti Std}, ItalicFont={AR PL New Kai}]{Adobe Song Std}
\usepackage{amsmath}
\usepackage{amsfonts}
\usepackage{array}
\usepackage{gnuplot-lua-tikz}
\usepackage{cite}
\usepackage{url}
\usepackage{hyperref}
\hypersetup{colorlinks=true,linkcolor=blue,anchorcolor=green,citecolor=blue}
\usepackage{wrapfig}
%\usepackage{lettrine}
%\usepackage{abstract}

% THEOREMS -------------------------------------------------------
%\newtheorem{Thm}{Theorem}
%\newtheorem{Cor}[Thm]{Corollary}
%\newtheorem{Conj}[Thm]{Conjecture}
%\newtheorem{Lem}[Thm]{Lemma}
%\newtheorem{Prop}[Thm]{Proposition}
\newtheorem{proposition}[theorem]{Proposition}
%\newtheorem{Prob}{Problem}
%\newtheorem{Exam}{Example}
%\newtheorem{Def}[Thm]{Definition}
%\newtheorem{Rem}[theorem]{Remark}
\newtheorem{remark}[theorem]{Remark}
%\newtheorem{Not}[Thm]{Notation}
%\newtheorem*{Sol}{Solution}

% MATH -----------------------------------------------------------
\newcommand{\norm}[1]{\left\Vert#1\right\Vert}
\newcommand{\abs}[1]{\left\vert#1\right\vert}
\newcommand{\set}[1]{\left\{#1\right\}}
\newcommand{\Real}{\mathbb R}
\newcommand{\eps}{\varepsilon}
\newcommand{\To}{\longrightarrow}
\newcommand{\BX}{\mathbf{B}(X)}
\newcommand{\A}{\mathcal{A}}
\newcommand{\CommentS}[1]{}
% CODE ----------------------------------------------------------
\newcommand{\PltImg}[1]{
\begin{center}
\input{#1}
\end{center}
}

\newenvironment{code}%
{\vglue 5pt \VerbatimEnvironment\begin{Sbox}\begin{minipage}{0.9\textwidth}\begin{small}\begin{Verbatim}}%
{\end{Verbatim}\end{small}\end{minipage}\end{Sbox}\setlength{\shadowsize}{2pt}\shadowbox{\TheSbox}\vglue 5pt}


\usepackage{pgf}
%\usepackage{tikz}
%\usetikzlibrary{arrows,automata}
%\usepackage[latin1]{inputenc}
\usepackage{verbatim}
\usepackage{listings}
%\usepackage{algorithmic} %old version; we can use algorithmicx instead
\usepackage{algorithm}
\usepackage[noend]{algpseudocode} %for pseudo code, include algorithmicsx automatically

\lstloadlanguages{C++, Lisp, Haskell, Python, Mathematica, Java,bash,Gnuplot,make,Matlab,PHP,Prolog,R,Ruby,sh,SQL,TeX,XML}


\title[\scshape Counting Edge Covers]{\scshape A Simple FPTAS for Counting Edge Covers}
\author[Jingcheng Liu]{Chengyu Lin\inst{1} \and Jingcheng Liu\inst{1} \and Pinyan Lu\inst{2}}
\institute[SJTU]{\scshape \inst{1} Shanghai Jiao Tong University \and \inst{2} Microsoft Research Asia}
\date[SODA 2014]{\scshape ACM-SIAM Symposium on Discrete Algorithms, 2014}
%\usetheme{Warsaw}
\usetheme{Madrid}
%\usetheme[numbers,totalnumber,compress,sidebarshades]{PaloAlto}
%\usetheme{Copenhagen}
%\usecolortheme{whale}
\usecolortheme{seahorse}
%\setbeamercolor{background canvas}{bg=blue!9}
%\setbeamertemplate{theorems}[ams style]
\setbeamertemplate{navigation symbols}{}
\setbeamercovered{transparent}

\begin{document}

\begin{frame}
	\titlepage
\end{frame}

\begin{frame}
	\frametitle{Overview}
	\tableofcontents
\end{frame}

\section{Introduction}

\begin{frame}
	\frametitle{Edge cover}
	\begin{definition}
		For an undirected input graph $G=(V,E)$, an {\bf edge cover} of $G$ is a set of edges $C$ covering all vertices.
	\end{definition}
	\begin{example}
		\begin{figure}[htbp]
			\centering
			
\definecolor{c26497f}{RGB}{38,73,127}
\definecolor{cb20000}{RGB}{178,0,0}
\definecolor{ca3b3cc}{RGB}{163,179,204}


\begin{tikzpicture}[y=0.80pt,x=0.80pt,yscale=-1, inner sep=0pt, outer sep=0pt, scale=0.4]
\begin{scope}[shift={(0,0)}]
  \begin{scope}[cm={{1.0,0.0,0.0,1.0,(0.0,0.0)}}]
    \path[draw=c26497f,line width=1.600pt] (260.6910,162.8940) --
      (130.6130,257.4020);
    \path[draw=c26497f,line width=1.600pt] (259.8930,160.4380) --
      (99.1074,160.4380);
    \path[draw=c26497f,line width=1.600pt] (268.0450,159.1470) -- (344.6680,134.2510);
    \path<2->[draw=cb20000,line width=1.600pt] (268.0450,159.1470) -- (344.6680,134.2510);
    \path[draw=c26497f,line width=1.600pt] (228.3870,257.4020) --
      (98.3094,162.8940);
    \path[draw=c26497f,line width=1.600pt] (230.4770,255.8840) --
      (180.7910,102.9680);
    \path[draw=c26497f,line width=1.600pt] (234.2240,263.2380) -- (281.5790,328.4170);
    \path<2->[draw=cb20000,line width=1.600pt] (234.2240,263.2380) -- (281.5790,328.4170);
    \path[draw=c26497f,line width=1.600pt] (128.5230,255.8840) --
      (178.2090,102.9680);
    \path[draw=c26497f,line width=1.600pt] (124.7760,263.2380) --
      (77.4206,328.4170);
    \path<2->[draw=cb20000,line width=1.600pt] (124.7760,263.2380) --
      (77.4206,328.4170);
    \path[draw=c26497f,line width=1.600pt] (90.9554,159.1470) -- (14.3321,134.2510);
    \path<2->[draw=cb20000,line width=1.600pt] (90.9554,159.1470) -- (14.3321,134.2510);
    \path[draw=c26497f,line width=1.600pt] (179.5000,94.8159) -- (179.5000,14.2494);
    \path<2->[draw=cb20000,line width=1.600pt] (179.5000,94.8159) -- (179.5000,14.2494);
    \path[draw=c26497f,line width=1.600pt] (347.3510,136.9340) --
      (285.3260,327.8240);
    \path[draw=c26497f,line width=1.600pt] (345.2610,130.5040) --
      (182.8800,12.5270);
    \path[draw=c26497f,line width=1.600pt] (279.8570,331.7980) --
      (79.1429,331.7980);
    \path[draw=c26497f,line width=1.600pt] (73.6735,327.8240) -- (11.6494,136.9340);
    \path[draw=c26497f,line width=1.600pt] (13.7386,130.5040) -- (176.1200,12.5270);
    \path[fill=ca3b3cc] (264.0000,160.0000) ellipse (0.0897cm and 0.0897cm);
    \path[draw=black,draw opacity=0.700,line width=0.800pt] (264.0000,160.0000)
      ellipse (0.0897cm and 0.0897cm);
    \path[fill=ca3b3cc] (232.0000,260.0000) ellipse (0.0897cm and 0.0897cm);
    \path[draw=black,draw opacity=0.700,line width=0.800pt] (232.0000,260.0000)
      ellipse (0.0897cm and 0.0897cm);
    \path[fill=ca3b3cc] (127.0000,260.0000) ellipse (0.0897cm and 0.0897cm);
    \path[draw=black,draw opacity=0.700,line width=0.800pt] (127.0000,260.0000)
      ellipse (0.0897cm and 0.0897cm);
    \path[fill=ca3b3cc] (95.0000,160.0000) ellipse (0.0897cm and 0.0897cm);
    \path[draw=black,draw opacity=0.700,line width=0.800pt] (95.0000,160.0000)
      ellipse (0.0897cm and 0.0897cm);
    \path[fill=ca3b3cc] (179.0000,99.0000) ellipse (0.0897cm and 0.0897cm);
    \path[draw=black,draw opacity=0.700,line width=0.800pt] (179.0000,99.0000)
      ellipse (0.0897cm and 0.0897cm);
    \path[fill=ca3b3cc] (349.0000,133.0000) ellipse (0.0897cm and 0.0897cm);
    \path[draw=black,draw opacity=0.700,line width=0.800pt] (349.0000,133.0000)
      ellipse (0.0897cm and 0.0897cm);
    \path[fill=ca3b3cc] (284.0000,332.0000) ellipse (0.0897cm and 0.0897cm);
    \path[draw=black,draw opacity=0.700,line width=0.800pt] (284.0000,332.0000)
      ellipse (0.0897cm and 0.0897cm);
    \path[fill=ca3b3cc] (75.0000,332.0000) ellipse (0.0897cm and 0.0897cm);
    \path[draw=black,draw opacity=0.700,line width=0.800pt] (75.0000,332.0000)
      ellipse (0.0897cm and 0.0897cm);
    \path[fill=ca3b3cc] (10.0000,133.0000) ellipse (0.0897cm and 0.0897cm);
    \path[draw=black,draw opacity=0.700,line width=0.800pt] (10.0000,133.0000)
      ellipse (0.0897cm and 0.0897cm);
    \path[fill=ca3b3cc] (179.0000,10.0000) ellipse (0.0897cm and 0.0897cm);
    \path[draw=black,draw opacity=0.700,line width=0.800pt] (179.0000,10.0000)
      ellipse (0.0897cm and 0.0897cm);
  \end{scope}
\end{scope}

\end{tikzpicture}

			\caption{An edge cover for Petersen graph\visible<2->{, with edges chosen being highlighted in red. Note that this is also a perfect matching.}}
		\end{figure}
	\end{example}

\end{frame}

\begin{frame}
	\frametitle{Edge cover}
	Edge cover is related to many other problems such as:
	\begin{itemize}
		\item Matching problem.
		\item Rtw-Mon-CNF. (read twice monotone CNF)
		\item Holant problem.
		\item \dots.
	\end{itemize}
\end{frame}

\begin{frame}
	\frametitle{Relation to Matching}
	The minimal edge cover could be found via a greedy algorithm based on a maximum matching.
	\begin{example}
		Find edge covers by maximal matching?
		\begin{figure}[htp]
			\begin{subfigure}[b]{0.49\textwidth}
				\centering
						\definecolor{cb20000}{RGB}{178,0,0}
		\definecolor{c26497f}{RGB}{38,73,127}
		\definecolor{ca3b3cc}{RGB}{163,179,204}
		\begin{tikzpicture}[y=0.80pt,x=0.80pt,yscale=-1, inner sep=0pt, outer sep=0pt, scale=0.5]
		\begin{scope}[shift={(0,0)}]
		  \begin{scope}[cm={{1.0,0.0,0.0,1.0,(0.0,0.0)}}]
			\path[draw=c26497f,line width=1.600pt] (14.8956,10.3642) -- (109.6800,12.8190);
			\path<2->[draw=cb20000,line width=1.600pt] (14.8956,10.3642) -- (109.6800,12.8190);
			\path[draw=c26497f,line width=1.600pt] (12.9926,13.8534) -- (61.3126,84.7757);
			\path[draw=c26497f,line width=1.600pt] (111.6200,16.5590) -- (66.1835,84.7506);
			\path[draw=c26497f,line width=1.600pt] (118.2150,14.1720) -- (228.9470,47.0357);
			\path[draw=c26497f,line width=1.600pt] (67.7653,90.1120) -- (159.1950,129.7660);
			\path<2->[draw=cb20000,line width=1.600pt] (67.7653,90.1120) -- (159.1950,129.7660);
			\path[draw=c26497f,line width=1.600pt] (230.3220,51.6126) --
			  (165.9970,128.1630);
			\path[draw=c26497f,line width=1.600pt] (237.4180,47.5162) -- (344.1700,28.6244);
			\path<2->[draw=cb20000,line width=1.600pt] (237.4180,47.5162) -- (344.1700,28.6244);
			\path[fill=ca3b3cc] (11.0000,10.0000) ellipse (0.0948cm and 0.0948cm);
			\path[draw=black,draw opacity=0.700,line width=0.800pt] (11.0000,10.0000)
			  ellipse (0.0948cm and 0.0948cm);
			\path[fill=ca3b3cc] (114.0000,13.0000) ellipse (0.0948cm and 0.0948cm);
			\path[draw=black,draw opacity=0.700,line width=0.800pt] (114.0000,13.0000)
			  ellipse (0.0948cm and 0.0948cm);
			\path[fill=ca3b3cc] (64.0000,88.0000) ellipse (0.0948cm and 0.0948cm);
			\path[draw=black,draw opacity=0.700,line width=0.800pt] (64.0000,88.0000)
			  ellipse (0.0948cm and 0.0948cm);
			\path[fill=ca3b3cc] (233.0000,48.0000) ellipse (0.0948cm and 0.0948cm);
			\path[draw=black,draw opacity=0.700,line width=0.800pt] (233.0000,48.0000)
			  ellipse (0.0948cm and 0.0948cm);
			\path[fill=ca3b3cc] (163.0000,132.0000) ellipse (0.0948cm and 0.0948cm);
			\path[draw=black,draw opacity=0.700,line width=0.800pt] (163.0000,132.0000)
			  ellipse (0.0948cm and 0.0948cm);
			\path[fill=ca3b3cc] (348.0000,28.0000) ellipse (0.0948cm and 0.0948cm);
			\path[draw=black,draw opacity=0.700,line width=0.800pt] (348.0000,28.0000)
			  ellipse (0.0948cm and 0.0948cm);
		  \end{scope}
		\end{scope}

		\end{tikzpicture}

				\caption{$G$ has a perfect matching.}
			\end{subfigure}
			\hfill
			\begin{subfigure}[b]{0.49\textwidth}
				\centering
						\definecolor{cb20000}{RGB}{178,0,0}
		\definecolor{c26497f}{RGB}{38,73,127}
		\definecolor{ca3b3cc}{RGB}{163,179,204}
		\begin{tikzpicture}[y=0.80pt,x=0.80pt,yscale=-1, inner sep=0pt, outer sep=0pt, scale=0.4]
		\begin{scope}[shift={(0,0)}]
		  \begin{scope}[cm={{1.0,0.0,0.0,1.0,(0.0,0.0)}}]
			\path[draw=c26497f,line width=1.600pt] (343.6900,97.3106) --
			  (202.5780,170.9690);
			\path<2->[draw=cb20000,line width=1.600pt] (343.6900,97.3106) --
			  (202.5780,170.9690);
			\path[draw=c26497f,line width=1.600pt] (343.6890,92.8326) -- (202.5880,19.2258);
			\path[draw=c26497f,line width=1.600pt] (198.2880,168.3690) --
			  (198.2960,21.8271);
			\path[draw=c26497f,line width=1.600pt] (193.4500,173.3690) --
			  (15.8694,179.2680);
			\path[draw=c26497f,line width=1.600pt] (193.4600,16.8258) -- (15.8568,10.8941);
			\path<2->[draw=cb20000,line width=1.600pt] (193.4600,16.8258) -- (15.8568,10.8941);
			\path[draw=c26497f,line width=1.600pt] (11.0320,174.5890) -- (11.0201,15.5723);
			\path[fill=ca3b3cc] (348.0000,95.0000) ellipse (0.1084cm and 0.1084cm);
			\path[draw=black,draw opacity=0.700,line width=0.800pt] (348.0000,95.0000)
			  ellipse (0.1084cm and 0.1084cm);
			\path[fill=ca3b3cc] (198.0000,173.0000) ellipse (0.1084cm and 0.1084cm);
			\path[draw=black,draw opacity=0.700,line width=0.800pt] (198.0000,173.0000)
			  ellipse (0.1084cm and 0.1084cm);
			\path[fill=ca3b3cc] (198.0000,17.0000) ellipse (0.1084cm and 0.1084cm);
			\path[draw=black,draw opacity=0.700,line width=0.800pt] (198.0000,17.0000)
			  ellipse (0.1084cm and 0.1084cm);
			\path[fill=ca3b3cc] (11.0000,179.0000) ellipse (0.1084cm and 0.1084cm);
			\path<3->[fill=cb20000] (11.0000,179.0000) ellipse (0.2084cm and 0.2084cm);
			\path[draw=black,draw opacity=0.700,line width=0.800pt] (11.0000,179.0000)
			  ellipse (0.1084cm and 0.1084cm);
			\path[fill=ca3b3cc] (11.0000,11.0000) ellipse (0.1084cm and 0.1084cm);
			\path[draw=black,draw opacity=0.700,line width=0.800pt] (11.0000,11.0000)
			  ellipse (0.1084cm and 0.1084cm);
		  \end{scope}
		\end{scope}
		\end{tikzpicture}


				\caption{$G$ doesn't have a perfect matching.}
			\end{subfigure}
		\end{figure}
	\end{example}
	\pause
	\begin{remark}
		For a graph with a perfect matching, enumerating (sampling) perfect matchings is equivalent to enumerating (sampling) minimum edge covers.
	\end{remark}
\end{frame}

\begin{frame}
	\frametitle{ Relation to Rtw-Mon-CNF}
	TBA.
\end{frame}

\begin{frame}
	\frametitle{Counting Problems}
	A list of problems in their search, optimization, and counting versions.

	\begin{minipage}[tb]{0.3\linewidth}
		{\bf Search problems:}
		\begin{itemize}
			\item SAT.
				\pause
			\item Find a (perfect) matching.
				\pause
			\item Find an edge cover.
			\item \dots.
		\end{itemize}
	\end{minipage}
	\pause
	\begin{minipage}[tb]{0.3\linewidth}
		{\bf Optimizations:}
		\begin{itemize}
			\item MAX-SAT.
				\pause
			\item Find a maximum matching.
				\pause
			\item Find a minimum edge cover.
			\item \dots.
		\end{itemize}
	\end{minipage}
	\pause
	\begin{minipage}[tb]{0.3\linewidth}
		{\bf Counting problems:}
		\begin{itemize}
			\item \#SAT.
				\pause
			\item Counting matchings.
				\pause
			\item Counting edge covers.
			\item \dots.
		\end{itemize}
	\end{minipage}
\end{frame}

\begin{frame}
	\frametitle{Why counting?}
Besides theoretical computer science, counting problems are also related to many problems from other discipline such as:
\begin{itemize}
	\item Partition function of Statistical physics. 
		\pause
	\item Graph polynomials. 
		\pause
	\item Sampling, learning and inference. 
		\pause
	\item Query evaluations of probabilistic database.
	\item \dots.
\end{itemize}
\end{frame}

\begin{frame}
	\frametitle{Approximate Counting}
	Many interesting problems in the exact counting regimes, including counting edge cover, is hard (\#P-complete).
	Instead we look for these two types of polynomial time approximation scheme:
	\pause
	\begin{definition}[FPTAS]
		For given parameter $\eps > 0$ and an instance of a particular problem class, if the algorithm outputs a number $\hat{N}$ such that $(1-\eps) N \leq \hat{N} \leq (1+\eps) N$, where $N$ is the accurate answer of the problem instance, and the running time is bounded by $poly(n, 1/ \eps)$ with $n$ being the size of instance, this is called the {\bf FPTAS (fully polynomial time approximation scheme)}.

	\end{definition}
	\pause
	\begin{definition}[FPRAS]
		A randomized relaxation of FPTAS is known as {\bf FPRAS (fully polynomial time randomized approximation scheme)}, which uses random bits and only outputs $\hat{N}$ to the desired precision with high probability.
	\end{definition}
\end{frame}

\begin{frame}
	\frametitle{Counting v.s. Marginal Probability}
	TBA.
\end{frame}<++>
\end{document}
