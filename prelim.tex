
\section{Preliminaries}
\subsection{Definitions}
An edge cover of a graph is a set of edges such that every vertex has at least an adjacent edge in it.

Given a graph $G=(V,E)$ with edge $e$,  we use $P(G, e)$ to denote the marginal probability over all edge covers of $G$ that edge $e$ is not chosen, or formally,
\begin{equation}
	P(G, e) \triangleq \mathbb{P} \left(\textrm{edge $e$ is not chosen in $X$ } \mid \textrm{ $X$ is an edge cover of $G$} \right)
	\label{defpge}
\end{equation}

Here are several concepts useful in the recursions.
\begin{Def}
	A {\bf dangling edge} $e=(u,\_)$ of a graph is such singleton edge with only one end-point vertex $u$, as shown in the Figure \ref{fig:G}.

	A {\bf completely dangling edge} $e=(\_, \_)$ of a graph is such edge with no end-point vertex. Note that a completely dangling edge is not a dangling edge.

	A graph with at least one dangling edge is called {\bf non-trivial dangling graph}.

	Here's the list of configurations of trivial dangling graphs:(FIXME DRAW FIGURES) 
	\begin{itemize}
		\item  $E=\emptyset$.
		\item $V=\emptyset$.
		\item (FIXME MAYBE NOT NEEDED \footnote{check if each $G_i$ is still dangling graphs, either trivial or not}) Graphs containing only a vertex with only dangling edges.
		\item (FIXME MAYBE NOT NEEDED \footnote{check if each $G_i$ is still dangling graphs, either trivial or not}) Graphs containing only completely dangling edges.
	\end{itemize}

	We use {\bf non-dangling graphs} to refer to graphs containing neither dangling edges, nor completely dangling edges, and not trivial dangling graphs.
\end{Def}

\subsection{Notations}
For a graph $G=(V,E)$ and edge $e=(u,v) \in E$, let $G-e \triangleq (V, E-e)$, and $G-e-u-v \triangleq (V-\set{u,v}, E-e)$.

For example, given a degree-3 vertex $u$ with dangling edge $e$, the result of $G-e-u$ is shown in Figure \ref{fig:G-e-u}.
FIXME: also define $G-e-u$.

\begin{figure}[htp]
	\begin{subfigure}[b]{0.45\textwidth}
		\centering
		\setlength{\unitlength}{1mm}
		\begin{picture}(20,30)
			\put(0,0){\circle*{10}}
			\put(0,0){\line(1,1){10}}
			\put(20,0){\circle*{10}}
			\put(20,0){\line(-1,1){10}}
			\put(10,10){\circle*{10}}
			\put(10,10){\line(0,1){10}}
			\put(14,10){$u$}
			\put(8,15){$e$}
			\put(2,6){$e_1$}
			\put(15,5){$e_2$}
		\end{picture}
		\caption{$G$}
		\label{fig:G}
	\end{subfigure}
	\hfill
	\begin{subfigure}[b]{0.45\textwidth}
		\centering
		\setlength{\unitlength}{1mm}
		\begin{picture}(20,20)
			\put(0,0){\circle*{10}}
			\put(0,0){\line(0,1){10}}
			\put(20,0){\circle*{10}}
			\put(20,0){\line(0,1){10}}
			\put(2,6){$e_1$}
			\put(15,6){$e_2$}
		\end{picture}
		\caption{$G-e-u$}
		\label{fig:G-e-u}
	\end{subfigure}
	\caption{Dangling graphs examples.}
\end{figure}

We use $0$ to denote scalar value $0$, and $\mathbf{0}$ to denote vector value 0, and $\set{e_i}_{i=1}^{d}$ denote the $d$-dimensional vector with $i$-th coordinate being $e_i$, so $\set{e_i} = \mathbf{0}$ means $\forall i, e_i = 0$.

