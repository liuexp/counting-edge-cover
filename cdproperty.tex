\section{Correlation Decay Property}

In the last section we show an algorithm $P(G,e,L)$ for estimating the marginal probability $P(G,e)$,
so here we establish the exponential correlation decay property, in the stronger sense with the $M$-based depth, of the estimation error in $P(G,e,L)$.%, hence $P(G,e,L)$

\begin{Thm}
	Given graph $G$, edge $e$ and depth $L$,
	\[\abs{P(G,e,L) - P(G,e)} \leq 3\cdot(\frac{1}{2})^{L+1}\]
\end{Thm}

Such phenomenon is usually refered to as exponential correlation decay. Before we prove the main theorem, we will introduce a few useful propositions and lemmas.

\begin{Prop}
	\[P(G, e) \leq \frac{1}{2}\]
\end{Prop}

\begin{proof}
	Although one may examine this case by case algebraically, this propositions can be seen quite obvious combinatorially in that, for any edge cover $X \in EC(G)$ s.t. $e \notin X$, $X+e$ is also an edge cover in $G$, and $\forall X,Y \in EC(G)$ s.t. $X \neq Y, e \notin X, e\notin Y$, we have $X+e \neq Y+e$. So the edge covers with $e$ chosen is at least as many as the edge covers with $e$ not chosen, hence concludes the proposition follows.
\end{proof}

For notational convenience, given a d-dimensional vector ${\bf x} \in [0, \frac{1}{2}]^d$, we denote
\[ f({\bf x}) \triangleq \frac{1- \prod_i x_i}{2 - \prod_i x_i}\]

Given a $d_1$-dimensional vector ${\bf x} \in [0, \frac{1}{2}]^{d_1}$ and two $d_2$-dimensional vectors ${\bf y,z} \in [0, \frac{1}{2}]^{d_2}$, let
\[ g({\bf x,y,z}) \triangleq  1- \frac{1}{2+\prod_i x_i \cdot \prod_i y_i - \prod_i x_i - \prod_i z_i} \]


	\begin{Lem}
		For $d$-variate function $f({\bf x})$,
		\begin{align*}
			\abs{\sum_i \cfrac{\partial f({\bf x})}{\partial x_i}} \leq & \frac{1}{2}. \\
			\abs{\sum_i \cfrac{\partial f({\bf x})}{\partial x_i}} \leq & d \left( \frac{1}{2} \right)^{d-1}
		\end{align*}
	\end{Lem}

	\begin{proof}
		Denote $i^*$ be one of the indices of smallest $x_i$, since $x_i \leq \frac{1}{2}$, we have
	\begin{align*}
		\abs{\sum_i \cfrac{\partial f({\bf x})}{\partial x_i}}  =& \sum_i \cfrac{ \prod_{k \neq i} x_k  }{\left( 2 - \prod_i x_i \right)^2} \\
		\leq & d \prod_{k \neq i^*} x_k \\
		\leq & d \left( \frac{1}{2} \right)^{d-1}
	\end{align*}

	So for $d \geq 4$ we have $\abs{\sum_i \cfrac{\partial f({\bf x})}{\partial x_i}} \leq \frac{1}{2}$.

	For $d=0, \abs{\sum_i \cfrac{\partial f({\bf x})}{\partial x_i}} = 0$.

	Now consider $d=1$, $\abs{\sum_i \cfrac{\partial f({\bf x})}{\partial x_i}} = \frac{1}{\left( 2 - x_1 \right)^2} \leq \frac{4}{9} $.

	Next consider $d=2$,  $\abs{\sum_i \cfrac{\partial f({\bf x})}{\partial x_i}} = \frac{x_1 + x_2}{\left( 2 - x_1x_2 \right)^2} \leq \frac{16}{49} $.

	Finally for $d=3$,  $\abs{\sum_i \cfrac{\partial f({\bf x})}{\partial x_i}} = \frac{x_1 + x_2 + x_3}{\left( 2 - x_1x_2x_3 \right)^2} \leq \frac{16}{75} $.
	\end{proof}


	\begin{Lem}
		\begin{align*}
			\abs{ \sum_i \frac{\partial g({\bf x,y,z})}{\partial x_i} } \leq & 1 \\
			\abs{ \sum_i \frac{\partial g({\bf x,y,z})}{\partial y_i} } \leq & 1 \\
			\abs{ \sum_i \frac{\partial g({\bf x,y,z})}{\partial z_i} } \leq & 1
		\end{align*}
	\end{Lem}

	\begin{proof}
		\begin{align*}
		\abs{ \sum_i \frac{\partial g({\bf x,y,z})}{\partial x_i} } &= \sum_i \frac{\prod_{k\neq i} x_k \left( 1 - \prod_k y_k \right)}{(2+\prod_i x_i \cdot \prod_i y_i - \prod_i x_i - \prod_i z_i)^2} \\
		&\leq d_1 \frac{1}{2^{d_1 - 1}}  \leq 1 \\
		\abs{ \sum_i \frac{\partial g({\bf x,y,z})}{\partial y_i} } &= \sum_i \frac{\prod_{k\neq i} x_k \left( 1 - \prod_k y_k \right)}{(2+\prod_i x_i \cdot \prod_i y_i - \prod_i x_i - \prod_i z_i)^2} \\
		&\leq d_2 \frac{1}{2^{d_1 + d_2 - 1}} \leq 1 \\
		\abs{ \sum_i \frac{\partial g({\bf x,y,z})}{\partial z_i} } &= \sum_i \frac{\prod_{k\neq i} x_k \left( 1 - \prod_k y_k \right)}{(2+\prod_i x_i \cdot \prod_i y_i - \prod_i x_i - \prod_i z_i)^2} \\
		&\leq d_2 \frac{1}{2^{d_2 - 1}} \leq 1 \\
		\end{align*}
	\end{proof}

	Now we are ready for the main theorem.

	\begin{proof}
		First we note that by Mean Value Theorem,
		given estimated $\hat{\bf x}$, let ${\bf x}$ be the true value, let $\epsilon = \max_i{\abs{x_i - \hat{x_i}}}$.
		we have for $d$-variate function $f$,
		\begin{align*}
		\abs{f(\hat{\bf x}) - f({\bf x})} \leq& \abs{\sum_i \frac{\partial f({\bf x})}{\partial x_i}} \cdot \epsilon \\
		\leq & d\cdot \left( \frac{1}{2} \right)^{d-1} \cdot \epsilon
		\end{align*}

		Given estimated $\hat{\bf x},\hat{\bf y},\hat{\bf z}$, let $\epsilon = \max \set{\abs{x_i - \hat{x_i}}, \abs{y_i - \hat{y_i} } , \abs{z_i - \hat{z_i}}}$.
		\begin{align*}
		\abs{g(\hat{\bf x}, \hat{\bf y}, \hat{\bf z}) - g({\bf x,y,z})} \leq & \left(  \abs{\sum_i \frac{\partial g({\bf x,y,z})}{\partial x_i}} +\abs{ \sum_i \frac{\partial g({\bf x,y,z})}{\partial y_i}} + \abs{ \sum_i \frac{\partial g({\bf x,y,z})}{\partial z_i}} \right) \cdot \epsilon \\
		\leq & 3\epsilon
		\end{align*}


		Also note that the recursion for normal edge case is applied only once.
		Now we prove by simple induction.

		Induction hypothesis:
		\[\abs{P(G,e,L) - P(G,e)} \leq 3\cdot(\frac{1}{2})^{L+1}, \textrm{for normal edge $e$}\]
		\[\abs{P(G,e,L) - P(G,e)} \leq (\frac{1}{2})^{L+1}, \textrm{for free or dangling edge $e$}\]
		
		For base case $L=0, \abs{P(G,e,L) - P(G,e)} \leq \frac{1}{2}$ holds when $e$ is free or dangling. When $e$ is normal, since the normal case only appears once, we have the maximal estimation error for the first two cases that $\epsilon \leq \frac{1}{2}$, so to sum up we have $\abs{P(G,e,L) - P(G,e)} \leq 3\cdot\frac{1}{2}$.

		Now suppose for $L<k$ we have the induction hypothesis true, now we try to show it's true for $L=k$.

		{\bf Case 1}, $e$ is free edge, then $\abs{P(G,e,L) - P(G,e)} = 0$.

		{\bf Case 2}, $e=(u,\_)$ is dangling with $deg(u)=d+1$, then by induction hypothesis we have $\epsilon \leq \frac{1}{2}^{L-\lceil \log_6{d+1}\rceil}$.
		Now we only need to show that for $d \leq 4$,
		\[\frac{1}{2^{1+L-\lceil \log_6{(d+1)}\rceil}} \leq \frac{1}{2^L}\]
		which is obvious because $\lceil\log_6{d+1}\rceil \leq 1$.
		Next we show for $d >4$,
		\[ d\cdot \left( \frac{1}{2} \right)^{d-1 + L - \lceil \log_6{d+1}\rceil}  \leq \left( \frac{1}{2} \right)^L \]
		Namely for $d \geq 5$,
		\[ \log_2 d + \lceil \log_6{(d+1)} \rceil \leq d-1\]

		For $d=5,6$, one can directly examine that as $\log_2 d < 3$ and $\log_6 (d+1) < 2$.

        For $d\geq7$, by simply taking the derivative one can show that
		\[ \log_2 d + 1 + \log_6{(d+1)} \leq d-1\]

		Therefore, the second part of the hypothesis for $L=k$ is verified.
		\[\abs{P(G,e,L) - P(G,e)} \leq (\frac{1}{2})^{L+1}, \textrm{for free or dangling edge $e$}\]

		{\bf Case 3}, $e$ is normal edge. Since the normal case only appears once at the root of the computation tree,
		by induction hypothesis we have the maximal estimation error $\epsilon \leq (\frac{1}{2})^{L+1}$,
		now we have the first part of the hypothesis for $L=k$ verified.
		\[\abs{P(G,e,L) - P(G,e)} \leq 3\cdot(\frac{1}{2})^{L+1}, \textrm{for normal edge $e$}\]
		
		To sum up, this concludes the proof for our main theorem.
	\end{proof}
