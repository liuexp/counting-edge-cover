\section{Correlation Decay Property}
In this section, we establish the correlation decay property first on nontrivial dangling graphs, then generalize it to general graphs and graphs without degree bound.

\begin{Def}
	The recursion on $P(G,e, C, L)$ is said to exhibit exponential correlation decay phenomenon if for every two boundary conditions $C, C'$ and some constant $\alpha <1$,
	\[\abs{P(G,e,C,L) - P(G,e,C',L)} = O(\alpha ^L)\]
\end{Def}

FIXME: is this equivalent to saying $L$ is just the distance of $e$ from the set of edges in which $C,C'$ differs?


	\subsection{Nontrivial Dangling Graphs}
	\begin{Prop}
		If every $G_i$ is nontrivial dangling graphs, and $e_i$ is not free, 
		\begin{align*}
			\abs{\sum_i \cfrac{\partial P(G,e)}{\partial P(G_i,e_i)}} < 1 
		\end{align*}
	\end{Prop}

	\begin{proof}
		Denote $i^*$ be one of the indices of smallest $P(G_i, e_i)$, since for nontrivial dangling graphs, $P(G,e) \leq \frac{1}{2}$.
	\begin{align*}
		\abs{\sum_i \cfrac{\partial P(G,e)}{\partial P(G_i,e_i)}}  =& \cfrac{\sum_i \prod_{k \neq i} P(G_k, e_k)  }{\left( 2 - \prod_i P(G_i, e_i) \right)^2} \\
		\leq & d \prod_{k \neq i^*} P(G_k, e_k) \\
		\leq & d \left( \frac{1}{2} \right)^{d-1}
	\end{align*}

	So for $d \geq 3$ we already have $\abs{\sum_i \cfrac{\partial P(G,e)}{\partial P(G_i,e_i)}} < 1$.

	Note that $d\geq 1$ because the graph is nontrivial.

	Now first consider $d=1$, $\abs{\sum_i \cfrac{\partial P(G,e)}{\partial P(G_i,e_i)}} = \frac{1}{\left( 2 - P(G_1,e_1) \right)^2} < \frac{4}{9} $.

	Next consider $d=2$,  $\abs{\sum_i \cfrac{\partial P(G,e)}{\partial P(G_i,e_i)}} = \frac{P(G_1,e_1) + P(G_2,e_2)}{\left( 2 - P(G_1,e_1)P(G_2,e_2) \right)^2} < \frac{16}{49} $.
	\end{proof}

	\subsection{General Graphs}

	Note that the recursion for general graph is applied only once, so it's sufficient to show that the sum of the partial derivatives is bounded.

	\begin{Prop}
		\begin{align*}
			\frac{\partial \mathbb{P}\left( \alpha = 0, \beta = 0 \right) }{ \partial P(G_i^1, e_i) } \leq (\frac{1}{2})^{d_1 + d_2 -1} \\
			\frac{\partial \mathbb{P}\left( \alpha = 0, \beta = 0 \right) }{ \partial P(G_i^2, f_i) } \leq (\frac{1}{2})^{d_1 + d_2 -1} 
		\end{align*}
	\end{Prop}
	\begin{proof}
		\begin{align*}
			\frac{\partial \mathbb{P}\left( \alpha = 0, \beta = 0 \right) }{ \partial P(G_k^1, e_k) } = &\frac{\prod_{i=1}^{d_1} P(G_i^1, e_i) \cdot \prod_{i=1}^{d_2} P(G_i^2, f_i) }{\partial P(G_k^1, e_k)} \\
			=&\prod_{i=1,i\neq k}^{d_1} P(G_i^1, e_i) \cdot \prod_{i=1}^{d_2} P(G_i^2, f_i)\\
			\leq & (\frac{1}{2})^{d_1 + d_2 -1}
		\end{align*}
		Similarly for $\frac{\partial \mathbb{P}\left( \alpha = 0, \beta = 0 \right) }{ \partial P(G_i^2, f_i) }$.
	\end{proof}

	\begin{Cor}
		\[\sum_k \frac{\partial \mathbb{P}\left( \alpha = 0, \beta = 0 \right) }{ \partial P(G_k^1, e_k) } + \sum_k \frac{\partial \mathbb{P}\left( \alpha = 0, \beta = 0 \right) }{ \partial P(G_k^2, f_k) } \leq (d_1 + d_2) (\frac{1}{2})^{d_1+d_2-1} \leq 1\]
	\end{Cor}

	\begin{Prop}
		\begin{align*}
			\frac{\partial \mathbb{P}\left( \alpha = 0 \right) }{ \partial P(G_i^1, e_i) } \leq (\frac{1}{2})^{d_1 -1} \\
			\frac{\partial \mathbb{P}\left( \beta = 0 \right) }{ \partial P(G_i^2, f_i) } \leq (\frac{1}{2})^{d_2 -1} 
		\end{align*}
	\end{Prop}

	\begin{Cor}
		\[ 
			\sum_i \frac{\partial \mathbb{P}\left( \alpha = 0 \right) }{ \partial P(G_i^1, e_i) } +
			\sum_i \frac{\partial \mathbb{P}\left( \beta = 0 \right) }{ \partial P(G_i^2, f_i) } \leq
			d_1 \left( \frac{1}{2} \right)^{d_1-1} +d_2 \left( \frac{1}{2} \right)^{d_2-1} \leq
			2
		\]
	\end{Cor}

	\subsection{Computationally Efficient Correlation Decay}
	TODO: define the recursion depth of an edge.
	\begin{Def}
		Let $T$ be a rooted tree, $M\geq 2$ be a constant, the M-based depth $L_M (e)$ of a edge $e$ in $T$ is defined recursively as follows:
		$L_M (e) = 0$ if ???
	\end{Def}
