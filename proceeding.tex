\documentclass[twoside,leqno,twocolumn]{article}
\usepackage{ltexpprt}
\usepackage{amsmath}
\usepackage{amssymb}
\usepackage{amsfonts}
\usepackage{subcaption}
\usepackage[ruled,vlined]{algorithm2e}

% MATH -----------------------------------------------------------
\newcommand{\norm}[1]{\left\Vert#1\right\Vert}
\newcommand{\abs}[1]{\left\vert#1\right\vert}
\newcommand{\set}[1]{\left\{#1\right\}}
\newcommand{\Real}{\mathbb R}
\newcommand{\eps}{\varepsilon}
\newcommand{\To}{\longrightarrow}
\newcommand{\BX}{\mathbf{B}(X)}
\newcommand{\A}{\mathcal{A}}
\newcommand{\CommentS}[1]{}

%--------------Now the document begins------------------

\title{A Simple FPTAS for Counting Edge Covers}
\date{}
\begin{document}
\author{
	Chengyu Lin
	\thanks{Shanghai Jiaotong University. {\tt linmrain@gmail.com}}
	\and
	Jingcheng Liu
	\thanks{Shanghai Jiaotong University. {\tt liuexp@gmail.com}}
	\and
	Pinyan Lu\thanks{Microsoft Research. {\tt pinyanl@microsoft.com}}
}
\maketitle
\begin{abstract}
An edge cover of a graph is a set of edges such that every vertex has at least an adjacent edge in it. We design a very simple deterministic fully polynomial-time approximation scheme  (FPTAS) for counting the number of edge covers for any graph. Previously, approximation algorithm is only known for 3 regular graphs and it is randomized~\cite{MFCS09}. Our main technique is correlation decay, which is a powerful tool to design FPTAS for counting problems. In order to get FPTAS for general graphs without degree bound, we make use of a stronger notion called computationally efficient correlation decay, which was introduced in~\cite{LLY12}.
\end{abstract}

\section{Introduction}
An edge cover of a graph is a set of edges such that every vertex has at least an adjacent edge in it. For a given input graph, we count the number of edge covers for that graph. This is a \#P-complete problem even when we restrict the input to 3 regular graphs. In this paper, we study the approximation version. For any given parameter $\epsilon>0$, the algorithm output a number $\hat{N}$ such that $(1-\epsilon) N\leq \hat{N} \leq (1+\epsilon) N$, where $N$ is the accurate number of edge covers of the input graph. We also require that the running time of the algorithm is bounded by $Poly(n,1/\epsilon)$, where $n$ is the number of vertices of the given graph. This is called a fully polynomial-time approximation scheme (FPTAS). Our main result of this paper is a FPTAS for edge covers for any graph. Previously, approximation algorithm is only known for 3 regular graphs and the algorithm is randomized~\cite{MFCS09}. The randomized version of FPTAS is called FPRAS, which uses random bits in the algorithms and we require that the final output is within the range $[(1-\epsilon) N, (1+\epsilon) N]$ with high probability.

Edge cover is related to many other graph problems such as (perfect) matching, and $k$-factor problems. All of them are talking about a set of edges which satisfies some local constraint defined on each vertex. For edge cover, it says that at least one incident edge should be chosen; while for matching, it is at most one edge. For generic constraints, it is the Holant framework, which is well studied in terms of exactly counting, recently in approximate counting. For counting matchings, there is a FPRAS based on MCMC and deterministic FPTAS is only known for graphs with bounded degree. For counting perfect matchings, it is a long standing open question if there is a FPRAS or FPTAS for it. For bipartite graphs, there is a FPRAS. The weighted version is exactly computing permanent of a non-negative matrix. This is one great achievement. It is still widely open if there exists a FPTAS for it. The current best deterministic algorithm can only approximate the permanent with an exponential large factor. There are many other counting problems, where there is a FPRAS and we do not know if there is a FPTAS or not such as counting the number of solution for a DNF formula~\cite{}. In this paper, we give a complete FPTAS for a problem, where even FPRAS is only known for very special family of graphs.

Another view point of edge cover is read twice monotone CNF formula (Rtw-Mon-CNF): each edge is viewed as a boolean variable and it is connected with two vertices (read twice); the constraint on each vertex is exactly a monotone CNF constrain as at least one edge variable is assigned to be True. Counting number of solutions for a Boolean formula is another set of interesting problem studied both in exact counting and approximate counting. One famous example is the FPRAS for counting the solutions for a  DNF formula. It is important open question if we can derandomized it. Our FPTAS for counting edge covers can also be viewed as a FPTAS for counting the solutions for a  Rtw-Mon-CNF formula. If we do not restrict that each variable appears in at most two constraints, There is no FPTAS or FPRAS unless NP is equal to P or RP.  

The common overall approach for designing approximate counting algorithms is to relate counting with probability distribution. 
 This is usually referred as ``counting vs sampling" paradigm when one mainly focuses on randomized counting.  If we can compute (estimate) the marginal probability, which in our problem is the probability of a given edge is chosen when we sample a edge cover uniformly at random, we can in turn to approximate count. In randomized FPRAS, we estimate the marginal probability by sampling, and the most successful approach is sampling by Markov chain.
In FPTAS, one calculate the marginal probability directly, and the most successful approach is correlation decay as introduce in \cite{BG08} and \cite{Weitz06}. We elaborate a bit on the ideas. 
 The marginal probability is estimated using only a local neighborhood around the edge. To justify the precision of the estimation, we show that far-away edges have little influence on the marginal distribution.
One most successful example is in anti-ferromagnetic two-spin system, including counting independent set. 
The correlation decay based FPTAS is beyond the best known MCMC based FPRAS and approaches the boundary of tractable and intractable. 
To the best of our knowledge, that was the only example for which the best tractable range for correlation decay based FPTAS exceeds the sampling based FPRAS. This paper offers another such example. FPRAS was \emph{the} solution concept for approximate counting, the recently development of correlation decay based FPTAS is changing the picture. It is interesting question to establish more deep relation between these two approaches.

A set of tools was developed for establishing correlation decay property. These are something like coupling argument, canonic path and so on for establish to rapid mixing for Markov Chains.
There are Self avoid walk tree, computational tree, potential function, bounded variables and so on. Armed with these powerful tools, there are recently many FPTAS were designed for many counting problems. 
Many of these techniques are also used in designing and analyzing the FPRAS for counting edge covers.  

Usually, the correlation decay property only implies FPTAS for system with bounded degree such as~\cite{}. The reason is that
we need to explore a local neighborhood with radius of order $\log n$, then the total running time is sup polynomial $n^{\log n}$ if there is no degree bound. To overcome this, we make use of stronger notion called computationally efficient correlation decay
as introduced in~\cite{LLY12}. The observation is that we will go through a vertex with sup constant degree, the error is also decreased by a supper constant. Thus we do not need to explore a depth of $\log n$ if the degrees are large. The tradeoff relation between degree and decay rate defined by  computationally efficient correlation decay can support FPTAS with unbounded degree system. Previously, this notation is only used in anti-ferromagnetic two-spin system. In this paper, we prove that the distribution defined by edge covers also satisfies this stronger version of correlation decay and thus we give FPTAS for counting edge covers for any graph.  


\section{Preliminaries}
\subsection{Definitions}
An edge cover of a graph is a set of edges such that every vertex has at least an adjacent edge in it.

Given a graph $G=(V,E)$ with edge $e$,  we use $EC(G)$ to denote the set of all edge covers of graph $G$, and $P(G, e)$ to denote the marginal probability over $EC(G)$ that edge $e$ is not chosen, or formally, with $X \sim EC(G)$ uniformly,
\begin{equation}
	P(G, e) \triangleq \mathbb{P} \left(\textrm{edge $e$ is not chosen in $X$ } \right)
	\label{defpge}
\end{equation}

In this paper, we deal with an extended notion of undirected graphs where dangling edges and free edges may be allowed.
\begin{Def}
	A {\bf dangling edge} $e=(u,\_)$ of a graph is such singleton edge with only one end-point vertex $u$, as shown in the Figure \ref{fig:G}.

	A {\bf free edge} $e=(\_, \_)$ of a graph is such edge with no end-point vertex. Note that a free edge is not a dangling edge.

%	A graph with at least one dangling edge is called {\bf non-trivial dangling graph}.
%
%	We use {\bf non-dangling graphs} to refer to graphs containing neither dangling edges, nor free edges, and not trivial dangling graphs.

\end{Def}

	So we use graph to refer graphs in the usual sense with or without dangling edges or free edges.
	Edges in the usual sense (i.e. not dangling and not free), will be refered to as normal edges, 
	and graphs with only normal edges(i.e. graphs in the usual sense) will be refered to as normal graphs.

	We remark that an alternative view to these combinatorial definitions is from Rtw-Mon-CNF,
	a dangling edge is just a variable which only appears at one clause, and a free edge is a variable
	that does not appear at all, whereas normal edge just corresponds to variables appearing twice.

\subsection{Notations}
For a graph $G=(V,E)$ and an edge (may or may not be normal) $e \in E$, let $G-e \triangleq (V, E-e)$, and $G-v \triangleq (V-\set{v}, \set{(x,y):x\neq v, y\neq v, (x,y) \in E} \cup \set{(x,\_):(v,x) \in E})$.

For example, given a degree-3 vertex $u$ with dangling edge $e$ shown in Figure \ref{fig:G} , the result of $G-e-u$ is shown in Figure \ref{fig:G-e-u}.

\begin{figure}[htp]
	\begin{subfigure}[b]{0.45\textwidth}
		\centering
		\setlength{\unitlength}{1mm}
		\begin{picture}(20,30)
			\put(0,0){\circle*{10}}
			\put(0,0){\line(1,1){10}}
			\put(20,0){\circle*{10}}
			\put(20,0){\line(-1,1){10}}
			\put(10,10){\circle*{10}}
			\put(10,10){\line(0,1){10}}
			\put(14,10){$u$}
			\put(8,15){$e$}
			\put(2,6){$e_1$}
			\put(15,5){$e_2$}
		\end{picture}
		\caption{$G$}
		\label{fig:G}
	\end{subfigure}
	\hfill
	\begin{subfigure}[b]{0.45\textwidth}
		\centering
		\setlength{\unitlength}{1mm}
		\begin{picture}(20,20)
			\put(0,0){\circle*{10}}
			\put(0,0){\line(0,1){10}}
			\put(20,0){\circle*{10}}
			\put(20,0){\line(0,1){10}}
			\put(2,6){$e_1$}
			\put(15,6){$e_2$}
		\end{picture}
		\caption{$G-e-u$}
		\label{fig:G-e-u}
	\end{subfigure}
	\caption{Dangling edges examples.}
\end{figure}

We use $0$ to denote scalar value $0$, and $\mathbf{0}$ to denote vector value 0, and $\set{e_i}_{i=1}^{d}$ denote the $d$-dimensional vector with $i$-th coordinate being $e_i$, so $\set{e_i} = \mathbf{0}$ means $\forall i, e_i = 0$.



\section{The Computation Tree Recursion}
In this section, we provide a recursion for computing the marginal probability $P(G, e)$ with that of smaller instances.

\subsection{$e$ is free}
%This case is trivial.
\begin{Prop}
	\[P(G,e) = \frac{1}{2}\]
\end{Prop}
\begin{proof}
	If $e$ is a free edge, then any edge cover with $e$ chosen is in one-to-one correspondence to an edge cover with $e$ not chosen. Hence exactly half of the edge covers in $EC(G)$ does not choose $e$, so $P(G, e) = \frac{1}{2}$.
\end{proof}

\subsection{$e$ is dangling}
\begin{Lem}
For graph $G=(V,E)$ with a dangling edge $e=(u,\_)$, denote the $d$
edges incident with $u$ except $e$ as $e_1, e_2, \ldots, e_d$,
let $G_i \triangleq G - e - u - \sum_{k=1}^{i-1} e_k$ (specifically, $G_1 \triangleq G - e - u$),
	\begin{equation}
		P(G, e) = \frac{1-\prod_{i=1}^d P(G_i, e_i)}{2 - \prod_{i=1}^d P(G_i, e_i)} %= \frac{1}{2} - \frac{0.5 \prod_{i=1}^d P(G_i, e_i)}{2 - \prod_{i=1}^d P(G_i, e_i)}
		\label{propp3rg}
	\end{equation}
\end{Lem}
\begin{proof}
	For $\boldsymbol\alpha \in \set{0,1}^d$, let $EC_{\boldsymbol\alpha}(G-e-u)$ be the set of edge coverings in $G-e-u$ such that its restriction onto $\set{e_i}_{i=1}^{d}$ is consistent with $\boldsymbol\alpha$, denote $Z_{\boldsymbol\alpha} = \norm{EC_{\boldsymbol\alpha}(G-e-u)}$, and $Z = \sum_{\boldsymbol\alpha \in \set{0,1}^d} Z_{\boldsymbol\alpha} = \norm{EC(G-e-u)}$. % \triangleq \set{X : X\subseteq E, $

		Also note that as long as $\boldsymbol\alpha \neq 0$, counting edge coverings with restriction $\boldsymbol\alpha$ is the same in either $G$, $G-e$, or $G-e-u$, so it's enough to work with $G-e-u$. Note that in $G-e-u$, for every $i$, $e_i$ is either dangling or free, but not normal.
	\begin{align*}
		P(G,e) = & \frac{\norm{EC(G-e)}}{\norm{EC(G)}}
		= \frac{\sum_{\boldsymbol\alpha \in \set{0,1}^d, \boldsymbol\alpha \neq \mathbf{0}} Z_{\boldsymbol\alpha} }{ Z_{\mathbf{0}} + 2 \sum_{\boldsymbol\alpha \in \set{0,1}^d, \boldsymbol\alpha \neq \mathbf{0}} Z_{\boldsymbol\alpha}}
		=\frac{Z - Z_0}{2 Z - Z_0}
		= \frac{1 - \frac{Z_{\mathbf{0}}}{Z}}{ 2 - \frac{Z_{\mathbf{0}}}{Z}}.
	\end{align*}

	Now consider the term $\frac{Z_{\mathbf{0}}}{Z}$, it says the probability that a uniformly random edge cover drawn from $EC(G-e-u)$ picked none of $\set{e_i}_{i=1}^{d}$, so
	\begin{align*}
		\frac{Z_{\mathbf{0}}}{Z}=\mathbb{P} \left( \set{e_i} = \mathbf{0}\right) = \mathbb{P} (e_1 = 0) \prod_{i=2}^d \mathbb{P} \left(e_i = 0 \mid \set{e_j}_{j=1}^{i-1} = \mathbf{0}\right) = \prod_{i=1}^d P(G_i, e_i).
	\end{align*}

	Hence concludes the proof.
	
\end{proof}


\subsection{$e$ is a normal edge}
By definition we have
\begin{equation}
	P(G,e) = \frac{\norm{EC(G-e)}}{\norm{EC(G-e)} + \norm{EC(G-e-u-v)} }.
\end{equation}


	For $e=(u,v)$ as a normal edge, let $\set{e_i}$ be the set of edges incident with vertex $u$ except $e$, and $\set{f_i}$ be the set of edges incident with vertex $v$ except $e$, and $d_1 = \norm{\set{e_i}}, d_2 = \norm{\set{f_i}}$, now for $\boldsymbol\alpha \in \set{0,1}^{d_1}, \boldsymbol\beta \in \set{0,1}^{d_2}$, we use $E_{\boldsymbol\alpha,\boldsymbol\beta}^G$ to denote the set of edge covers for $G$ such that its restriction to $\set{e_i}_{i=1}^{d_1}$ is consistent with $\boldsymbol\alpha$, and restriction to $\set{f_i}_{i=1}^{d_2}$ is consistent with $\boldsymbol\beta$.

	Denote $Z_{\boldsymbol\alpha, \boldsymbol\beta}^G \triangleq \norm{EC_{\boldsymbol\alpha, \boldsymbol\beta}(G)}$, $G_1 \triangleq G-e, G_2 \triangleq G-e-u-v$. As an illustration, given a normal edge $e=(u,v)$ in $G$ as in Figure \ref{fig:generalG}, $G_1$ and $G_2$ is like Figure \ref{fig:generalG-e} and Figure \ref{fig:generalG-e-u-v}.

\begin{figure}[htp]
	\begin{subfigure}[b]{0.3\textwidth}
		\centering
		\setlength{\unitlength}{1mm}
		\begin{picture}(20,30)
			\put(0,0){\circle*{10}}
			\put(0,0){\line(1,1){10}}
			\put(20,0){\circle*{10}}
			\put(20,0){\line(-1,1){10}}
			\put(10,10){\circle*{10}}
			\put(10,10){\line(0,1){10}}
			\put(10,20){\circle*{10}}
			\put(10,20){\line(1,1){10}}
			\put(10,20){\line(-1,1){10}}
			\put(20,30){\circle*{10}}
			\put(0,30){\circle*{10}}
			\put(14,20){$v$}
			\put(14,10){$u$}
			\put(8,14.5){$e$}
			\put(2,6){$e_1$}
			\put(15,5){$e_2$}
			\put(3,28){$f_1$}
			\put(13,28){$f_2$}
		\end{picture}
		\caption{$G$}
		\label{fig:generalG}
	\end{subfigure}
	\hfill
	\begin{subfigure}[b]{0.3\textwidth}
		\centering
		\setlength{\unitlength}{1mm}
		\begin{picture}(20,30)
			\put(0,1){\circle*{10}}
			\put(0,0){\line(1,1){10}}
			\put(20,1){\circle*{10}}
			\put(20,0){\line(-1,1){10}}
			\put(10,10){\circle*{10}}
			\put(10,20){\circle*{10}}
			\put(10,20){\line(1,1){10}}
			\put(10,20){\line(-1,1){10}}
			\put(20,30){\circle*{10}}
			\put(0,30){\circle*{10}}
			\put(14,20){$v$}
			\put(14,10){$u$}
			\put(2,6){$e_1$}
			\put(15,5){$e_2$}
			\put(3,28){$f_1$}
			\put(13,28){$f_2$}
		\end{picture}
		\caption{$G_1 = G-e$}
		\label{fig:generalG-e}
	\end{subfigure}
	\hfill
	\begin{subfigure}[b]{0.3\textwidth}
		\centering
		\setlength{\unitlength}{1mm}
		\begin{picture}(20,20)
			\put(0,1){\circle*{10}}
			\put(0,1){\line(0,1){10}}
			\put(20,1){\circle*{10}}
			\put(20,1){\line(0,1){10}}
			\put(20,30){\circle*{10}}
			\put(20,30){\line(0,-1){10}}
			\put(0,30){\circle*{10}}
			\put(0,30){\line(0,-1){10}}
			\put(2,6){$e_1$}
			\put(15,6){$e_2$}
			\put(2,24){$f_1$}
			\put(15,24){$f_2$}
		\end{picture}
		\caption{$G_2 = G-e-u-v$}
		\label{fig:generalG-e-u-v}
	\end{subfigure}
	\caption{Normal edge examples.}
\end{figure}

	Note that as long as $\boldsymbol\alpha \neq \mathbf{0} , \boldsymbol\beta \neq \mathbf{0}$, working with $G_1$ and working with $G_2$ is the same with restriction to $\boldsymbol\alpha$ and $\boldsymbol\beta$, or formally,
\[\norm{EC(G-e)} = \sum_{\boldsymbol\alpha \neq \mathbf{0}, \boldsymbol\beta \neq \mathbf{0}} Z_{\boldsymbol\alpha, \boldsymbol\beta}^{G_1} = \sum_{\boldsymbol\alpha \neq \mathbf{0}, \boldsymbol\beta \neq \mathbf{0}} Z_{\boldsymbol\alpha, \boldsymbol\beta}^{G_2}\]
%\[C_2 = \sum_{\boldsymbol\alpha , \boldsymbol\beta} Z_{\boldsymbol\alpha, \boldsymbol\beta}^{G_2}\]

Since only $G_2$ is involved, denote $Z_{\boldsymbol\alpha, \boldsymbol\beta} \triangleq Z_{\boldsymbol\alpha, \boldsymbol\beta}^{G_2}, Z \triangleq \sum_{\boldsymbol\alpha \in \set{0,1}^{d_1} , \boldsymbol\beta \in \set{0,1}^{d_2}} Z_{\boldsymbol\alpha, \boldsymbol\beta}$,
and
$G_i^1 \triangleq G_2 - \sum_{k=1}^{i-1} e_k$,
$G_i^2 \triangleq G_2 - \sum_{k=1}^{d_1}e_k - \sum_{k=1}^{i-1} f_k$,
$G_i^3 \triangleq G_2 - \sum_{k=1}^{i-1} f_k$,


\begin{Lem}
	
	\[P(G,e) =  1 - \frac{1}{2 + \prod_{i=1}^{d_1} P(G_i^1, e_i) \cdot \prod_{i=1}^{d_2} P(G_i^2, f_i) - \prod_{i=1}^{d_1} P(G_i^1, e_i) - \prod_{i=1}^{d_2} P(G_i^3, f_i)}\]

\end{Lem}
\begin{proof}

	\begin{align*}
P(G,e) &= \frac{\sum_{\boldsymbol\alpha \neq \mathbf{0}, \boldsymbol\beta \neq \mathbf{0}} Z_{\boldsymbol\alpha, \boldsymbol\beta}}{Z + \sum_{\boldsymbol\alpha \neq \mathbf{0}, \boldsymbol\beta \neq \mathbf{0}} Z_{\boldsymbol\alpha, \boldsymbol\beta}} \\
&=\frac{Z - \sum_{\boldsymbol\alpha}Z_{\boldsymbol\alpha,\mathbf{0}} - \sum_{\boldsymbol\beta} Z_{\mathbf{0}, \boldsymbol\beta} + Z_{\mathbf{0}, \mathbf{0}}}{2Z - \sum_{\boldsymbol\alpha}Z_{\boldsymbol\alpha,\mathbf{0}} - \sum_{\boldsymbol\beta} Z_{\mathbf{0}, \boldsymbol\beta} + Z_{\mathbf{0}, \mathbf{0}}} \\
&= 1 - \frac{1}{2 + \mathbb{P}\left( \boldsymbol\alpha = \mathbf{0}, \boldsymbol\beta = \mathbf{0} \right) - \mathbb{P} \left( \boldsymbol\alpha = \mathbf{0} \right) - \mathbb{P} \left( \boldsymbol\beta = \mathbf{0} \right)}
	\end{align*}

	where $\mathbb{P} \left( \boldsymbol\alpha = \mathbf{0}, \boldsymbol\beta = \mathbf{0} \right) \triangleq \frac{Z_{\mathbf{0},\mathbf{0}}}{Z}, \mathbb{P} \left( \boldsymbol\alpha = \mathbf{0} \right) \triangleq \frac{\sum_{\boldsymbol\beta} Z_{\mathbf{0}, \boldsymbol\beta} }{Z}, \mathbb{P} \left( \boldsymbol\beta = \mathbf{0} \right) \triangleq \frac{\sum_{\boldsymbol\alpha} Z_{ \boldsymbol\alpha , \mathbf{0}} }{Z}$.

Now consider the three terms respectively,
	\begin{align*}
		\mathbb{P}\left( \boldsymbol\alpha = \mathbf{0}\right) &= \mathbb{P} \left( \set{e_i} = \mathbf{0}\right) =	\prod_{i=1}^{d_1} P(G_i^1, e_i) \\
		\mathbb{P}\left( \boldsymbol\beta = \mathbf{0}\right) &= \mathbb{P} \left( \set{f_i} = \mathbf{0}\right) =	\prod_{i=1}^{d_2} P(G_i^3, f_i) \\
		\mathbb{P}\left( \boldsymbol\alpha = \mathbf{0}, \boldsymbol\beta = \mathbf{0} \right) &=  \mathbb{P} \left( \boldsymbol\alpha = \mathbf{0} \right) \cdot \mathbb{P}\left( \boldsymbol\beta = \mathbf{0} \mid \boldsymbol\alpha = \mathbf{0} \right) \\
		&=  \mathbb{P} \left( \set{e_i} = \mathbf{0}\right) \cdot \mathbb{P} \left( \set{f_i} = \mathbf{0} \mid \set{e_i} = \mathbf{0} \right) \\
		&= \prod_{i=1}^{d_1} \mathbb{P} \left( e_i = 0 \mid \set{e_j}_{j=1}^{i-1} = \mathbf{0} \right) \cdot \prod_{i=1}^{d_2} \mathbb{P} \left( f_i = 0 \mid \set{e_j}_{j=1}^{d_1} = \mathbf{0},\set{f_j}_{j=1}^{i-1} = \mathbf{0} \right) \\
		&= \prod_{i=1}^{d_1} P(G_i^1, e_i) \cdot \prod_{i=1}^{d_2} P(G_i^2, f_i)
	\end{align*}

	Hence concludes the proof.
\end{proof}

Note that for every $i$, $e_i$ is dangling or free in $G_i^1$, $f_i$ is dangling or free in $G_i^3$, and in $G_i^2$, neither $e_i$ nor $f_i$ is normal.




\section{Computation Tree for Marginal Probability}
%P(G,e,C,L) - P(G,e,C',L)

We may compute the marginal probability $P(G, e)$ with the previous recursion, but
that could take recursion depth of $O(n)$ which results in exponential computation time.

So here we use a truncated computation tree for an estimate of $P(G,e)$.

Note the recursion depth used here is actually the so-called $M$-based depth introduced in \cite{LLY12} with $M=6$. We remark that actually $M$ could take any value as long as $M \geq 6$.
%Note the recursion depth used here is just a natural generalization of the so-called $M$-based depth introduced in \cite{LLY12}, we remark that it's sufficient to get an FPTAS with $M\geq 6$ using the $M$-based depth, here we show a stronger efficient algorithm via a slightly modified notion of recursion depth.

\IncMargin{1em}
\begin{algorithm}[H]
\SetKwInOut{Input}{input}\SetKwInOut{Output}{output}
\emph{ \textbf{function} $P(G, e, L):$}
\BlankLine
\Input{Graph $G$; edge $e$; Recursion depth $L$; }
\Output{Estimate of $P(G,e)$ up to depth $L$ .}
\Begin{
	\If{$L\leq0$ }{\Return{ $\frac{1}{2}$}}
	\ElseIf{$e$ is free }{
		\Return{ $\frac{1}{2}$}\;
	}
	\ElseIf{$e$ is dangling }{
		$L' \leftarrow L - \lceil \log_6{(d+1)} \rceil$\;
		%$L' \leftarrow L - \lceil d/5 \rceil$\;
		\Return{ $\frac{1-\prod_{i=1}^d P(G_i, e_i, L')}{2 - \prod_{i=1}^d P(G_i, e_i, L')}$} \;
	}
	\Else(// $e$ is normal ){
		$X \leftarrow \prod_{i=1}^{d_1} P(G_i^1, e_i, L)$\;
		$Y \leftarrow \prod_{i=1}^{d_2} P(G_i^2, f_i, L)$\;
		$Z \leftarrow \prod_{i=1}^{d_2} P(G_i^3, f_i, L)$\;
		\Return{ $1 - \cfrac{1}{2+ X\cdot Y  - X - Z }$ }\;
	}
 }
 \caption{Estimate $P(G,e)$ up to depth $L$}
\end{algorithm}
\DecMargin{1em}

\subsection{Analysis of the Algorithm}
Note that the normal case is invoked only once, so the algorithm keeps exploring in the third cases, until it hits the first 2 cases. Let $B(L)$ be the set of vertices in the recursion tree involved, and $R(L) \triangleq \norm{B(L)}$,
 by the recursion on $P(G,e,L)$ in the third case we have the recursive relation for $R(L)$,
 % Note that the corner case when $e$ is free so this is \leq rather than =
 \begin{align*}
	 %R(L) \leq d R(L-d/5) , L > 0\\
	 R(L) \leq d R(L-\lceil \log_6{(d+1)} \rceil) , L > 0\\
	 R(L) = 1, L\leq 0
 \end{align*}

 Therefore one may conclude that $R(L) \leq d^{1+\frac{L}{\log_6{(d+1)}}} \leq d\cdot M^L$, in other words, the running time of the above algorithm with recursion depth $L$ is at most $O(d\cdot M^L)$.


\section{Correlation Decay Property}

In the last section we show an algorithm $P(G,e,L)$ for estimating the marginal probability $P(G,e)$,
so here we establish the exponential correlation decay property, in the stronger sense with the $M$-based depth, of the estimation error in $P(G,e,L)$.%, hence $P(G,e,L)$

\begin{Thm}
	Given graph $G$, edge $e$ and depth $L$,
	\[\abs{P(G,e,L) - P(G,e)} \leq (\frac{1}{2})^{L}\]
\end{Thm}

Such phenomenon is usually refered to as exponential correlation decay. Before we prove the main theorem, we will introduce a few useful propositions and lemmas.

\begin{Prop}
	\[P(G, e) \leq \frac{1}{2}\]
\end{Prop}

\begin{proof}
	Although one may examine this case by case algebraically, this propositions can be seen quite obvious combinatorially in that, for any edge cover $X \in EC(G)$ s.t. $e \notin X$, $X+e$ is also an edge cover in $G$, and $\forall X,Y \in EC(G)$ s.t. $X \neq Y, e \notin X, e\notin Y$, we have $X+e \neq Y+e$. So the edge covers with $e$ chosen is at least as many as the edge covers with $e$ not chosen, hence concludes the proposition follows.
\end{proof}

For notational convenience, given a d-dimensional vector $x \in [0, \frac{1}{2}]^d$, we denote
\[ f(x) \triangleq \frac{1- \prod_i x_i}{2 - \prod_i x_i}\]

Given a $d_1$-dimensional vector $x \in [0, \frac{1}{2}]^{d_1}$ and two $d_2$-dimensional vectors $y,z \in [0, \frac{1}{2}]^{d_1}$, let
\[ g(x,y,z) \triangleq  1- \frac{1}{2+\prod_i x_i \cdot \prod_i y_i - \prod_i x_i - \prod_i z_i} \]


	\begin{Lem}
		\begin{align*}
			\abs{\sum_i \cfrac{f(x)}{\partial x_i}} \leq \frac{1}{2}. 
		\end{align*}
	\end{Lem}

	\begin{proof}
		Denote $i^*$ be one of the indices of smallest $x_i$, since $x_i \leq \frac{1}{2}$, we have
	\begin{align*}
		\abs{\sum_i \cfrac{\partial f(x)}{\partial x_i}}  =& \cfrac{\sum_i \prod_{k \neq i} x_k  }{\left( 2 - \prod_i P(G_i, e_i) \right)^2} \\
		\leq & d \prod_{k \neq i^*} P(G_k, e_k) \\
		\leq & d \left( \frac{1}{2} \right)^{d-1}
	\end{align*}

	So for $d \geq 4$ we have $\abs{\sum_i \cfrac{\partial P(G,e)}{\partial P(G_i,e_i)}} \leq \frac{1}{2}$.

	For $d=0, \abs{\sum_i \cfrac{f(x)}{\partial x_i}} = 0$.

	Now consider $d=1$, $\abs{\sum_i \cfrac{\partial P(G,e)}{\partial P(G_i,e_i)}} = \frac{1}{\left( 2 - P(G_1,e_1) \right)^2} \leq \frac{4}{9} $.

	Next consider $d=2$,  $\abs{\sum_i \cfrac{\partial P(G,e)}{\partial P(G_i,e_i)}} = \frac{P(G_1,e_1) + P(G_2,e_2)}{\left( 2 - P(G_1,e_1)P(G_2,e_2) \right)^2} \leq \frac{16}{49} $.

	Finally for $d=3$,  $\abs{\sum_i \cfrac{\partial P(G,e)}{\partial P(G_i,e_i)}} = \frac{P(G_1,e_1) + P(G_2,e_2)}{\left( 2 - P(G_1,e_1)P(G_2,e_2) \right)^2} < \frac{16}{49} $.
	\end{proof}



	\begin{Prop}
		\begin{align*}
			\frac{\partial \mathbb{P}\left( \alpha = 0, \beta = 0 \right) }{ \partial P(G_i^1, e_i) } \leq (\frac{1}{2})^{d_1 + d_2 -1} \\
			\frac{\partial \mathbb{P}\left( \alpha = 0, \beta = 0 \right) }{ \partial P(G_i^2, f_i) } \leq (\frac{1}{2})^{d_1 + d_2 -1} 
		\end{align*}
	\end{Prop}
	\begin{proof}
		\begin{align*}
			\frac{\partial \mathbb{P}\left( \alpha = 0, \beta = 0 \right) }{ \partial P(G_k^1, e_k) } = &\frac{\prod_{i=1}^{d_1} P(G_i^1, e_i) \cdot \prod_{i=1}^{d_2} P(G_i^2, f_i) }{\partial P(G_k^1, e_k)} \\
			=&\prod_{i=1,i\neq k}^{d_1} P(G_i^1, e_i) \cdot \prod_{i=1}^{d_2} P(G_i^2, f_i)\\
			\leq & (\frac{1}{2})^{d_1 + d_2 -1}
		\end{align*}
		Similarly for $\frac{\partial \mathbb{P}\left( \alpha = 0, \beta = 0 \right) }{ \partial P(G_i^2, f_i) }$.
	\end{proof}

	\begin{Cor}
		\[\sum_k \frac{\partial \mathbb{P}\left( \alpha = 0, \beta = 0 \right) }{ \partial P(G_k^1, e_k) } + \sum_k \frac{\partial \mathbb{P}\left( \alpha = 0, \beta = 0 \right) }{ \partial P(G_k^2, f_k) } \leq (d_1 + d_2) (\frac{1}{2})^{d_1+d_2-1} \leq 1\]
	\end{Cor}

	\begin{Prop}
		\begin{align*}
			\frac{\partial \mathbb{P}\left( \alpha = 0 \right) }{ \partial P(G_i^1, e_i) } \leq (\frac{1}{2})^{d_1 -1} \\
			\frac{\partial \mathbb{P}\left( \beta = 0 \right) }{ \partial P(G_i^2, f_i) } \leq (\frac{1}{2})^{d_2 -1} 
		\end{align*}
	\end{Prop}

	\begin{Cor}
		\[ 
			\sum_i \frac{\partial \mathbb{P}\left( \alpha = 0 \right) }{ \partial P(G_i^1, e_i) } +
			\sum_i \frac{\partial \mathbb{P}\left( \beta = 0 \right) }{ \partial P(G_i^2, f_i) } \leq
			d_1 \left( \frac{1}{2} \right)^{d_1-1} +d_2 \left( \frac{1}{2} \right)^{d_2-1} \leq
			2
		\]
	\end{Cor}

	Note that the recursion for general graph is applied only once, so it's sufficient to show that the sum of the partial derivatives is bounded.


\section{Counting Edge Covers}

Finally, we present the procedures for approximately counting edge covers given good estimations of the marginal probability $P(G,e)$, hence an FPTAS for the approximate counting of edge covers problem.

\begin{Prop}
	
	Let $Z(G) \triangleq \norm{EC(G)}$, order the set of edges in $G$ in any order as $\set{e_i}$, define $G_1 \triangleq G, G_i \triangleq G_{i-1} - e_{i-1} - u_{i-1} - v_{i-1}, 1 < i \leq m $.

	\[ Z(G) = \frac{1}{\prod_{i=1}^m (1 - P(G_i, e_i))} \]

\end{Prop}

\begin{proof}
	Note that $EC(G) \neq \emptyset$, since the set of all edges $E$ is an edge cover.

	Now with $X \sim EC(G)$ uniformly, $\mathbb{P}(X=E)$ has two equivalent expressions,
	\begin{align*}
		\mathbb{P} (X = E) =& \frac{1}{Z(G)} \\
		\mathbb{P} (X = E) =& \prod_i \mathbb{P} \left(e_i = 1 \mid \set{e_j}_{j=1}^{i-1} = \mathbf{1} \right) \\
		=&\prod_i P(G_i, e_i)
	\end{align*}

	Therefore we have 
	\[ Z(G) = \frac{1}{\prod_{i=1}^m (1 - P(G_i, e_i))} \]
\end{proof}

We now show the main theorem of this section.
Define $Z(G, L) \triangleq \frac{1}{\prod_{i=1}^m (1 - P(G_i, e_i, L))}$ as the estimated number of edge covers given estimated $P(G_i, e_i, L)$

\begin{Thm}
	For $0< \epsilon <1$, take $L=\log m + \log(8/ \epsilon) $, 
	\[ 1- \epsilon \leq \frac{Z(G, L)}{Z(G)} \leq 1+ \epsilon\]
\end{Thm}

\begin{proof}

	\begin{align*}
		\frac{Z(G,L)}{Z(G)} &= \prod_{i=1}^m \frac{1-P(G_i, e_i, L)}{1-P(G_i, e_i)}
	\end{align*}

	By Theorem \ref{cd-main-theorem},

	\[\abs{P(G_i, e_i, L) - P(G,e)} \leq \frac{\epsilon}{4m}\]

	Since $1-P(G,e) \geq \frac{1}{2}$,
	\[ \frac{\abs{P(G_i, e_i, L) - P(G,e)}}{1 - P(G,e)} \leq \frac{\epsilon}{2m}\]
	
	Namely $\forall i$,
	\[ \left( 1 - \frac{\epsilon}{2m} \right) \leq \frac{1-P(G_i, e_i, L)}{1 - P(G,e)} \leq \left( 1 + \frac{\epsilon}{2m} \right)\]
	So we have
	\[ \left( 1 - \frac{\epsilon}{2m} \right)^m \leq \prod_{i=1}^m \frac{1-P(G_i, e_i, L)}{1 - P(G,e)} \leq \left( 1 + \frac{\epsilon}{2m} \right)^m\]
	\[ 1- \epsilon \leq \frac{Z(G, L)}{Z(G)} \leq 1+ \epsilon\]


\end{proof}

To sum up, run $Z(G, L)$ with $L = \log m + \log(8/ \epsilon)$, is the FPTAS for counting edge covers,
the total running time is $O(m^2 \cdot n \cdot \frac{1}{\epsilon})$.


\section{Conclusions and Open Problems}
We have presented an FPTAS for approximately counting the number of edge covers for any graph. A natural question to ask is whether there is also an FPTAS for approximately counting weighted edge covers? Will there be a phase transition as in the case of counting independent sets? Our current approach can be directly extended to the case where $\lambda$ is not too big (e.g. $\lambda > \frac{4}{9}$), leaving the region where $\lambda$ is small open.

As we have noted, another view point of the edge cover problem is Rtw-Mon-CNF, hence another 2 natural problems are:
\begin{itemize}
	\item For what integer value of $k$, counting read $k$ times monotone CNF admits an FPTAS?
	\item For counting read twice CNF (Rtw-CNF), is there an FPTAS?
\end{itemize}
We remark that Rtw-CNF is also called Twice-SAT and admits an FPRAS~\cite{TwiceSAT}, while even counting read thrice 2CNF (without the monotone restriction) is as hard as counting 2CNF (without the read restriction) and hence does not admit FPRAS unless $RP=NP$.
In general, it is of interest to see how far the correlation decay technique could get in designing FPTAS for problems in which Markov Chain Monte Carlo method has succeeded, and such investigation in their relation might shed some light on the understanding of the power of randomization.


\bibliographystyle{plain}

\bibliography{refs}
\end{document}
