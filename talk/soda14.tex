\documentclass[mathserif]{beamer}
\usepackage{indentfirst}
\usepackage{amsmath,amsthm,amssymb}
\usepackage{fancybox}
\usepackage{fancyvrb}
%\usepackage{minted}
\usepackage{color}
\usepackage{makeidx}
%\usepackage{xeCJK}
%\usepackage{fontspec}
%\usepackage{lmodern}
\usepackage{subcaption}
%\setCJKmainfont[BoldFont={Adobe Heiti Std}, ItalicFont={AR PL New Kai}]{Adobe Song Std}
\usepackage{amsmath}
\usepackage{amsfonts}
\usepackage{array}
\usepackage{cite}
\usepackage{url}
\usepackage{hyperref}
\hypersetup{colorlinks=true,linkcolor=blue,anchorcolor=green,citecolor=blue}
\usepackage{wrapfig}
%\usepackage{lettrine}
%\usepackage{abstract}

% THEOREMS -------------------------------------------------------
%\newtheorem{Thm}{Theorem}
%\newtheorem{Cor}[Thm]{Corollary}
%\newtheorem{Conj}[Thm]{Conjecture}
%\newtheorem{Lem}[Thm]{Lemma}
%\newtheorem{Prop}[Thm]{Proposition}
\newtheorem{proposition}[theorem]{Proposition}
%\newtheorem{Prob}{Problem}
%\newtheorem{Exam}{Example}
%\newtheorem{Def}[Thm]{Definition}
%\newtheorem{Rem}[theorem]{Remark}
\newtheorem{remark}[theorem]{Remark}
%\newtheorem{Not}[Thm]{Notation}
%\newtheorem*{Sol}{Solution}

% MATH -----------------------------------------------------------
\newcommand{\norm}[1]{\left\Vert#1\right\Vert}
\newcommand{\abs}[1]{\left\vert#1\right\vert}
\newcommand{\set}[1]{\left\{#1\right\}}
\newcommand{\Real}{\mathbb R}
\newcommand{\eps}{\varepsilon}
\newcommand{\To}{\longrightarrow}
\newcommand{\BX}{\mathbf{B}(X)}
\newcommand{\A}{\mathcal{A}}
\newcommand{\CommentS}[1]{}
% CODE ----------------------------------------------------------
\newcommand{\PltImg}[1]{
\begin{center}
\input{#1}
\end{center}
}

\newenvironment{code}%
{\vglue 5pt \VerbatimEnvironment\begin{Sbox}\begin{minipage}{0.9\textwidth}\begin{small}\begin{Verbatim}}%
{\end{Verbatim}\end{small}\end{minipage}\end{Sbox}\setlength{\shadowsize}{2pt}\shadowbox{\TheSbox}\vglue 5pt}


%\usepackage{pgf}
\usepackage{tikz}
%\usetikzlibrary{arrows,automata}
%\usepackage[latin1]{inputenc}
\usepackage{verbatim}
\usepackage{listings}
%\usepackage{algorithmic} %old version; we can use algorithmicx instead
\usepackage{algorithm}
\usepackage[noend]{algpseudocode} %for pseudo code, include algorithmicsx automatically

\lstloadlanguages{C++, Lisp, Haskell, Python, Mathematica, Java,bash,Gnuplot,make,Matlab,PHP,Prolog,R,Ruby,sh,SQL,TeX,XML}


\title[\scshape Counting Edge Covers]{\scshape A Simple FPTAS for Counting Edge Covers}
\author[Jingcheng Liu]{Chengyu Lin\inst{1} \and Jingcheng Liu\inst{1} \and Pinyan Lu\inst{2}}
\institute[SJTU]{\scshape \inst{1} Shanghai Jiao Tong University \and \inst{2} Microsoft Research Asia}
\date[SODA 2014]{\scshape ACM-SIAM Symposium on Discrete Algorithms, 2014}
%\usetheme{Warsaw}
\usetheme{Madrid}
%\usetheme[numbers,totalnumber,compress,sidebarshades]{PaloAlto}
%\usetheme{Copenhagen}
%\usecolortheme{whale}
\usecolortheme{seahorse}
%\setbeamercolor{background canvas}{bg=blue!9}
%\setbeamertemplate{theorems}[ams style]
\setbeamertemplate{navigation symbols}{}
\setbeamercovered{transparent}

\begin{document}

\begin{frame}
	\titlepage
\end{frame}

\begin{frame}{Overview}
	\tableofcontents
\end{frame}

\section{\scshape Introduction}
\subsection{Definition}
\begin{frame}{Overview}
	\tableofcontents[currentsubsection, hideothersubsections, sectionstyle=show/shaded, subsectionstyle=show/shaded]
\end{frame}
\begin{frame}{Edge cover}
	\begin{definition}
		For an undirected input graph $G=(V,E)$, an {\bf edge cover} of $G$ is a set of edges $C$ covering all vertices.
	\end{definition}
	\begin{example}
		\begin{figure}[htbp]
			\centering
			
\definecolor{c26497f}{RGB}{38,73,127}
\definecolor{cb20000}{RGB}{178,0,0}
\definecolor{ca3b3cc}{RGB}{163,179,204}


\begin{tikzpicture}[y=0.80pt,x=0.80pt,yscale=-1, inner sep=0pt, outer sep=0pt, scale=0.4]
\begin{scope}[shift={(0,0)}]
  \begin{scope}[cm={{1.0,0.0,0.0,1.0,(0.0,0.0)}}]
    \path[draw=c26497f,line width=1.600pt] (260.6910,162.8940) --
      (130.6130,257.4020);
    \path[draw=c26497f,line width=1.600pt] (259.8930,160.4380) --
      (99.1074,160.4380);
    \path[draw=c26497f,line width=1.600pt] (268.0450,159.1470) -- (344.6680,134.2510);
    \path<2->[draw=cb20000,line width=1.600pt] (268.0450,159.1470) -- (344.6680,134.2510);
    \path[draw=c26497f,line width=1.600pt] (228.3870,257.4020) --
      (98.3094,162.8940);
    \path[draw=c26497f,line width=1.600pt] (230.4770,255.8840) --
      (180.7910,102.9680);
    \path[draw=c26497f,line width=1.600pt] (234.2240,263.2380) -- (281.5790,328.4170);
    \path<2->[draw=cb20000,line width=1.600pt] (234.2240,263.2380) -- (281.5790,328.4170);
    \path[draw=c26497f,line width=1.600pt] (128.5230,255.8840) --
      (178.2090,102.9680);
    \path[draw=c26497f,line width=1.600pt] (124.7760,263.2380) --
      (77.4206,328.4170);
    \path<2->[draw=cb20000,line width=1.600pt] (124.7760,263.2380) --
      (77.4206,328.4170);
    \path[draw=c26497f,line width=1.600pt] (90.9554,159.1470) -- (14.3321,134.2510);
    \path<2->[draw=cb20000,line width=1.600pt] (90.9554,159.1470) -- (14.3321,134.2510);
    \path[draw=c26497f,line width=1.600pt] (179.5000,94.8159) -- (179.5000,14.2494);
    \path<2->[draw=cb20000,line width=1.600pt] (179.5000,94.8159) -- (179.5000,14.2494);
    \path[draw=c26497f,line width=1.600pt] (347.3510,136.9340) --
      (285.3260,327.8240);
    \path[draw=c26497f,line width=1.600pt] (345.2610,130.5040) --
      (182.8800,12.5270);
    \path[draw=c26497f,line width=1.600pt] (279.8570,331.7980) --
      (79.1429,331.7980);
    \path[draw=c26497f,line width=1.600pt] (73.6735,327.8240) -- (11.6494,136.9340);
    \path[draw=c26497f,line width=1.600pt] (13.7386,130.5040) -- (176.1200,12.5270);
    \path[fill=ca3b3cc] (264.0000,160.0000) ellipse (0.0897cm and 0.0897cm);
    \path[draw=black,draw opacity=0.700,line width=0.800pt] (264.0000,160.0000)
      ellipse (0.0897cm and 0.0897cm);
    \path[fill=ca3b3cc] (232.0000,260.0000) ellipse (0.0897cm and 0.0897cm);
    \path[draw=black,draw opacity=0.700,line width=0.800pt] (232.0000,260.0000)
      ellipse (0.0897cm and 0.0897cm);
    \path[fill=ca3b3cc] (127.0000,260.0000) ellipse (0.0897cm and 0.0897cm);
    \path[draw=black,draw opacity=0.700,line width=0.800pt] (127.0000,260.0000)
      ellipse (0.0897cm and 0.0897cm);
    \path[fill=ca3b3cc] (95.0000,160.0000) ellipse (0.0897cm and 0.0897cm);
    \path[draw=black,draw opacity=0.700,line width=0.800pt] (95.0000,160.0000)
      ellipse (0.0897cm and 0.0897cm);
    \path[fill=ca3b3cc] (179.0000,99.0000) ellipse (0.0897cm and 0.0897cm);
    \path[draw=black,draw opacity=0.700,line width=0.800pt] (179.0000,99.0000)
      ellipse (0.0897cm and 0.0897cm);
    \path[fill=ca3b3cc] (349.0000,133.0000) ellipse (0.0897cm and 0.0897cm);
    \path[draw=black,draw opacity=0.700,line width=0.800pt] (349.0000,133.0000)
      ellipse (0.0897cm and 0.0897cm);
    \path[fill=ca3b3cc] (284.0000,332.0000) ellipse (0.0897cm and 0.0897cm);
    \path[draw=black,draw opacity=0.700,line width=0.800pt] (284.0000,332.0000)
      ellipse (0.0897cm and 0.0897cm);
    \path[fill=ca3b3cc] (75.0000,332.0000) ellipse (0.0897cm and 0.0897cm);
    \path[draw=black,draw opacity=0.700,line width=0.800pt] (75.0000,332.0000)
      ellipse (0.0897cm and 0.0897cm);
    \path[fill=ca3b3cc] (10.0000,133.0000) ellipse (0.0897cm and 0.0897cm);
    \path[draw=black,draw opacity=0.700,line width=0.800pt] (10.0000,133.0000)
      ellipse (0.0897cm and 0.0897cm);
    \path[fill=ca3b3cc] (179.0000,10.0000) ellipse (0.0897cm and 0.0897cm);
    \path[draw=black,draw opacity=0.700,line width=0.800pt] (179.0000,10.0000)
      ellipse (0.0897cm and 0.0897cm);
  \end{scope}
\end{scope}

\end{tikzpicture}

			\caption{An edge cover for Petersen graph\visible<2->{, with edges chosen being highlighted in red. Note that this is also a perfect matching.}}
		\end{figure}
	\end{example}

\end{frame}


\begin{frame}{Edge cover}
	Edge cover is related to many other problems such as:
	\begin{itemize}
		\item Matching problem.
		\item Rtw-Mon-CNF. (read twice monotone CNF)
		\item Holant problem.
		\item \dots.
	\end{itemize}
\end{frame}

\subsection{Edge Covers and Matching}
\begin{frame}{Overview}
	\tableofcontents[currentsubsection, hideothersubsections, sectionstyle=show/shaded, subsectionstyle=show/shaded]
\end{frame}
\begin{frame}{Relation to Matching}
	%The minimal edge cover could be found via a greedy algorithm based on a maximum matching.
For a graph with a perfect matching, enumerating (sampling) perfect matchings is equivalent to enumerating (sampling) minimum edge covers.
	\begin{example}[Minimum edge covers]
		Find a minimum edge cover by maximal matching?
		\begin{figure}[htp]
			\begin{subfigure}[b]{0.49\textwidth}
				\centering
						\definecolor{cb20000}{RGB}{178,0,0}
		\definecolor{c26497f}{RGB}{38,73,127}
		\definecolor{ca3b3cc}{RGB}{163,179,204}
		\begin{tikzpicture}[y=0.80pt,x=0.80pt,yscale=-1, inner sep=0pt, outer sep=0pt, scale=0.5]
		\begin{scope}[shift={(0,0)}]
		  \begin{scope}[cm={{1.0,0.0,0.0,1.0,(0.0,0.0)}}]
			\path[draw=c26497f,line width=1.600pt] (14.8956,10.3642) -- (109.6800,12.8190);
			\path<2->[draw=cb20000,line width=1.600pt] (14.8956,10.3642) -- (109.6800,12.8190);
			\path[draw=c26497f,line width=1.600pt] (12.9926,13.8534) -- (61.3126,84.7757);
			\path[draw=c26497f,line width=1.600pt] (111.6200,16.5590) -- (66.1835,84.7506);
			\path[draw=c26497f,line width=1.600pt] (118.2150,14.1720) -- (228.9470,47.0357);
			\path[draw=c26497f,line width=1.600pt] (67.7653,90.1120) -- (159.1950,129.7660);
			\path<2->[draw=cb20000,line width=1.600pt] (67.7653,90.1120) -- (159.1950,129.7660);
			\path[draw=c26497f,line width=1.600pt] (230.3220,51.6126) --
			  (165.9970,128.1630);
			\path[draw=c26497f,line width=1.600pt] (237.4180,47.5162) -- (344.1700,28.6244);
			\path<2->[draw=cb20000,line width=1.600pt] (237.4180,47.5162) -- (344.1700,28.6244);
			\path[fill=ca3b3cc] (11.0000,10.0000) ellipse (0.0948cm and 0.0948cm);
			\path[draw=black,draw opacity=0.700,line width=0.800pt] (11.0000,10.0000)
			  ellipse (0.0948cm and 0.0948cm);
			\path[fill=ca3b3cc] (114.0000,13.0000) ellipse (0.0948cm and 0.0948cm);
			\path[draw=black,draw opacity=0.700,line width=0.800pt] (114.0000,13.0000)
			  ellipse (0.0948cm and 0.0948cm);
			\path[fill=ca3b3cc] (64.0000,88.0000) ellipse (0.0948cm and 0.0948cm);
			\path[draw=black,draw opacity=0.700,line width=0.800pt] (64.0000,88.0000)
			  ellipse (0.0948cm and 0.0948cm);
			\path[fill=ca3b3cc] (233.0000,48.0000) ellipse (0.0948cm and 0.0948cm);
			\path[draw=black,draw opacity=0.700,line width=0.800pt] (233.0000,48.0000)
			  ellipse (0.0948cm and 0.0948cm);
			\path[fill=ca3b3cc] (163.0000,132.0000) ellipse (0.0948cm and 0.0948cm);
			\path[draw=black,draw opacity=0.700,line width=0.800pt] (163.0000,132.0000)
			  ellipse (0.0948cm and 0.0948cm);
			\path[fill=ca3b3cc] (348.0000,28.0000) ellipse (0.0948cm and 0.0948cm);
			\path[draw=black,draw opacity=0.700,line width=0.800pt] (348.0000,28.0000)
			  ellipse (0.0948cm and 0.0948cm);
		  \end{scope}
		\end{scope}

		\end{tikzpicture}

				\caption{$G$ has a perfect matching.}
			\end{subfigure}
			\hfill
			\begin{subfigure}[b]{0.49\textwidth}
				\centering
						\definecolor{cb20000}{RGB}{178,0,0}
		\definecolor{c26497f}{RGB}{38,73,127}
		\definecolor{ca3b3cc}{RGB}{163,179,204}
		\begin{tikzpicture}[y=0.80pt,x=0.80pt,yscale=-1, inner sep=0pt, outer sep=0pt, scale=0.4]
		\begin{scope}[shift={(0,0)}]
		  \begin{scope}[cm={{1.0,0.0,0.0,1.0,(0.0,0.0)}}]
			\path[draw=c26497f,line width=1.600pt] (343.6900,97.3106) --
			  (202.5780,170.9690);
			\path<2->[draw=cb20000,line width=1.600pt] (343.6900,97.3106) --
			  (202.5780,170.9690);
			\path[draw=c26497f,line width=1.600pt] (343.6890,92.8326) -- (202.5880,19.2258);
			\path[draw=c26497f,line width=1.600pt] (198.2880,168.3690) --
			  (198.2960,21.8271);
			\path[draw=c26497f,line width=1.600pt] (193.4500,173.3690) --
			  (15.8694,179.2680);
			\path[draw=c26497f,line width=1.600pt] (193.4600,16.8258) -- (15.8568,10.8941);
			\path<2->[draw=cb20000,line width=1.600pt] (193.4600,16.8258) -- (15.8568,10.8941);
			\path[draw=c26497f,line width=1.600pt] (11.0320,174.5890) -- (11.0201,15.5723);
			\path[fill=ca3b3cc] (348.0000,95.0000) ellipse (0.1084cm and 0.1084cm);
			\path[draw=black,draw opacity=0.700,line width=0.800pt] (348.0000,95.0000)
			  ellipse (0.1084cm and 0.1084cm);
			\path[fill=ca3b3cc] (198.0000,173.0000) ellipse (0.1084cm and 0.1084cm);
			\path[draw=black,draw opacity=0.700,line width=0.800pt] (198.0000,173.0000)
			  ellipse (0.1084cm and 0.1084cm);
			\path[fill=ca3b3cc] (198.0000,17.0000) ellipse (0.1084cm and 0.1084cm);
			\path[draw=black,draw opacity=0.700,line width=0.800pt] (198.0000,17.0000)
			  ellipse (0.1084cm and 0.1084cm);
			\path[fill=ca3b3cc] (11.0000,179.0000) ellipse (0.1084cm and 0.1084cm);
			\path<3->[fill=cb20000] (11.0000,179.0000) ellipse (0.2084cm and 0.2084cm);
			\path[draw=black,draw opacity=0.700,line width=0.800pt] (11.0000,179.0000)
			  ellipse (0.1084cm and 0.1084cm);
			\path[fill=ca3b3cc] (11.0000,11.0000) ellipse (0.1084cm and 0.1084cm);
			\path[draw=black,draw opacity=0.700,line width=0.800pt] (11.0000,11.0000)
			  ellipse (0.1084cm and 0.1084cm);
		  \end{scope}
		\end{scope}
		\end{tikzpicture}


				\caption{$G$ doesn't have a perfect matching.}
			\end{subfigure}
		\end{figure}
	\end{example}
\end{frame}

\subsection{Edge Covers and Rtw-Mon-CNF Formulae}
\begin{frame}{Overview}
	\tableofcontents[currentsubsection, hideothersubsections, sectionstyle=show/shaded, subsectionstyle=show/shaded]
\end{frame}

\begin{frame}{Relation to Rtw-Mon-CNF}
    \begin{definition}
        A formula is {\bf read twice} if every variables appears at most twice.

        A formula is {\bf monotone} if every variables appears positively.
    \end{definition}
    \pause
	Consider the following Rtw-Mon-CNF formula, its satisfying assignments are exactly edge covers of its graph representation, where we write edges as variables, and vertices as clauses.
	\[
		\phi = (e_1 \vee e_2 \vee e_3) \wedge (e_1 \vee e_4) \wedge (e_4 \vee e_5 \vee e_2 ) \wedge (e_3 \vee e_5).
	\]

	\begin{figure}[htp]
		\centering
		
\definecolor{c26497f}{RGB}{38,73,127}
\definecolor{cff7f7f}{RGB}{255,127,127}
\definecolor{ca3b3cc}{RGB}{163,179,204}


\begin{tikzpicture}[y=0.80pt,x=0.80pt,yscale=-1, inner sep=0pt, outer sep=0pt, scale=0.6]
\begin{scope}[shift={(0,0)}]
  \begin{scope}[cm={{1.0,0.0,0.0,1.0,(0.0,0.0)}}]
    \path[draw=c26497f,draw opacity=0.700,line width=1.600pt] (175.9690,14.4163) --
      (14.7035,175.6820);
  \end{scope}
  \begin{scope}[cm={{1.0,0.0,0.0,1.0,(99.33646,81.04926)}}]
    \path[fill=cff7f7f] (0,21) node[above left,minimum size=20pt] () {$e_1$};
  \end{scope}
  \begin{scope}[cm={{1.0,0.0,0.0,1.0,(0.0,0.0)}}]
    \path[draw=c26497f,draw opacity=0.700,line width=1.600pt] (179.5000,15.8786) --
      (179.5000,174.2200);
  \end{scope}
  \begin{scope}[cm={{1.0,0.0,0.0,1.0,(170.5,81.04926)}}]
    \path[fill=cff7f7f] (0,21) node[above right,minimum size=20pt] () {$e_2$};
  \end{scope}
  \begin{scope}[cm={{1.0,0.0,0.0,1.0,(0.0,0.0)}}]
    \path[draw=c26497f,draw opacity=0.700,line width=1.600pt] (183.0310,14.4163) --
      (344.2970,175.6820);
  \end{scope}
  \begin{scope}[cm={{1.0,0.0,0.0,1.0,(254.16354,81.04926)}}]
    \path[fill=cff7f7f] (0,21) node[above right,minimum size=20pt] () {$e_3$};
  \end{scope}
  \begin{scope}[cm={{1.0,0.0,0.0,1.0,(0.0,0.0)}}]
    \path[draw=c26497f,draw opacity=0.700,line width=1.600pt] (16.1658,179.2130) --
      (174.5070,179.2130);
  \end{scope}
  \begin{scope}[cm={{1.0,0.0,0.0,1.0,(86.33646,165.2128)}}]
    \path[fill=cff7f7f] (0,21) node[above right,minimum size=20pt] () {$e_4$};
  \end{scope}
  \begin{scope}[cm={{1.0,0.0,0.0,1.0,(0.0,0.0)}}]
    \path[draw=c26497f,draw opacity=0.700,line width=1.600pt] (184.4930,179.2130) --
      (342.8340,179.2130);
  \end{scope}
  \begin{scope}[cm={{1.0,0.0,0.0,1.0,(254.66354,165.2128)}}]
    \path[fill=cff7f7f] (0,21) node[above right,minimum size=20pt] () {$e_5$};
  \end{scope}
  \begin{scope}[cm={{1.0,0.0,0.0,1.0,(0.0,0.0)}}]
    \path[fill=ca3b3cc] (180.0000,11.0000) ellipse (0.1127cm and 0.1127cm);
    \path[draw=black,draw opacity=0.700,line width=0.800pt] (180.0000,11.0000)
      ellipse (0.1127cm and 0.1127cm);
    \path[fill=ca3b3cc] (11.0000,179.0000) ellipse (0.1127cm and 0.1127cm);
    \path[draw=black,draw opacity=0.700,line width=0.800pt] (11.0000,179.0000)
      ellipse (0.1127cm and 0.1127cm);
    \path[fill=ca3b3cc] (180.0000,179.0000) ellipse (0.1127cm and 0.1127cm);
    \path[draw=black,draw opacity=0.700,line width=0.800pt] (180.0000,179.0000)
      ellipse (0.1127cm and 0.1127cm);
    \path[fill=ca3b3cc] (348.0000,179.0000) ellipse (0.1127cm and 0.1127cm);
    \path[draw=black,draw opacity=0.700,line width=0.800pt] (348.0000,179.0000)
      ellipse (0.1127cm and 0.1127cm);
  \end{scope}
\end{scope}

\end{tikzpicture}

		\caption{Graph representation for $\phi$.}
	\end{figure}
\end{frame}

\subsection{Counting Problems}
\begin{frame}{Overview}
	\tableofcontents[currentsubsection, hideothersubsections, sectionstyle=show/shaded, subsectionstyle=show/shaded]
\end{frame}
\begin{frame}{Counting Problems}
	A list of problems in their search, optimization, and counting versions.
	\bigskip

	\begin{minipage}[tb]{0.3\linewidth}
		{\bf Search problems:}
		\begin{itemize}
			\item SAT.
			\item Find a (perfect) matching.
			\item Find an edge cover.
			\item \dots.
		\end{itemize}
	\end{minipage}
	\pause
	\begin{minipage}[tb]{0.3\linewidth}
		{\bf Optimizations:}
		\begin{itemize}
			\item MAX-SAT.
			\item Find a maximum matching.
			\item Find a minimum edge cover.
			\item \dots.
		\end{itemize}
	\end{minipage}
	\pause
	\begin{minipage}[tb]{0.3\linewidth}
		{\bf Counting problems:}
		\begin{itemize}
			\item \#SAT.
			\item Counting matchings.
			\item\alert{Counting edge covers}.
			\item \dots.
		\end{itemize}
	\end{minipage}
\end{frame}

\begin{frame}{Why counting?}
Besides theoretical computer science, counting problems are also related to many problems from other discipline such as:
\begin{itemize}
	\item Partition function in statistical physics.
		\pause
	\item Graph polynomials.
		\pause
	\item Sampling, learning and inference.
		\pause
    \item Pricing in combinatorial prediction markets.
	\item Query evaluations of probabilistic database.
	\item \dots.
\end{itemize}
\end{frame}

\begin{frame}
	\frametitle{Approximate Counting}
	Many interesting problems in the exact counting regimes, including counting edge cover, is hard (\#P-complete).
	Instead we look for these two types of polynomial time approximation scheme:
	\pause
	\begin{definition}[FPTAS]
		For given parameter $\eps > 0$ and an instance of a particular problem class, if the algorithm outputs a number $\hat{N}$ such that 
\[(1-\eps) N \leq \hat{N} \leq (1+\eps) N,\]
 where $N$ is the accurate answer of the problem instance, and the running time is bounded by $poly(n, 1/ \eps)$ with $n$ being the size of instance, this is called the {\bf FPTAS (fully polynomial time approximation scheme)}.

	\end{definition}
	%\pause
	%\begin{definition}[FPRAS]
		A randomized relaxation of FPTAS is known as {\bf FPRAS (fully polynomial time randomized approximation scheme)}, which uses random bits and only outputs $\hat{N}$ to the desired precision with high probability.
	%\end{definition}
\end{frame}

%\begin{frame}
%	\frametitle{MCMC vs. correlation decay}
%	
%\end{frame}


\section{\scshape Main Result}

\subsection{Previous Results and Our Result}
\begin{frame}{Overview}
	\tableofcontents[currentsubsection, hideothersubsections, sectionstyle=show/shaded, subsectionstyle=show/shaded]
\end{frame}
\begin{frame}{Previous Results}
	\begin{itemize}
		\item For counting edge covers, only an MCMC-based FPRAS is known graphs with maximum degree $3$.
			\pause
		\item For \#DNF and counting matchings, only FPRAS is known.
			\pause
		\item For counting perfect matchings, it's still open whether or not it admits FPRAS (or FPTAS).
			\pause
		\item For anti-ferromagnetic 2-spins systems (e.g. counting independent sets), an FPTAS is known, and it's correlation decay based, and goes beyond the best known MCMC based FPRAS and achieves the boundary of approximability.
	\end{itemize}

\end{frame}
\begin{frame}{Our Result}
	\begin{itemize}
		\item An FPTAS for counting edge covers in \emph{general} graphs.
		\item Our algorithm is correlation decay based.
		\item This provides another example where the tractable range of correlation decay based FPTAS exceeds the sampling based FPRAS.
	\end{itemize}

\end{frame}

\begin{frame}{Proof Sketch}
Let $EC(G)$ be the set of edge covers.
Here is an overall work-flow:
\begin{itemize}
  \item Relate $\abs{EC(G)}$ with a marginal probability $P(G,e)$.
  \item Derive a computation tree recursion for $P(G,e)$.
  \item $P(G,e,L)$: Truncate the tree at depth $L$ for some notion of depth.
  \item{ Show exponential correlation decay with respect to that tree depth:
  \[
    \abs{ P(G,e,L) - P(G,e) } \leq exp(-\Omega(L))
  \]}
\end{itemize}
\end{frame}

\begin{frame}{Devising sub-problems} % Here we introduce the only concept that you need to keep in mind.

\begin{definition}[Dangling edge]
	A {\bf dangling edge} $e=(u,\_)$ of a graph is such singleton edge with exactly one end-point vertex $u$, as shown in the Figure \ref{fig:G}.
\[ G - e \triangleq \left(V, E \setminus e\right) \]
\[ G - u \triangleq \left(V \setminus u, E - u\right) \]
\end{definition}

\begin{figure}[htp]
	\begin{subfigure}[b]{0.15\textwidth}
		\centering
		\setlength{\unitlength}{0.8mm}
		\begin{picture}(20,20)
			\put(0,0){\circle*{6}}
			\put(0,0){\line(1,1){10}}
			\put(20,0){\circle*{6}}
			\put(20,0){\line(-1,1){10}}
			\put(10,10){\circle*{6}}
			\put(10,10){\line(0,1){10}}
			\put(14,10){$u$}
			\put(8,15){$e$}
			\put(2,6){$e_1$}
			\put(15,5){$e_2$}
		\end{picture}
		\caption{$G$}
		\label{fig:G}
	\end{subfigure}
	\hfill
    \begin{subfigure}[b]{0.15\textwidth}
		\centering
		\setlength{\unitlength}{0.8mm}
		\begin{picture}(20,20)
			\put(0,0){\circle*{6}}
			\put(0,0){\line(0,1){10}}
			\put(20,0){\circle*{6}}
			\put(20,0){\line(-1,1){10}}
			\put(10,10){\circle*{6}}
			\put(10,10){\line(0,1){10}}
			\put(14,10){$u$}
			\put(8,15){$e$}
			\put(2,6){$e_1$}
			\put(15,5){$e_2$}
		\end{picture}
		\caption{$e_1-u$}
		\label{fig:e-u}
	\end{subfigure}
    \hfill
	\begin{subfigure}[b]{0.16\textwidth}
		\centering
		\setlength{\unitlength}{0.8mm}
		\begin{picture}(20,20)
			\put(0,0){\circle*{6}}
			\put(0,0){\line(0,1){10}}
			\put(20,0){\circle*{6}}
			\put(20,0){\line(0,1){10}}
			\put(2,6){$e_1$}
			\put(15,6){$e_2$}
		\end{picture}
		\caption{$G-e-u$}
		\label{fig:G-e-u}
	\end{subfigure}
	\caption{Dangling edges examples.}
\end{figure}

\end{frame}

\begin{frame}{Proof Sketch}
Let $EC(G)$ be the set of edge covers.
Here is an overall work-flow:
\begin{itemize}
  \item \alert{Relate $\abs{EC(G)}$ with a marginal probability $P(G,e)$.}
  \item Derive a computation tree recursion for $P(G,e)$.
  \item $P(G,e,L)$: Truncate the tree at depth $L$ for some notion of depth.
  \item{ Show exponential correlation decay with respect to that tree depth:
  \[
    \abs{ P(G,e,L) - P(G,e) } \leq exp(-\Omega(L))
  \]}
\end{itemize}

\end{frame}

\subsection{Counting via Marginal Probability}
\begin{frame}{Counting v.s. Marginal Probability}
	\begin{block}{Problem}
	Goal: estimate $\abs{EC(G)}$.
	\end{block}
	
	Let $X$ be an edge cover sampled uniformly from $EC(G)$, consider the following marginal probability:

	for an edge $e$, we write $P(G,e) \triangleq \Pr ( e \notin X)$.

	\bigskip

	Solution: estimate $P(G,e)$.
\end{frame}

\begin{frame}{Why $P(G,e)$?}
	Recall that the set of all edges $E$ is an edge cover.
	For a randomly sampled edge cover $X$, what is $\Pr (X=E)$?

	\pause
    Let $E=\set{e_i}$, and $e_i = (u_i,v_i)$.
	\begin{align*}
		\uncover<+->{\Pr (X=E) &= \frac{1}{\abs{EC(G)}} \\}
		\visible<+->{\Pr (X=E) &= \Pr ( \forall i, e_i \in X) \\}
		\visible<+->{&= \Pr (e_1 \in X) \Pr(e_2 \in X \mid e_1 \in X) \cdots\\ }
		\visible<+->{&= \prod_i \Pr( e_i \in X \mid \set{e_j}_{j=1}^{i-1} \subseteq X) \\}
		\visible<+->{&= \prod_i \left(1-P(G_i,e_i)\right), }
	\end{align*}
	\visible<.->{where $G_1 = G, G_{i+1} = G_i - e_i - u_i - v_i$.}

	\pause
	Therefore,
	\[\frac{1}{\abs{EC(G)}} = \prod_i \left(1-P(G_i,e_i)\right). \]
	
\end{frame}

% maybe add figures of normal edge examples here
\begin{frame}{Proof Sketch}
Let $EC(G)$ be the set of edge covers.
Here is an overall work-flow:
\begin{itemize}
  \item{Relate $\abs{EC(G)}$ with a marginal probability $P(G,e)$.}
  \item \alert{Derive a computation tree recursion for $P(G,e)$.}
  \item\alert{$P(G,e,L)$: Truncate the tree at depth $L$ for some notion of depth.}
  \item{ Show exponential correlation decay with respect to that tree depth:
  \[
    \abs{ P(G,e,L) - P(G,e) } \leq exp(-\Omega(L))
  \]}
\end{itemize}
\end{frame}

\subsection{Computation Tree Recursion}
\begin{frame}{Computation Tree Recursion}
	We focus on $e$ is dangling.
	\[
		P(G, e) = \frac{1-\prod_{i=1}^d P(G_i, e_i)}{2 - \prod_{i=1}^d P(G_i, e_i)}, %= \frac{1}{2} - \frac{0.5 \prod_{i=1}^d P(G_i, e_i)}{2 - \prod_{i=1}^d P(G_i, e_i)}
	\]
	where $G_1 \triangleq G - e - u$, and $G_{i+1} \triangleq G_{i} - e_{i}$.

	\pause
	Truncate with a modified recursion depth:
	\[
		P(G, e, L) =
		\begin{cases}
			\frac{1}{2}, & \hbox{ if $L \leq 0$;} \\
			\frac{1-\prod_{i=1}^d P(G_i, e_i, L - \lceil \log_6{(d+1)} \rceil)}{2 - \prod_{i=1}^d P(G_i, e_i, L - \lceil \log_6{(d+1)} \rceil)}, &\hbox{otherwise.}
		\end{cases}
	\]

\end{frame}

\begin{frame}{Proof Sketch}
Let $EC(G)$ be the set of edge covers.
Here is an overall work-flow:
\begin{itemize}
  \item{Relate $\abs{EC(G)}$ with a marginal probability $P(G,e)$.}
  \item{Derive a computation tree recursion for $P(G,e)$.}
  \item{$P(G,e,L)$: Truncate the tree at depth $L$ for some notion of depth.}
  \item\alert{ Show exponential correlation decay with respect to that tree depth:
  \[
    \abs{ P(G,e,L) - P(G,e) } \leq exp(-\Omega(L))
  \]}
\end{itemize}
\end{frame}

\subsection{Correlation Decay}
\begin{frame}{Correlation Decay}

\begin{proposition}
	Given graph $G$, edge $e$ and depth $L$,
	\[\abs{P(G,e,L) - P(G,e)} \leq 3\cdot(\frac{1}{2})^{L+1}\]
\end{proposition}
    Now we just take $L = \log_2 \left(\frac{6m}{\eps} \right)$, this gives the desired FPTAS.
\end{frame}

\begin{frame}{Conclusions and Upcoming Results}
As a conclusion, we have shown an FPTAS for counting edge covers in general graphs.
\bigskip

	  \pause
We later generalized our techniques for establishing correlation decay based FPTAS and obtained the following:
\begin{itemize}
  \item FPTAS for counting weighted edge covers. (Arbitrary edge weights).
  \item FPTAS for counting Read-5-Mon-CNF, while Read-6-Mon-CNF does not admit FPTAS unless $P=NP$.
  \item FPTAS for counting hypergraph matching in certain range (e.g. $3$D-Matching with maximum degree $4$).
  \item \dots
\end{itemize}
\end{frame}
\begin{frame}
	\begin{center}
		\Huge \scshape Thank you!
	\end{center}

	\bigskip
	\begin{center}
		\huge Q \& A.
	\end{center}
\end{frame}
\end{document}
