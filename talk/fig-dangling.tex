
\begin{figure}[htp]
	\begin{subfigure}[b]{0.3\textwidth}
		\centering
		\setlength{\unitlength}{0.8mm}
		\begin{picture}(20,20)
			\put(0,0){\circle*{6}}
			\put(0,0){\line(1,1){10}}
			\put(20,0){\circle*{6}}
			\put(20,0){\line(-1,1){10}}
			\put(10,10){\circle*{6}}
			\put(10,10){\line(0,1){10}}
			\put(14,10){$u$}
			\put(7,15){$e$}
			\put(2,6){$f$}
		\end{picture}
		\caption{$G$}
		\label{fig:G}
	\end{subfigure}
	\hfill
    \begin{subfigure}[b]{0.3\textwidth}
		\centering
		\setlength{\unitlength}{0.8mm}
		\begin{picture}(20,20)
			\put(0,0){\circle*{6}}
			\put(0,0){\line(0,1){10}}
			\put(20,0){\circle*{6}}
			\put(20,0){\line(-1,1){10}}
			\put(10,10){\circle*{6}}
			\put(10,10){\line(0,1){10}}
			\put(14,10){$u$}
			\put(7,15){$e$}
			\put(2,6){$f$}
		\end{picture}
		\caption{$f-u$}
		\label{fig:e-u}
	\end{subfigure}
    \hfill
	\begin{subfigure}[b]{0.35\textwidth}
		\centering
		\setlength{\unitlength}{0.8mm}
		\begin{picture}(20,20)
			\put(0,0){\circle*{6}}
			\put(0,0){\line(0,1){10}}
			\put(20,0){\circle*{6}}
			\put(20,0){\line(0,1){10}}
			\put(2,6){$f$}
		\end{picture}
		\caption{$G-e-u\triangleq (G-e)-u$}
		\label{fig:G-e-u}
	\end{subfigure}
	\caption{Dangling edges examples.}
\end{figure}
