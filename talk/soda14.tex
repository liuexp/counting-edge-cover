\documentclass[mathserif]{beamer}
\usepackage{indentfirst}
\usepackage{amsmath,amsthm,amssymb}
\usepackage{fancybox}
\usepackage{fancyvrb}
%\usepackage{minted}
\usepackage{color}
\usepackage{makeidx}
%\usepackage{xeCJK}
%\usepackage{fontspec}
%\usepackage{lmodern}
\usepackage{subcaption}
%\setCJKmainfont[BoldFont={Adobe Heiti Std}, ItalicFont={AR PL New Kai}]{Adobe Song Std}
\usepackage{amsmath}
\usepackage{amsfonts}
\usepackage{array}
\usepackage{gnuplot-lua-tikz}
\usepackage{cite}
\usepackage{url}
\usepackage{hyperref}
\hypersetup{colorlinks=true,linkcolor=blue,anchorcolor=green,citecolor=blue}
\usepackage{wrapfig}
%\usepackage{lettrine}
%\usepackage{abstract}

% THEOREMS -------------------------------------------------------
%\newtheorem{Thm}{Theorem}
%\newtheorem{Cor}[Thm]{Corollary}
%\newtheorem{Conj}[Thm]{Conjecture}
%\newtheorem{Lem}[Thm]{Lemma}
%\newtheorem{Prop}[Thm]{Proposition}
\newtheorem{proposition}[theorem]{Proposition}
%\newtheorem{Prob}{Problem}
%\newtheorem{Exam}{Example}
%\newtheorem{Def}[Thm]{Definition}
%\newtheorem{Rem}[theorem]{Remark}
\newtheorem{remark}[theorem]{Remark}
%\newtheorem{Not}[Thm]{Notation}
%\newtheorem*{Sol}{Solution}

% MATH -----------------------------------------------------------
\newcommand{\norm}[1]{\left\Vert#1\right\Vert}
\newcommand{\abs}[1]{\left\vert#1\right\vert}
\newcommand{\set}[1]{\left\{#1\right\}}
\newcommand{\Real}{\mathbb R}
\newcommand{\eps}{\varepsilon}
\newcommand{\To}{\longrightarrow}
\newcommand{\BX}{\mathbf{B}(X)}
\newcommand{\A}{\mathcal{A}}
\newcommand{\CommentS}[1]{}
% CODE ----------------------------------------------------------
\newcommand{\PltImg}[1]{
\begin{center}
\input{#1}
\end{center}
}

\newenvironment{code}%
{\vglue 5pt \VerbatimEnvironment\begin{Sbox}\begin{minipage}{0.9\textwidth}\begin{small}\begin{Verbatim}}%
{\end{Verbatim}\end{small}\end{minipage}\end{Sbox}\setlength{\shadowsize}{2pt}\shadowbox{\TheSbox}\vglue 5pt}


\usepackage{pgf}
%\usepackage{tikz}
%\usetikzlibrary{arrows,automata}
%\usepackage[latin1]{inputenc}
\usepackage{verbatim}
\usepackage{listings}
%\usepackage{algorithmic} %old version; we can use algorithmicx instead
\usepackage{algorithm}
\usepackage[noend]{algpseudocode} %for pseudo code, include algorithmicsx automatically

\lstloadlanguages{C++, Lisp, Haskell, Python, Mathematica, Java,bash,Gnuplot,make,Matlab,PHP,Prolog,R,Ruby,sh,SQL,TeX,XML}


\title[\scshape Counting Edge Covers]{\scshape A Simple FPTAS for Counting Edge Covers}
\author[C. Lin, J. Liu, P. Lu]{Chengyu Lin\inst{1} \and Jingcheng Liu\inst{1} \and Pinyan Lu\inst{2}}
\institute[SJTU, MSRA]{\scshape \inst{1} Shanghai Jiao Tong University \and \inst{2} Microsoft Research Asia}
\date[SODA 2014]{\scshape ACM-SIAM Symposium on Discrete Algorithms, 2014}
%\usetheme{Warsaw}
\usetheme{Madrid}
%\usetheme[numbers,totalnumber,compress,sidebarshades]{PaloAlto}
%\usetheme{Copenhagen}
%\usecolortheme{whale}
\usecolortheme{seahorse}
%\setbeamercolor{background canvas}{bg=blue!9} 
%\setbeamertemplate{theorems}[ams style]
\setbeamertemplate{navigation symbols}{}      
\setbeamercovered{transparent}

\begin{document}

\begin{frame}
	\titlepage
\end{frame}

\begin{frame}
	\frametitle{Overview}
	\tableofcontents
\end{frame}

\section{Introduction}

\begin{frame}
	\frametitle{Edge cover}
	\begin{definition}
		For an undirected input graph $G=(V,E)$, an {\bf edge cover} of $G$ is a set of edges $C$ covering all vertices.
	\end{definition}
	\begin{example}
		\begin{figure}[htp]
	\begin{subfigure}[b]{0.35\textwidth}
		\centering
		\setlength{\unitlength}{0.8mm}
		\begin{picture}(30,25)
			\put(00,10){\circle*{4}}
			\put(00,10){\line(1,-1){10}}
			\put(10,20){\circle*{4}}
			\put(10,20){\line(1,0){10}}
			\put(10,00){\circle*{4}}
			\put(10,00){\line(0,1){20}}
			\put(20,20){\circle*{4}}
			\put(20,00){\circle*{4}}
			\put(20,00){\line(0,1){20}}

			\color{red}
			\thicklines
			\put(00,10){\line(1,1){10}}
			\put(10,00){\line(1,0){10}}
			\put(20,20){\line(1,0){10}}
		\end{picture}
		\caption{$G$}
		\label{fig:G}
	\end{subfigure}
	\hfill
    \begin{subfigure}[b]{0.35\textwidth}
		\centering
		\setlength{\unitlength}{0.8mm}
		\begin{picture}(20,20)
			\put(0,0){\circle*{6}}
			\put(0,0){\line(0,1){10}}
			\put(20,0){\circle*{6}}
			\put(20,0){\line(-1,1){10}}
			\put(10,10){\circle*{6}}
			\put(10,10){\line(0,1){10}}
			\put(14,10){$u$}
			\put(8,15){$e$}
			\put(2,6){$e_1$}
			\put(15,5){$e_2$}
		\end{picture}
		\caption{$e_1-u$}
		\label{fig:e-u}
	\end{subfigure}
\end{figure}

	\end{example}
\end{frame}

\begin{frame}
	\frametitle{Approximation Schemes}
	We are interested primarily in two type of polynomial time approximation scheme:
	\pause
	\begin{definition}
		(Informally) For given parameter $\eps > 0$ and an instance of a particular problem class, if the algorithm outputs a number $\hat{N}$ such that $(1-\eps) N \leq \hat{N} \leq (1+\eps) N$, where $N$ is the accurate answer of the problem instance, and the running time is bounded by $poly(n, 1/ \eps)$ with $n$ being the size of instance, this is called the {\bf FPTAS (fully polynomial time approximation scheme)}.

		A randomized relaxation of FPTAS is known as {\bf FPRAS (fully polynomial time randomized approximation scheme)}, which uses random bits and only outputs $\hat{N}$ to the desired precision with high probability.
	\end{definition}
\end{frame}

\end{document}
