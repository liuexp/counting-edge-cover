
\section{Algorithm for Estimating Marginal Probability}

\IncMargin{1em}
\begin{algorithm}[H]
\SetKwInOut{Input}{input}\SetKwInOut{Output}{output}
\emph{ \textbf{function} $P(G, e, L):$}
\BlankLine
\Input{Graph $G$; edge $e$; Recursion depth $L$; }
\Output{Estimate of $P(G,e)$ up to depth $L$ .}
\Begin{
	\If{$L\leq0$ }{\Return{ $\frac{1}{2}$}}
	\ElseIf{$e$ is free }{
		\Return{ $\frac{1}{2}$}\;
	}
	\ElseIf{$e$ is dangling }{
		$L' \leftarrow L - \lceil \log_6{(d+1)} \rceil$\;
		%$L' \leftarrow L - \lceil d/5 \rceil$\;
		\Return{ $\frac{1-\prod_{i=1}^d P(G_i, e_i, L')}{2 - \prod_{i=1}^d P(G_i, e_i, L')}$} \;
	}
	\Else(// $e$ is normal ){
		$X \leftarrow \prod_{i=1}^{d_1} P(G_i^1, e_i, L)$\;
		$Y \leftarrow \prod_{i=1}^{d_2} P(G_i^2, f_i, L)$\;
		$Z \leftarrow \prod_{i=1}^{d_2} P(G_i^3, f_i, L)$\;
		\Return{ $1 - \cfrac{1}{2+ X\cdot Y  - X - Z }$ }\;
	}
 }
 \caption{Estimate $P(G,e)$ up to depth $L$}
\end{algorithm}
\DecMargin{1em}

We may compute the marginal probability $P(G, e)$ with the previous recursion, but
that could take recursion depth of $O(n)$ which results in exponential computation time.
So here we use a truncated computation tree for an estimate of $P(G,e)$.

We remark that the recursion depth used here is actually the so-called $M$-based depth introduced in \cite{LLY12} with $M=6$. 
%In fact $M$ could take any value as long as $M \geq 6$. %, and a slightly extended recursion depth could improve the algorithm running time for large $d$.
%Note the recursion depth used here is just a natural generalization of the so-called $M$-based depth introduced in \cite{LLY12}, we remark that it's sufficient to get an FPTAS with $M\geq 6$ using the $M$-based depth, here we show a stronger efficient algorithm via a slightly modified notion of recursion depth.

%\subsection{Analysis of the Algorithm}

Note that the normal case is invoked only once, so the algorithm keeps exploring in the third cases, until it hits the first 2 cases. We remark that an alternative view of the recursion depth is, we replace every node with degree greater than 6 with a $6$-ary branching subtree.
Now with this alternative view, it is easy to see that the nodes involved in the branching tree up to depth $L$ is at most $6^L$,
and for the initial normal edge case it involves at most $n$ subtrees, and for second-to-base-case nodes they involve at most $n$ extra base cases, 
so the algorithm $P(G,e,L)$ has running time $O(n^2 \cdot 6^L)$.
%Now let $B(L)$ be the set of vertices in the recursion tree involved, and $R(L) \triangleq \norm{B(L)}$,
% by the recursion on $P(G,e,L)$ in the third case we have the recursive relation for $R(L)$,
%
% TODO structural induction.

 % Note that the corner case when $e$ is free so this is \leq rather than =
% \begin{align*}
%	 %R(L) \leq d R(L-d/5) , L > 0\\
%	 R(L) \leq d R(L-\lceil \log_6{(d+1)} \rceil) , L > 0\\
%	 R(L) = 1, L\leq 0
% \end{align*}

 %Therefore we conclude that $R(L) \leq d^{1+\frac{L}{\log_6{(d+1)}}} \leq d\cdot 6^L$, in other words, the running time of the above algorithm with recursion depth $L$ is at most $O(n\cdot 6^L)$.
